We started before six o'clock the next morning, as we were still
twenty-five miles from our resting place, and Dick wanted to be there
before dusk. The journey was pleasant, though to those who do not know
the upper Thames, there is little to say about it. Ellen and I were once
more together in her boat, though Dick, for fairness' sake, was for
having me in his, and letting the two women scull the green toy. Ellen,
however, would not allow this, but claimed me as the interesting person
of the company. "After having come so far," said she, "I will not be put
off with a companion who will be always thinking of somebody else than
me: the guest is the only person who can amuse me properly. I mean that
really," said she, turning to me, "and have not said it merely as a
pretty saying."

Clara blushed and looked very happy at all this; for I think up to this
time she had been rather frightened of Ellen. As for me I felt young
again, and strange hopes of my youth were mingling with the pleasure of
the present; almost destroying it, and quickening it into something like
pain.

As we passed through the short and winding reaches of the now quickly
lessening stream, Ellen said: "How pleasant this little river is to me,
who am used to a great wide wash of water; it almost seems as if we
shall have to stop at every reach-end. I expect before I get home this
evening I shall have realised what a little country England is, since we
can so soon get to the end of its biggest river."

"It is not big," said I, "but it is pretty."

"Yes," she said, "and don't you find it difficult to imagine the times
when this little pretty country was treated by its folk as if it had
been an ugly characterless waste, with no delicate beauty to be guarded,
with no heed taken of the ever fresh pleasure of the recurring seasons,
and changeful weather, and diverse quality of the soil, and so forth?
How could people be so cruel to themselves?"

"And to each other," said I. Then a sudden resolution took hold of me,
and I said: "Dear neighbour, I may as well tell you at once that I find
it easier to imagine all that ugly past than you do, because I myself
have been part of it. I see both that you have divined something of this
in me; and also I think you will believe me when I tell you of it, so
that I am going to hide nothing from you at all."

She was silent a little, and then she said: "My friend, you have guessed
right about me; and to tell you the truth I have followed you up from
Runnymede in order that I might ask you many questions, and because I
saw that you were not one of us; and that interested and pleased me, and
I wanted to make you as happy as you could be. To say the truth, there
was a risk in it," said she, blushing---"I mean as to Dick and Clara;
for I must tell you, since we are going to be such close friends, that
even amongst us, where there are so many beautiful women, I have often
troubled men's minds disastrously. That is one reason why I was living
alone with my father in the cottage at Runnymede. But it did not answer
on that score; for of course people came there, as the place is not a
desert, and they seemed to find me all the more interesting for living
alone like that, and fell to making stories of me to themselves---like I
know you did, my friend. Well, let that pass. This evening, or to-morrow
morning, I shall make a proposal to you to do something which would
please me very much, and I think would not hurt you."

I broke in eagerly, saying that I would do anything in the world for
her; for indeed, in spite of my years and the too obvious signs of them
(though that feeling of renewed youth was not a mere passing sensation,
I think)---in spite of my years, I say, I felt altogether too happy in
the company of this delightful girl, and was prepared to take her
confidences for more than they meant perhaps.

She laughed now, but looked very kindly on me. "Well," she said,
"meantime for the present we will let it be; for I must look at this new
country that we are passing through. See how the river has changed
character again: it is broad now, and the reaches are long and very
slow-running. And look, there is a ferry!"

I told her the name of it, as I slowed off to put the ferry-chain over
our heads; and on we went passing by a bank clad with oak trees on our
left hand, till the stream narrowed again and deepened, and we rowed on
between walls of tall reeds, whose population of reed sparrows and
warblers were delightfully restless, twittering and chuckling as the
wash of the boats stirred the reeds from the water upwards in the still,
hot morning.

She smiled with pleasure, and her lazy enjoyment of the new scene seemed
to bring out her beauty doubly as she leaned back amidst the cushions,
though she was far from languid; her idleness being the idleness of a
person, strong and well-knit both in body and mind, deliberately
resting.

"Look!" she said, springing up suddenly from her place without any
obvious effort, and balancing herself with exquisite grace and ease;
"look at the beautiful old bridge ahead!"

"I need scarcely look at that," said I, not turning my head away from
her beauty. "I know what it is; though" (with a smile) "we used not to
call it the Old Bridge time agone."

She looked down upon me kindly, and said, "How well we get on now you
are no longer on your guard against me!"

And she stood looking thoughtfully at me still, till she had to sit down
as we passed under the middle one of the row of little pointed arches of
the oldest bridge across the Thames.

"O the beautiful fields!" she said; "I had no idea of the charm of a
very small river like this. The smallness of the scale of everything,
the short reaches, and the speedy change of the banks, give one a
feeling of going somewhere, of coming to something strange, a feeling of
adventure which I have not felt in bigger waters."

I looked up at her delightedly; for her voice, saying the very thing
which I was thinking, was like a caress to me. She caught my eye and her
cheeks reddened under their tan, and she said simply:

"I must tell you, my friend, that when my father leaves the Thames this
summer he will take me away to a place near the Roman wall in
Cumberland; so that this voyage of mine is farewell to the south; of
course with my goodwill in a way; and yet I am sorry for it. I hadn't
the heart to tell Dick yesterday that we were as good as gone from the
Thames-side; but somehow to you I must needs tell it."

She stopped and seemed very thoughtful for awhile, and then said
smiling:

"I must say that I don't like moving about from one home to another; one
gets so pleasantly used to all the detail of the life about one; it fits
so harmoniously and happily into one's own life, that beginning again,
even in a small way, is a kind of pain. But I daresay in the country
which you come from, you would think this petty and unadventurous, and
would think the worse of me for it."

She smiled at me caressingly as she spoke, and I made haste to answer:
"O, no, indeed; again you echo my very thoughts. But I hardly expected
to hear you speak so. I gathered from all I have heard that there was a
great deal of changing of abode amongst you in this country."

"Well," she said, "of course people are free to move about; but except
for pleasure-parties, especially in harvest and hay-time, like this of
ours, I don't think they do so much. I admit that I also have other
moods than that of stay-at-home, as I hinted just now, and I should like
to go with you all through the west country---thinking of nothing,"
concluded she smiling.

"I should have plenty to think of," said I.
