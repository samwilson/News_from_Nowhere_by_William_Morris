Dick broke the silence at last, saying: "Guest, forgive us for a little
after-dinner dulness. What would you like to do? Shall we have out
Greylocks and trot back to Hammersmith? or will you come with us and
hear some Welsh folk sing in a hall close by here? or would you like
presently to come with me into the City and see some really fine
building? or---what shall it be?"

"Well," said I, "as I am a stranger, I must let you choose for me."

In point of fact, I did not by any means want to be 'amused' just then;
and also I rather felt as if the old man, with his knowledge of past
times, and even a kind of inverted sympathy for them caused by his
active hatred of them, was as it were a blanket for me against the cold
of this very new world, where I was, so to say, stripped bare of every
habitual thought and way of acting; and I did not want to leave him too
soon. He came to my rescue at once, and said---

"Wait a bit, Dick; there is someone else to be consulted besides you and
the guest here, and that is I. I am not going to lose the pleasure of
his company just now, especially as I know he has something else to ask
me. So go to your Welshmen, by all means; but first of all bring us
another bottle of wine to this nook, and then be off as soon as you
like; and come again and fetch our friend to go westward, but not too
soon."

Dick nodded smilingly, and the old man and I were soon alone in the
great hall, the afternoon sun gleaming on the red wine in our tall
quaint-shaped glasses. Then said Hammond:

"Does anything especially puzzle you about our way of living, now you
have heard a good deal and seen a little of it?"

Said I: "I think what puzzles me most is how it all came about."

"It well may," said he, "so great as the change is. It would be
difficult indeed to tell you the whole story, perhaps impossible:
knowledge, discontent, treachery, disappointment, ruin, misery,
despair---those who worked for the change because they could see further
than other people went through all these phases of suffering; and
doubtless all the time the most of men looked on, not knowing what was
doing, thinking it all a matter of course, like the rising and setting
of the sun---and indeed it was so."

"Tell me one thing, if you can," said I. "Did the change, the
'revolution' it used to be called, come peacefully?"

"Peacefully?" said he; "what peace was there amongst those poor confused
wretches of the nineteenth century? It was war from beginning to end:
bitter war, till hope and pleasure put an end to it."

"Do you mean actual fighting with weapons?" said I, "or the strikes and
lock-outs and starvation of which we have heard?"

"Both, both," he said. "As a matter of fact, the history of the terrible
period of transition from commercial slavery to freedom may thus be
summarised. When the hope of realising a communal condition of life for
all men arose, quite late in the nineteenth century, the power of the
middle classes, the then tyrants of society, was so enormous and
crushing, that to almost all men, even those who had, you may say
despite themselves, despite their reason and judgment, conceived such
hopes, it seemed a dream. So much was this the case that some of those
more enlightened men who were then called Socialists, although they well
knew, and even stated in public, that the only reasonable condition of
Society was that of pure Communism (such as you now see around you), yet
shrunk from what seemed to them the barren task of preaching the
realisation of a happy dream. Looking back now, we can see that the
great motive-power of the change was a longing for freedom and equality,
akin if you please to the unreasonable passion of the lover; a sickness
of heart that rejected with loathing the aimless solitary life of the
well-to-do educated man of that time: phrases, my dear friend, which
have lost their meaning to us of the present day; so far removed we are
from the dreadful facts which they represent.

"Well, these men, though conscious of this feeling, had no faith in it,
as a means of bringing about the change. Nor was that wonderful: for
looking around them they saw the huge mass of the oppressed classes too
much burdened with the misery of their lives, and too much overwhelmed
by the selfishness of misery, to be able to form a conception of any
escape from it except by the ordinary way prescribed by the system of
slavery under which they lived; which was nothing more than a remote
chance of climbing out of the oppressed into the oppressing class.

"Therefore, though they knew that the only reasonable aim for those who
would better the world was a condition of equality; in their impatience
and despair they managed to convince themselves that if they could by
hook or by crook get the machinery of production and the management of
property so altered that the 'lower classes' (so the horrible word ran)
might have their slavery somewhat ameliorated, they would be ready to
fit into this machinery, and would use it for bettering their condition
still more and still more, until at last the result would be a practical
equality (they were very fond of using the word 'practical'), because
'the rich' would be forced to pay so much for keeping 'the poor' in a
tolerable condition that the condition of riches would become no longer
valuable and would gradually die out. Do you follow me?"

"Partly," said I. "Go on."

Said old Hammond: "Well, since you follow me, you will see that as a
theory this was not altogether unreasonable; but 'practically,' it
turned out a failure."

"How so?" said I.

"Well, don't you see," said he, "because it involved the making of a
machinery by those who didn't know what they wanted the machines to do.
So far as the masses of the oppressed class furthered this scheme of
improvement, they did it to get themselves improved slave-rations---as
many of them as could. And if those classes had really been incapable of
being touched by that instinct which produced the passion for freedom
and equality aforesaid, what would have happened, I think, would have
been this: that a certain part of the working classes would have been so
far improved in condition that they would have approached the condition
of the middling rich men; but below them would have been a great class
of most miserable slaves, whose slavery would have been far more
hopeless than the older class-slavery had been."

"What stood in the way of this?" said I.

"Why, of course," said he, "just that instinct for freedom aforesaid. It
is true that the slave-class could not conceive the happiness of a free
life. Yet they grew to understand (and very speedily too) that they were
oppressed by their masters, and they assumed, you see how justly, that
they could do without them, though perhaps they scarce knew how; so that
it came to this, that though they could not look forward to the
happiness or peace of the freeman, they did at least look forward to the
war which a vague hope told them would bring that peace about."

"Could you tell me rather more closely what actually took place?" said
I; for I thought \emph{him} rather vague here.

"Yes," he said, "I can. That machinery of life for the use of people who
didn't know what they wanted of it, and which was known at the time as
State Socialism, was partly put in motion, though in a very piecemeal
way. But it did not work smoothly; it was, of course, resisted at every
turn by the capitalists; and no wonder, for it tended more and more to
upset the commercial system I have told you of; without providing
anything really effective in its place. The result was growing
confusion, great suffering amongst the working classes, and, as a
consequence, great discontent. For a long time matters went on like
this. The power of the upper classes had lessened, as their command over
wealth lessened, and they could not carry things wholly by the high hand
as they had been used to in earlier days. So far the State Socialists
were justified by the result. On the other hand, the working classes
were ill-organised, and growing poorer in reality, in spite of the gains
(also real in the long run) which they had forced from the masters. Thus
matters hung in the balance; the masters could not reduce their slaves
to complete subjection, though they put down some feeble and partial
riots easily enough. The workers forced their masters to grant them
ameliorations, real or imaginary, of their condition, but could not
force freedom from them. At last came a great crash. To explain this you
must understand that very great progress had been made amongst the
workers, though as before said but little in the direction of improved
livelihood."

I played the innocent and said: "In what direction could they improve,
if not in livelihood?"

Said he: "In the power to bring about a state of things in which
livelihood would be full, and easy to gain. They had at last learned how
to combine after a long period of mistakes and disasters. The workmen
had now a regular organization in the struggle against their masters, a
struggle which for more than half a century had been accepted as an
inevitable part of the conditions of the modern system of labour and
production. This combination had now taken the form of a federation of
all or almost all the recognised wage-paid employments, and it was by
its means that those betterments of the conditions of the workmen had
been forced from the masters: and though they were not seldom mixed up
with the rioting that happened, especially in the earlier days of their
organization, it by no means formed an essential part of their tactics;
indeed at the time I am now speaking of they had got to be so strong
that most commonly the mere threat of a 'strike' was enough to gain any
minor point: because they had given up the foolish tactics of the
ancient trades unions of calling out of work a part only of the workers
of such and such an industry, and supporting them while out of work on
the labour of those that remained in. By this time they had a biggish
fund of money for the support of strikes, and could stop a certain
industry altogether for a time if they so determined."

Said I: "Was there not a serious danger of such moneys being
misused---of jobbery, in fact?"

Old Hammond wriggled uneasily on his seat, and said:

"Though all this happened so long ago, I still feel the pain of mere
shame when I have to tell you that it was more than a danger: that such
rascality often happened; indeed more than once the whole combination
seemed dropping to pieces because of it: but at the time of which I am
telling, things looked so threatening, and to the workmen at least the
necessity of their dealing with the fast-gathering trouble which the
labour-struggle had brought about, was so clear, that the conditions of
the times had begot a deep seriousness amongst all reasonable people; a
determination which put aside all non-essentials, and which to thinking
men was ominous of the swiftly-approaching change: such an element was
too dangerous for mere traitors and self-seekers, and one by one they
were thrust out and mostly joined the declared reactionaries."

"How about those ameliorations," said I; "what were they? or rather of
what nature?"

Said he: "Some of them, and these of the most practical importance to
the mens' livelihood, were yielded by the masters by direct compulsion
on the part of the men; the new conditions of labour so gained were
indeed only customary, enforced by no law: but, once established, the
masters durst not attempt to withdraw them in face of the growing power
of the combined workers. Some again were steps on the path of 'State
Socialism'; the most important of which can be speedily summed up. At
the end of the nineteenth century the cry arose for compelling the
masters to employ their men a less number of hours in the day: this cry
gathered volume quickly, and the masters had to yield to it. But it was,
of course, clear that unless this meant a higher price for work per
hour, it would be a mere nullity, and that the masters, unless forced,
would reduce it to that. Therefore after a long struggle another law was
passed fixing a minimum price for labour in the most important
industries; which again had to be supplemented by a law fixing the
maximum price on the chief wares then considered necessary for a
workman's life."

"You were getting perilously near to the late Roman poor-rates," said I,
smiling, "and the doling out of bread to the proletariat."

"So many said at the time," said the old man drily; "and it has long
been a commonplace that that slough awaits State Socialism in the end,
if it gets to the end, which as you know it did not with us. However it
went further than this minimum and maximum business, which by the by we
can now see was necessary. The government now found it imperative on
them to meet the outcry of the master class at the approaching
destruction of Commerce (as desirable, had they known it, as the
extinction of the cholera, which has since happily taken place). And
they were forced to meet it by a measure hostile to the masters, the
establishment of government factories for the production of necessary
wares, and markets for their sale. These measures taken altogether did
do something: they were in fact of the nature of regulations made by the
commander of a beleaguered city. But of course to the privileged classes
it seemed as if the end of the world were come when such laws were
enacted.

"Nor was that altogether without a warrant: the spread of communistic
theories, and the partial practice of State Socialism had at first
disturbed, and at last almost paralysed the marvellous system of
commerce under which the old world had lived so feverishly, and had
produced for some few a life of gambler's pleasure, and for many, or
most, a life of mere misery: over and over again came 'bad times' as
they were called, and indeed they were bad enough for the wage-slaves.
The year 1952 was one of the worst of these times; the workmen suffered
dreadfully: the partial, inefficient government factories, which were
terribly jobbed, all but broke down, and a vast part of the population
had for the time being to be fed on undisguised "charity" as it was
called.

"The Combined Workers watched the situation with mingled hope and
anxiety. They had already formulated their general demands; but now by a
solemn and universal vote of the whole of their federated societies,
they insisted on the first step being taken toward carrying out their
demands: this step would have led directly to handing over the
management of the whole natural resources of the country, together with
the machinery for using them into the power of the Combined Workers, and
the reduction of the privileged classes into the position of pensioners
obviously dependent on the pleasure of the workers. The 'Resolution,' as
it was called, which was widely published in the newspapers of the day,
was in fact a declaration of war, and was so accepted by the master
class. They began henceforward to prepare for a firm stand against the
'brutal and ferocious communism of the day,' as they phrased it. And as
they were in many ways still very powerful, or seemed so to be; they
still hoped by means of brute force to regain some of what they had
lost, and perhaps in the end the whole of it. It was said amongst them
on all hands that it had been a great mistake of the various governments
not to have resisted sooner; and the liberals and radicals (the name as
perhaps you may know of the more democratically inclined part of the
ruling classes) were much blamed for having led the world to this pass
by their mis-timed pedantry and foolish sentimentality: and one
Gladstone, or Gledstein (probably, judging by this name, of Scandinavian
descent), a notable politician of the nineteenth century, was especially
singled out for reprobation in this respect. I need scarcely point out
to you the absurdity of all this. But terrible tragedy lay hidden behind
this grinning through a horse-collar of the reactionary party. 'The
insatiable greed of the lower classes must be repressed'---'The people
must be taught a lesson'---these were the sacramental phrases current
amongst the reactionists, and ominous enough they were."

The old man stopped to look keenly at my attentive and wondering face;
and then said:

"I know, dear guest, that I have been using words and phrases which few
people amongst us could understand without long and laborious
explanation; and not even then perhaps. But since you have not yet gone
to sleep, and since I am speaking to you as to a being from another
planet, I may venture to ask you if you have followed me thus far?"

"O yes," said I, "I quite understand: pray go on; a great deal of what
you have been saying was common place with us---when---when---"

"Yes," said he gravely, "when you were dwelling in the other planet.
Well, now for the crash aforesaid.

"On some comparatively trifling occasion a great meeting was summoned by
the workmen leaders to meet in Trafalgar Square (about the right to meet
in which place there had for years and years been bickering). The civic
bourgeois guard (called the police) attacked the said meeting with
bludgeons, according to their custom; many people were hurt in the
\emph{melee}, of whom five in all died, either trampled to death on the
spot, or from the effects of their cudgelling; the meeting was
scattered, and some hundred of prisoners cast into gaol. A similar
meeting had been treated in the same way a few days before at a place
called Manchester, which has now disappeared. Thus the 'lesson' began.
The whole country was thrown into a ferment by this; meetings were held
which attempted some rough organisation for the holding of another
meeting to retort on the authorities. A huge crowd assembled in
Trafalgar Square and the neighbourhood (then a place of crowded
streets), and was too big for the bludgeon-armed police to cope with;
there was a good deal of dry-blow fighting; three or four of the people
were killed, and half a score of policemen were crushed to death in the
throng, and the rest got away as they could. This was a victory for the
people as far as it went. The next day all London (remember what it was
in those days) was in a state of turmoil. Many of the rich fled into the
country; the executive got together soldiery, but did not dare to use
them; and the police could not be massed in any one place, because riots
or threats of riots were everywhere. But in Manchester, where the people
were not so courageous or not so desperate as in London, several of the
popular leaders were arrested. In London a convention of leaders was got
together from the Federation of Combined Workmen, and sat under the old
revolutionary name of the Committee of Public Safety; but as they had no
drilled and armed body of men to direct, they attempted no aggressive
measures, but only placarded the walls with somewhat vague appeals to
the workmen not to allow themselves to be trampled upon. However, they
called a meeting in Trafalgar Square for the day fortnight of the
last-mentioned skirmish.

"Meantime the town grew no quieter, and business came pretty much to an
end. The newspapers---then, as always hitherto, almost entirely in the
hands of the masters---clamoured to the Government for repressive
measures; the rich citizens were enrolled as an extra body of police,
and armed with bludgeons like them; many of these were strong, well-fed,
full-blooded young men, and had plenty of stomach for fighting; but the
Government did not dare to use them, and contented itself with getting
full powers voted to it by the Parliament for suppressing any revolt,
and bringing up more and more soldiers to London. Thus passed the week
after the great meeting; almost as large a one was held on the Sunday,
which went off peaceably on the whole, as no opposition to it was
offered, and again the people cried 'victory.' But on the Monday the
people woke up to find that they were hungry. During the last few days
there had been groups of men parading the streets asking (or, if you
please, demanding) money to buy food; and what for goodwill, what for
fear, the richer people gave them a good deal. The authorities of the
parishes also (I haven't time to explain that phrase at present) gave
willy-nilly what provisions they could to wandering people; and the
Government, by means of its feeble national workshops, also fed a good
number of half-starved folk. But in addition to this, several bakers'
shops and other provision stores had been emptied without a great deal
of disturbance. So far, so good. But on the Monday in question the
Committee of Public Safety, on the one hand afraid of general
unorganised pillage, and on the other emboldened by the wavering conduct
of the authorities, sent a deputation provided with carts and all
necessary gear to clear out two or three big provision stores in the
centre of the town, leaving papers with the shop managers promising to
pay the price of them: and also in the part of the town where they were
strongest they took possession of several bakers' shops and set men at
work in them for the benefit of the people;---all of which was done with
little or no disturbance, the police assisting in keeping order at the
sack of the stores, as they would have done at a big fire.

"But at this last stroke the reactionaries were so alarmed, that they
were, determined to force the executive into action. The newspapers next
day all blazed into the fury of frightened people, and threatened the
people, the Government, and everybody they could think of, unless 'order
were at once restored.' A deputation of leading commercial people waited
on the Government and told them that if they did not at once arrest the
Committee of Public Safety, they themselves would gather a body of men,
arm them, and fall on 'the incendiaries,' as they called them.

"They, together with a number of the newspaper editors, had a long
interview with the heads of the Government and two or three military
men, the deftest in their art that the country could furnish. The
deputation came away from that interview, says a contemporary
eye-witness, smiling and satisfied, and said no more about raising an
anti-popular army, but that afternoon left London with their families
for their country seats or elsewhere.

"The next morning the Government proclaimed a state of siege in
London,---a thing common enough amongst the absolutist governments on
the Continent, but unheard-of in England in those days. They appointed
the youngest and cleverest of their generals to command the proclaimed
district; a man who had won a certain sort of reputation in the
disgraceful wars in which the country had been long engaged from time to
time. The newspapers were in ecstacies, and all the most fervent of the
reactionaries now came to the front; men who in ordinary times were
forced to keep their opinions to themselves or their immediate circle,
but who began to look forward to crushing once for all the Socialist,
and even democratic tendencies, which, said they, had been treated with
such foolish indulgence for the last sixty years.

"But the clever general took no visible action; and yet only a few of
the minor newspapers abused him; thoughtful men gathered from this that
a plot was hatching. As for the Committee of Public Safety, whatever
they thought of their position, they had now gone too far to draw back;
and many of them, it seems, thought that the government would not act.
They went on quietly organising their food supply, which was a miserable
driblet when all is said; and also as a retort to the state of siege,
they armed as many men as they could in the quarter where they were
strongest, but did not attempt to drill or organise them, thinking,
perhaps, that they could not at the best turn them into trained soldiers
till they had some breathing space. The clever general, his soldiers,
and the police did not meddle with all this in the least in the world;
and things were quieter in London that week-end; though there were riots
in many places of the provinces, which were quelled by the authorities
without much trouble. The most serious of these were at Glasgow and
Bristol.

"Well, the Sunday of the meeting came, and great crowds came to
Trafalgar Square in procession, the greater part of the Committee
amongst them, surrounded by their band of men armed somehow or other.
The streets were quite peaceful and quiet, though there were many
spectators to see the procession pass. Trafalgar Square had no body of
police in it; the people took quiet possession of it, and the meeting
began. The armed men stood round the principal platform, and there were
a few others armed amidst the general crowd; but by far the greater part
were unarmed.

"Most people thought the meeting would go off peaceably; but the members
of the Committee had heard from various quarters that something would be
attempted against them; but these rumours were vague, and they had no
idea of what threatened. They soon found out.

"For before the streets about the Square were filled, a body of soldiers
poured into it from the north-west corner and took up their places by
the houses that stood on the west side. The people growled at the sight
of the red-coats; the armed men of the Committee stood undecided, not
knowing what to do; and indeed this new influx so jammed the crowd
together that, unorganised as they were, they had little chance of
working through it. They had scarcely grasped the fact of their enemies
being there, when another column of soldiers, pouring out of the streets
which led into the great southern road going down to the Parliament
House (still existing, and called the Dung Market), and also from the
embankment by the side of the Thames, marched up, pushing the crowd into
a denser and denser mass, and formed along the south side of the Square.
Then any of those who could see what was going on, knew at once that
they were in a trap, and could only wonder what would be done with them.

"The closely-packed crowd would not or could not budge, except under the
influence of the height of terror, which was soon to be supplied to
them. A few of the armed men struggled to the front, or climbled up to
the base of the monument which then stood there, that they might face
the wall of hidden fire before them; and to most men (there were many
women amongst them) it seemed as if the end of the world had come, and
to-day seemed strangely different from yesterday. No sooner were the
soldiers drawn up aforesaid than, says an eye-witness, 'a glittering
officer on horseback came prancing out from the ranks on the south, and
read something from a paper which he held in his hand; which something,
very few heard; but I was told afterwards that it was an order for us to
disperse, and a warning that he had legal right to fire on the crowd
else, and that he would do so. The crowd took it as a challenge of some
sort, and a hoarse threatening roar went up from them; and after that
there was comparative silence for a little, till the officer had got
back into the ranks. I was near the edge of the crowd, towards the
soldiers,' says this eye-witness, 'and I saw three little machines being
wheeled out in front of the ranks, which I knew for mechanical guns. I
cried out, "Throw yourselves down! they are going to fire!" But no one
scarcely could throw himself down, so tight as the crowd were packed. I
heard a sharp order given, and wondered where I should be the next
minute; and then---It was as if---the earth had opened, and hell had
come up bodily amidst us. It is no use trying to describe the scene that
followed. Deep lanes were mowed amidst the thick crowd; the dead and
dying covered the ground, and the shrieks and wails and cries of horror
filled all the air, till it seemed as if there were nothing else in the
world but murder and death. Those of our armed men who were still unhurt
cheered wildly and opened a scattering fire on the soldiers. One or two
soldiers fell; and I saw the officers going up and down the ranks urging
the men to fire again; but they received the orders in sullen silence,
and let the butts of their guns fall. Only one sergeant ran to a
machine-gun and began to set it going; but a tall young man, an officer
too, ran out of the ranks and dragged him back by the collar; and the
soldiers stood there motionless while the horror-stricken crowd, nearly
wholly unarmed (for most of the armed men had fallen in that first
discharge), drifted out of the Square. I was told afterwards that the
soldiers on the west side had fired also, and done their part of the
slaughter. How I got out of the Square I scarcely know: I went, not
feeling the ground under me, what with rage and terror and despair.'

"So says our eye-witness. The number of the slain on the side of the
people in that shooting during a minute was prodigious; but it was not
easy to come at the truth about it; it was probably between one and two
thousand. Of the soldiers, six were killed outright, and a dozen
wounded."

I listened, trembling with excitement. The old man's eyes glittered and
his face flushed as he spoke, and told the tale of what I had often
thought might happen. Yet I wondered that he should have got so elated
about a mere massacre, and I said:

"How fearful! And I suppose that this massacre put an end to the whole
revolution for that time?"

"No, no," cried old Hammond; "it began it!"

He filled his glass and mine, and stood up and cried out, "Drink this
glass to the memory of those who died there, for indeed it would be a
long tale to tell how much we owe them."

I drank, and he sat down again and went on.

"That massacre of Trafalgar Square began the civil war, though, like all
such events, it gathered head slowly, and people scarcely knew what a
crisis they were acting in.

"Terrible as the massacre was, and hideous and overpowering as the first
terror had been, when the people had time to think about it, their
feeling was one of anger rather than fear; although the military
organisation of the state of siege was now carried out without shrinking
by the clever young general. For though the ruling-classes when the news
spread next morning felt one gasp of horror and even dread, yet the
Government and their immediate backers felt that now the wine was drawn
and must be drunk. However, even the most reactionary of the capitalist
papers, with two exceptions, stunned by the tremendous news, simply gave
an account of what had taken place, without making any comment upon it.
The exceptions were one, a so-called 'liberal' paper (the Government of
the day was of that complexion), which, after a preamble in which it
declared its undeviating sympathy with the cause of labour, proceeded to
point out that in times of revolutionary disturbance it behoved the
Government to be just but firm, and that by far the most merciful way of
dealing with the poor madmen who were attacking the very foundations of
society (which had made them mad and poor) was to shoot them at once, so
as to stop others from drifting into a position in which they would run
a chance of being shot. In short, it praised the determined action of
the Government as the acme of human wisdom and mercy, and exulted in the
inauguration of an epoch of reasonable democracy free from the
tyrannical fads of Socialism.

"The other exception was a paper thought to be one of the most violent
opponents of democracy, and so it was; but the editor of it found his
manhood, and spoke for himself and not for his paper. In a few simple,
indignant words he asked people to consider what a society was worth
which had to be defended by the massacre of unarmed citizens, and called
on the Government to withdraw their state of siege and put the general
and his officers who fired on the people on their trial for murder. He
went further, and declared that whatever his opinion might be as to the
doctrines of the Socialists, he for one should throw in his lot with the
people, until the Government atoned for their atrocity by showing that
they were prepared to listen to the demands of men who knew what they
wanted, and whom the decrepitude of society forced into pushing their
demands in some way or other.

"Of course, this editor was immediately arrested by the military power;
but his bold words were already in the hands of the public, and produced
a great effect: so great an effect that the Government, after some
vacillation, withdrew the state of siege; though at the same time it
strengthened the military organisation and made it more stringent. Three
of the Committee of Public Safety had been slain in Trafalgar Square: of
the rest the greater part went back to their old place of meeting, and
there awaited the event calmly. They were arrested there on the Monday
morning, and would have been shot at once by the general, who was a mere
military machine, if the Government had not shrunk before the
responsibility of killing men without any trial. There was at first a
talk of trying them by a special commission of judges, as it was
called---\emph{i.e.}, before a set of men bound to find them guilty, and
whose business it was to do so. But with the Government the cold fit had
succeeded to the hot one; and the prisoners were brought before a jury
at the assizes. There a fresh blow awaited the Government; for in spite
of the judge's charge, which distinctly instructed the jury to find the
prisoners guilty, they were acquitted, and the jury added to their
verdict a presentment, in which they condemned the action of the
soldiery, in the queer phraseology of the day, as 'rash, unfortunate,
and unnecessary.' The Committee of Public Safety renewed its sittings,
and from thenceforth was a popular rallying-point in opposition to the
Parliament. The Government now gave way on all sides, and made a show of
yielding to the demands of the people, though there was a widespread
plot for effecting a coup d'etat set on foot between the leaders of the
two so-called opposing parties in the parliamentary faction fight. The
well-meaning part of the public was overjoyed, and thought that all
danger of a civil war was over. The victory of the people was celebrated
by huge meetings held in the parks and elsewhere, in memory of the
victims of the great massacre.

"But the measures passed for the relief of the workers, though to the
upper classes they seemed ruinously revolutionary, were not thorough
enough to give the people food and a decent life, and they had to be
supplemented by unwritten enactments without legality to back them.
Although the Government and Parliament had the law-courts, the army, and
'society' at their backs, the Committee of Public Safety began to be a
force in the country, and really represented the producing classes. It
began to improve immensely in the days which followed on the acquittal
of its members. Its old members had little administrative capacity,
though with the exception of a few self-seekers and traitors, they were
honest, courageous men, and many of them were endowed with considerable
talent of other kinds. But now that the times called for immediate
action, came forward the men capable of setting it on foot; and a new
network of workmen's associations grew up very speedily, whose avowed
single object was the tiding over of the ship of the community into a
simple condition of Communism; and as they practically undertook also
the management of the ordinary labour-war, they soon became the
mouthpiece and intermediary of the whole of the working classes; and the
manufacturing profit-grinders now found themselves powerless before this
combination; unless \emph{their} committee, Parliament, plucked up
courage to begin the civil war again, and to shoot right and left, they
were bound to yield to the demands of the men whom they employed, and
pay higher and higher wages for shorter and shorter day's work. Yet one
ally they had, and that was the rapidly approaching breakdown of the
whole system founded on the World-Market and its supply; which now
became so clear to all people, that the middle classes, shocked for the
moment into condemnation of the Government for the great massacre,
turned round nearly in a mass, and called on the Government to look to
matters, and put an end to the tyranny of the Socialist leaders.

"Thus stimulated, the reactionist plot exploded probably before it was
ripe; but this time the people and their leaders were forewarned, and,
before the reactionaries could get under way, had taken the steps they
thought necessary.

"The Liberal Government (clearly by collusion) was beaten by the
Conservatives, though the latter were nominally much in the minority.
The popular representatives in the House understood pretty well what
this meant, and after an attempt to fight the matter out by divisions in
the House of Commons, they made a protest, left the House, and came in a
body to the Committee of Public Safety: and the civil war began again in
good earnest.

"Yet its first act was not one of mere fighting. The new Tory Government
determined to act, yet durst not re-enact the state of siege, but it
sent a body of soldiers and police to arrest the Committee of Public
Safety in the lump. They made no resistance, though they might have done
so, as they had now a considerable body of men who were quite prepared
for extremities. But they were determined to try first a weapon which
they thought stronger than street fighting.

"The members of the Committee went off quietly to prison; but they had
left their soul and their organisation behind them. For they depended
not on a carefully arranged centre with all kinds of checks and
counter-checks about it, but on a huge mass of people in thorough
sympathy with the movement, bound together by a great number of links of
small centres with very simple instructions. These instructions were now
carried out.

"The next morning, when the leaders of the reaction were chuckling at
the effect which the report in the newspapers of their stroke would have
upon the public---no newspapers appeared; and it was only towards noon
that a few straggling sheets, about the size of the gazettes of the
seventeenth century, worked by policemen, soldiers, managers, and
press-writers, were dribbled through the streets. They were greedily
seized on and read; but by this time the serious part of their news was
stale, and people did not need to be told that the GENERAL STRIKE had
begun. The railways did not run, the telegraph-wires were unserved;
flesh, fish, and green stuff brought to market was allowed to lie there
still packed and perishing; the thousands of middle-class families, who
were utterly dependant for the next meal on the workers, made frantic
efforts through their more energetic members to cater for the needs of
the day, and amongst those of them who could throw off the fear of what
was to follow, there was, I am told, a certain enjoyment of this
unexpected picnic---a forecast of the days to come, in which all labour
grew pleasant.

"So passed the first day, and towards evening the Government grew quite
distracted. They had but one resource for putting down any popular
movement---to wit, mere brute-force; but there was nothing for them
against which to use their army and police: no armed bodies appeared in
the streets; the offices of the Federated Workmen were now, in
appearance, at least, turned into places for the relief of people thrown
out of work, and under the circumstances, they durst not arrest the men
engaged in such business, all the more, as even that night many quite
respectable people applied at these offices for relief, and swallowed
down the charity of the revolutionists along with their supper. So the
Government massed soldiers and police here and there---and sat still for
that night, fully expecting on the morrow some manifesto from 'the
rebels,' as they now began to be called, which would give them an
opportunity of acting in some way or another. They were disappointed.
The ordinary newspapers gave up the struggle that morning, and only one
very violent reactionary paper (called the \emph{Daily Telegraph})
attempted an appearance, and rated 'the rebels' in good set terms for
their folly and ingratitude in tearing out the bowels of their 'common
mother,' the English Nation, for the benefit of a few greedy paid
agitators, and the fools whom they were deluding. On the other hand, the
Socialist papers (of which three only, representing somewhat different
schools, were published in London) came out full to the throat of
well-printed matter. They were greedily bought by the whole public, who,
of course, like the Government, expected a manifesto in them. But they
found no word of reference to the great subject. It seemed as if their
editors had ransacked their drawers for articles which would have been
in place forty years before, under the technical name of educational
articles. Most of these were admirable and straightforward expositions
of the doctrines and practice of Socialism, free from haste and spite
and hard words, and came upon the public with a kind of May-day
freshness, amidst the worry and terror of the moment; and though the
knowing well understood that the meaning of this move in the game was
mere defiance, and a token of irreconcilable hostility to the then
rulers of society, and though, also, they were meant for nothing else by
'the rebels,' yet they really had their effect as 'educational
articles.' However, 'education' of another kind was acting upon the
public with irresistible power, and probably cleared their heads a
little.

"As to the Government, they were absolutely terrified by this act of
'boycotting' (the slang word then current for such acts of abstention).
Their counsels became wild and vacillating to the last degree: one hour
they were for giving way for the present till they could hatch another
plot; the next they all but sent an order for the arrest in the lump of
all the workmen's committees; the next they were on the point of
ordering their brisk young general to take any excuse that offered for
another massacre. But when they called to mind that the soldiery in that
'Battle' of Trafalgar Square were so daunted by the slaughter which they
had made, that they could not be got to fire a second volley, they
shrank back again from the dreadful courage necessary for carrying out
another massacre. Meantime the prisoners, brought the second time before
the magistrates under a strong escort of soldiers, were the second time
remanded.

"The strike went on this day also. The workmen's committees were
extended, and gave relief to great numbers of people, for they had
organised a considerable amount of production of food by men whom they
could depend upon. Quite a number of well-to-do people were now
compelled to seek relief of them. But another curious thing happened: a
band of young men of the upper classes armed themselves, and coolly went
marauding in the streets, taking what suited them of such eatables and
portables that they came across in the shops which had ventured to open.
This operation they carried out in Oxford Street, then a great street of
shops of all kinds. The Government, being at that hour in one of their
yielding moods, thought this a fine opportunity for showing their
impartiality in the maintenance of 'order,' and sent to arrest these
hungry rich youths; who, however, surprised the police by a valiant
resistance, so that all but three escaped. The Government did not gain
the reputation for impartiality which they expected from this move; for
they forgot that there were no evening papers; and the account of the
skirmish spread wide indeed, but in a distorted form for it was mostly
told simply as an exploit of the starving people from the East-end; and
everybody thought it was but natural for the Government to put them down
when and where they could.

"That evening the rebel prisoners were visited in their cells by
\emph{very} polite and sympathetic persons, who pointed out to them what
a suicidal course they were following, and how dangerous these extreme
courses were for the popular cause. Says one of the prisoners: 'It was
great sport comparing notes when we came out anent the attempt of the
Government to "get at" us separately in prison, and how we answered the
blandishments of the highly "intelligent and refined" persons set on to
pump us. One laughed; another told extravagant long-bow stories to the
envoy; a third held a sulky silence; a fourth damned the polite spy and
bade him hold his jaw---and that was all they got out of us.'

"So passed the second day of the great strike. It was clear to all
thinking people that the third day would bring on the crisis; for the
present suspense and ill-concealed terror was unendurable. The ruling
classes, and the middle-class non-politicians who had been their real
strength and support, were as sheep lacking a shepherd; they literally
did not know what to do.

"One thing they found they had to do: try to get the 'rebels' to do
something. So the next morning, the morning of the third day of the
strike, when the members of the Committee of Public Safety appeared
again before the magistrate, they found themselves treated with the
greatest possible courtesy---in fact, rather as envoys and ambassadors
than prisoners. In short, the magistrate had received his orders; and
with no more to do than might come of a long stupid speech, which might
have been written by Dickens in mockery, he discharged the prisoners,
who went back to their meeting-place and at once began a due sitting. It
was high time. For this third day the mass was fermenting indeed. There
was, of course, a vast number of working people who were not organised
in the least in the world; men who had been used to act as their masters
drove them, or rather as the system drove, of which their masters were a
part. That system was now falling to pieces, and the old pressure of the
master having been taken off these poor men, it seemed likely that
nothing but the mere animal necessities and passions of men would have
any hold on them, and that mere general overturn would be the result.
Doubtless this would have happened if it had not been that the huge mass
had been leavened by Socialist opinion in the first place, and in the
second by actual contact with declared Socialists, many or indeed most
of whom were members of those bodies of workmen above said.

If anything of this kind had happened some years before, when the
masters of labour were still looked upon as the natural rulers of the
people, and even the poorest and most ignorant man leaned upon them for
support, while they submitted to their fleecing, the entire break-up of
all society would have followed. But the long series of years during
which the workmen had learned to despise their rulers, had done away
with their dependence upon them, and they were now beginning to trust
(somewhat dangerously, as events proved) in the non-legal leaders whom
events had thrust forward; and though most of these were now become mere
figure-heads, their names and reputations were useful in this crisis as
a stop-gap.

"The effect of the news, therefore, of the release of the Committee gave
the Government some breathing time: for it was received with the
greatest joy by the workers, and even the well-to-do saw in it a respite
from the mere destruction which they had begun to dread, and the fear of
which most of them attributed to the weakness of the Government. As far
as the passing hour went, perhaps they were right in this."

"How do you mean?" said I. "What could the Government have done? I often
used to think that they would be helpless in such a crisis."

Said old Hammond: "Of course I don't doubt that in the long run matters
would have come about as they did. But if the Government could have
treated their army as a real army, and used them strategically as a
general would have done, looking on the people as a mere open enemy to
be shot at and dispersed wherever they turned up, they would probably
have gained the victory at the time."

"But would the soldiers have acted against the people in this way?" said
I.

Said he: "I think from all I have heard that they would have done so if
they had met bodies of men armed however badly, and however badly they
had been organised. It seems also as if before the Trafalgar Square
massacre they might as a whole have been depended upon to fire upon an
unarmed crowd, though they were much honeycombed by Socialism. The
reason for this was that they dreaded the use by apparently unarmed men
of an explosive called dynamite, of which many loud boasts were made by
the workers on the eve of these events; although it turned out to be of
little use as a material for war in the way that was expected. Of course
the officers of the soldiery fanned this fear to the utmost, so that the
rank and file probably thought on that occasion that they were being led
into a desperate battle with men who were really armed, and whose weapon
was the more dreadful, because it was concealed. After that massacre,
however, it was at all times doubtful if the regular soldiers would fire
upon an unarmed or half-armed crowd."

Said I: "The regular soldiers? Then there were other combatants against
the people?"

"Yes," said he, "we shall come to that presently."

"Certainly," I said, "you had better go on straight with your story. I
see that time is wearing."

Said Hammond: "The Government lost no time in coming to terms with the
Committee of Public Safety; for indeed they could think of nothing else
than the danger of the moment. They sent a duly accredited envoy to
treat with these men, who somehow had obtained dominion over people's
minds, while the formal rulers had no hold except over their bodies.
There is no need at present to go into the details of the truce (for
such it was) between these high contracting parties, the Government of
the empire of Great Britain and a handful of working-men (as they were
called in scorn in those days), amongst whom, indeed, were some very
capable and 'square-headed' persons, though, as aforesaid, the abler men
were not then the recognised leaders. The upshot of it was that all the
definite claims of the people had to be granted. We can now see that
most of these claims were of themselves not worth either demanding or
resisting; but they were looked on at that time as most important, and
they were at least tokens of revolt against the miserable system of life
which was then beginning to tumble to pieces. One claim, however, was of
the utmost immediate importance, and this the Government tried hard to
evade; but as they were not dealing with fools, they had to yield at
last. This was the claim of recognition and formal status for the
Committee of Public Safety, and all the associations which it fostered
under its wing. This it is clear meant two things: first, amnesty for
'the rebels,' great and small, who, without a distinct act of civil war,
could no longer be attacked; and next, a continuance of the organised
revolution. Only one point the Government could gain, and that was a
name. The dreadful revolutionary title was dropped, and the body, with
its branches, acted under the respectable name of the 'Board of
Conciliation and its local offices.' Carrying this name, it became the
leader of the people in the civil war which soon followed."

"O," said I, somewhat startled, "so the civil war went on, in spite of
all that had happened?"

"So it was," said he. "In fact, it was this very legal recognition which
made the civil war possible in the ordinary sense of war; it took the
struggle out of the element of mere massacres on one side, and endurance
plus strikes on the other."

"And can you tell me in what kind of way the war was carried on?" said
I.

"Yes" he said; "we have records and to spare of all that; and the
essence of them I can give you in a few words. As I told you, the rank
and file of the army was not to be trusted by the reactionists; but the
officers generally were prepared for anything, for they were mostly the
very stupidest men in the country. Whatever the Government might do, a
great part of the upper and middle classes were determined to set on
foot a counter revolution; for the Communism which now loomed ahead
seemed quite unendurable to them. Bands of young men, like the marauders
in the great strike of whom I told you just now, armed themselves and
drilled, and began on any opportunity or pretence to skirmish with the
people in the streets. The Government neither helped them nor put them
down, but stood by, hoping that something might come of it. These
'Friends of Order,' as they were called, had some successes at first,
and grew bolder; they got many officers of the regular army to help
them, and by their means laid hold of munitions of war of all kinds. One
part of their tactics consisted in their guarding and even garrisoning
the big factories of the period: they held at one time, for instance,
the whole of that place called Manchester which I spoke of just now. A
sort of irregular war was carried on with varied success all over the
country; and at last the Government, which at first pretended to ignore
the struggle, or treat it as mere rioting, definitely declared for 'the
Friends of Order,' and joined to their bands whatsoever of the regular
army they could get together, and made a desperate effort to overwhelm
'the rebels,' as they were now once more called, and as indeed they
called themselves.

"It was too late. All ideas of peace on a basis of compromise had
disappeared on either side. The end, it was seen clearly, must be either
absolute slavery for all but the privileged, or a system of life founded
on equality and Communism. The sloth, the hopelessness, and if I may say
so, the cowardice of the last century, had given place to the eager,
restless heroism of a declared revolutionary period. I will not say that
the people of that time foresaw the life we are leading now, but there
was a general instinct amongst them towards the essential part of that
life, and many men saw clearly beyond the desperate struggle of the day
into the peace which it was to bring about. The men of that day who were
on the side of freedom were not unhappy, I think, though they were
harassed by hopes and fears, and sometimes torn by doubts, and the
conflict of duties hard to reconcile."

"But how did the people, the revolutionists, carry on the war? What were
the elements of success on their side?"

I put this question, because I wanted to bring the old man back to the
definite history, and take him out of the musing mood so natural to an
old man.

He answered: "Well, they did not lack organisers; for the very conflict
itself, in days when, as I told you, men of any strength of mind cast
away all consideration for the ordinary business of life, developed the
necessary talent amongst them. Indeed, from all I have read and heard, I
much doubt whether, without this seemingly dreadful civil war, the due
talent for administration would have been developed amongst the working
men. Anyhow, it was there, and they soon got leaders far more than equal
to the best men amongst the reactionaries. For the rest, they had no
difficulty about the material of their army; for that revolutionary
instinct so acted on the ordinary soldier in the ranks that the greater
part, certainly the best part, of the soldiers joined the side of the
people. But the main element of their success was this, that wherever
the working people were not coerced, they worked, not for the
reactionists, but for 'the rebels.' The reactionists could get no work
done for them outside the districts where they were all-powerful: and
even in those districts they were harassed by continual risings; and in
all cases and everywhere got nothing done without obstruction and black
looks and sulkiness; so that not only were their armies quite worn out
with the difficulties which they had to meet, but the non-combatants who
were on their side were so worried and beset with hatred and a thousand
little troubles and annoyances that life became almost unendurable to
them on those terms. Not a few of them actually died of the worry; many
committed suicide. Of course, a vast number of them joined actively in
the cause of reaction, and found some solace to their misery in the
eagerness of conflict. Lastly, many thousands gave way and submitted to
'the rebels'; and as the numbers of these latter increased, it at last
became clear to all men that the cause which was once hopeless, was now
triumphant, and that the hopeless cause was that of slavery and
privilege."
