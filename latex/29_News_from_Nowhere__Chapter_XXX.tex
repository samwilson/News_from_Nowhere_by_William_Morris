On we went. In spite of my new-born excitement about Ellen, and my
gathering fear of where it would land me, I could not help taking
abundant interest in the condition of the river and its banks; all the
more as she never seemed weary of the changing picture, but looked at
every yard of flowery bank and gurgling eddy with the same kind of
affectionate interest which I myself once had so fully, as I used to
think, and perhaps had not altogether lost even in this strangely
changed society with all its wonders. Ellen seemed delighted with my
pleasure at this, that, or the other piece of carefulness in dealing
with the river: the nursing of pretty corners; the ingenuity in dealing
with difficulties of water-engineering, so that the most obviously
useful works looked beautiful and natural also. All this, I say, pleased
me hugely, and she was pleased at my pleasure---but rather puzzled too.

"You seem astonished," she said, just after we had passed a
mill\textsuperscript{\protect\hyperlink{cite_note-1}{{[}1{]}}} which
spanned all the stream save the water-way for traffic, but which was as
beautiful in its way as a Gothic cathedral---"You seem astonished at
this being so pleasant to look at."

"Yes," I said, "in a way I am; though I don't see why it should not be."

"Ah!" she said, looking at me admiringly, yet with a lurking smile in
her face, "you know all about the history of the past. Were they not
always careful about this little stream which now adds so much
pleasantness to the country side? It would always be easy to manage this
little river. Ah! I forgot, though," she said, as her eye caught mine,
"in the days we are thinking of pleasure was wholly neglected in such
matters. But how did they manage the river in the days that you---"
Lived in she was going to say; but correcting herself, said---"in the
days of which you have record?"

"They \emph{mis}managed it," quoth I. "Up to the first half of the
nineteenth century, when it was still more or less of a highway for the
country people, some care was taken of the river and its banks; and
though I don't suppose anyone troubled himself about its aspect, yet it
was trim and beautiful. But when the railways---of which no doubt you
have heard---came into power, they would not allow the people of the
country to use either the natural or artificial waterways, of which
latter there were a great many. I suppose when we get higher up we shall
see one of these; a very important one, which one of these railways
entirely closed to the public, so that they might force people to send
their goods by their private road, and so tax them as heavily as they
could."

Ellen laughed heartily. "Well," she said, "that is not stated clearly
enough in our history-books, and it is worth knowing. But certainly the
people of those days must have been a curiously lazy set. We are not
either fidgety or quarrelsome now, but if any one tried such a piece of
folly on us, we should use the said waterways, whoever gaidsaid us:
surely that would be simple enough. However, I remember other cases of
this stupidity: when I was on the Rhine two years ago, I remember they
showed us ruins of old castles, which, according to what we heard, must
have been made for pretty much the same purpose as the railways were.
But I am interrupting your history of the river: pray go on."

"It is both short and stupid enough," said I. "The river having lost its
practical or commercial value---that is, being of no use to make money
of---"

She nodded. "I understand what that queer phrase means," said she. "Go
on!"

"Well, it was utterly neglected, till at last it became a nuisance---"

"Yes," quoth Ellen, "I understand: like the railways and the robber
knights. Yes?"

"So then they turned the makeshift business on to it, and handed it over
to a body up in London, who from time to time, in order to show that
they had something to do, did some damage here and there,---cut down
trees, destroying the banks thereby; dredged the river (where it was not
needed always), and threw the dredgings on the fields so as to spoil
them; and so forth. But for the most part they practised 'masterly
inactivity,' as it was then called---that is, they drew their salaries,
and let things alone."

"Drew their salaries," she said. "I know that means that they were
allowed to take an extra lot of other people's goods for doing nothing.
And if that had been all, it really might have been worth while to let
them do so, if you couldn't find any other way of keeping them quiet;
but it seems to me that being so paid, they could not help doing
something, and that something was bound to be mischief,---because," said
she, kindling with sudden anger, "the whole business was founded on lies
and false pretensions. I don't mean only these river-guardians, but all
these master-people I have read of."

"Yes," said I, "how happy you are to have got out of the parsimony of
oppression!"

"Why do you sigh?" she said, kindly and somewhat anxiously. "You seem to
think that it will not last?"

"It will last for you," quoth I.

"But why not for you?" said she. "Surely it is for all the world; and if
your country is somewhat backward, it will come into line before long.
Or," she said quickly, "are you thinking that you must soon go back
again? I will make my proposal which I told you of at once, and so
perhaps put an end to your anxiety. I was going to propose that you
should live with us where we are going. I feel quite old friends with
you, and should be sorry to lose you." Then she smiled on me, and said:
"Do you know, I begin to suspect you of wanting to nurse a sham sorrow,
like the ridiculous characters in some of those queer old novels that I
have come across now and then."

I really had almost begun to suspect it myself, but I refused to admit
so much; so I sighed no more, but fell to giving my delightful companion
what little pieces of history I knew about the river and its
borderlands; and the time passed pleasantly enough; and between the two
of us (she was a better sculler than I was, and seemed quite tireless)
we kept up fairly well with Dick, hot as the afternoon was, and
swallowed up the way at a great rate. At last we passed under another
ancient bridge; and through meadows bordered at first with huge
elm-trees mingled with sweet chestnut of younger but very elegant
growth; and the meadows widened out so much that it seemed as if the
trees must now be on the bents only, or about the houses, except for the
growth of willows on the immediate banks; so that the wide stretch of
grass was little broken here. Dick got very much excited now, and often
stood up in the boat to cry out to us that this was such and such a
field, and so forth; and we caught fire at his enthusiasm for the
hay-field and its harvest, and pulled our best.

At last as we were passing through a reach of the river where on the
side of the towing-path was a highish bank with a thick whispering bed
of reeds before it, and on the other side a higher bank, clothed with
willows that dipped into the stream and crowned by ancient elm-trees, we
saw bright figures coming along close to the bank, as if they were
looking for something; as, indeed, they were, and we---that is, Dick and
his company---were what they were looking for. Dick lay on his oars, and
we followed his example. He gave a joyous shout to the people on the
bank, which was echoed back from it in many voices, deep and sweetly
shrill; for there were above a dozen persons, both men, women, and
children. A tall handsome woman, with black wavy hair and deep-set grey
eyes, came forward on the bank and waved her hand gracefully to us, and
said:

"Dick, my friend, we have almost had to wait for you! What excuse have
you to make for your slavish punctuality? Why didn't you take us by
surprise, and come yesterday?"

"O," said Dick, with an almost imperceptible jerk of his head toward our
boat, "we didn't want to come too quick up the water; there is so much
to see for those who have not been up here before."

"True, true," said the stately lady, for stately is the word that must
be used for her; "and we want them to get to know the wet way from the
east thoroughly well, since they must often use it now. But come ashore
at once, Dick, and you, dear neighbours; there is a break in the reeds
and a good landing-place just round the corner. We can carry up your
things, or send some of the lads after them."

"No, no," said Dick; "it is easier going by water, though it is but a
step. Besides, I want to bring my friend here to the proper place. We
will go on to the Ford; and you can talk to us from the bank as we
paddle along."

He pulled his sculls through the water, and on we went, turning a sharp
angle and going north a little. Presently we saw before us a bank of
elm-trees, which told us of a house amidst them, though I looked in vain
for the grey walls that I expected to see there. As we went, the folk on
the bank talked indeed, mingling their kind voices with the cuckoo's
song, the sweet strong whistle of the blackbirds, and the ceaseless note
of the corn-crake as he crept through the long grass of the
mowing-field; whence came waves of fragrance from the flowering clover
amidst of the ripe grass.

In a few minutes we had passed through a deep eddying pool into the
sharp stream that ran from the ford, and beached our craft on a tiny
strand of limestone-gravel, and stepped ashore into the arms of our
up-river friends, our journey done.

I disentangled myself from the merry throng, and mounting on the
cart-road that ran along the river some feet above the water, I looked
round about me. The river came down through a wide meadow on my left,
which was grey now with the ripened seeding grasses; the gleaming water
was lost presently by a turn of the bank, but over the meadow I could
see the mingled gables of a building where I knew the lock must be, and
which now seemed to combine a mill with it. A low wooded ridge bounded
the river-plain to the south and south-east, whence we had come, and a
few low houses lay about its feet and up its slope. I turned a little to
my right, and through the hawthorn sprays and long shoots of the wild
roses could see the flat country spreading out far away under the sun of
the calm evening, till something that might be called hills with a look
of sheep-pastures about them bounded it with a soft blue line. Before
me, the elm-boughs still hid most of what houses there might be in this
river-side dwelling of men; but to the right of the cart-road a few grey
buildings of the simplest kind showed here and there.

There I stood in a dreamy mood, and rubbed my eyes as if I were not
wholly awake, and half expected to see the gay-clad company of beautiful
men and women change to two or three spindle-legged back-bowed men and
haggard, hollow-eyed, ill-favoured women, who once wore down the soil of
this land with their heavy hopeless feet, from day to day, and season to
season, and year to year. But no change came as yet, and my heart
swelled with joy as I thought of all the beautiful grey villages, from
the river to the plain and the plain to the uplands, which I could
picture to myself so well, all peopled now with this happy and lovely
folk, who had cast away riches and attained to wealth.

\hypertarget{footnotesedit}{%
\subsection[{{{[}}\href{/w/index.php?title=News_from_Nowhere/Chapter_XXX\&action=edit\&section=1}{edit}{{]}}}]{\texorpdfstring{\protect\hypertarget{Footnotes}{}{Footnotes}{{{[}}\href{/w/index.php?title=News_from_Nowhere/Chapter_XXX\&action=edit\&section=1}{edit}{{]}}}}{Footnotes{[}edit{]}}}\label{footnotesedit}}

\begin{enumerate}
\tightlist
\item
  \protect\hypertarget{cite_note-1}{}{{\protect\hyperlink{cite_ref-1}{↑}}
  {I should have said that all along the Thames there were abundance of
  mills used for various purposes; none of which were in any degree
  unsightly, and many strikingly beautiful; and the gardens about them
  marvels of loveliness.}}
\end{enumerate}
