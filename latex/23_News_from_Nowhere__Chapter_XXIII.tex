Though there were no rough noises to wake me, I could not lie long abed
the next morning, where the world seemed so well awake, and, despite the
old grumbler, so happy; so I got up, and found that, early as it was,
someone had been stirring, since all was trim and in its place in the
little parlour, and the table laid for the morning meal. Nobody was
afoot in the house as then, however, so I went out a-doors, and after a
turn or two round the superabundant garden, I wandered down over the
meadow to the river-side, where lay our boat, looking quite familiar and
friendly to me. I walked up stream a little, watching the light mist
curling up from the river till the sun gained power to draw it all away;
saw the bleak speckling the water under the willow boughs, whence the
tiny flies they fed on were falling in myriads; heard the great chub
splashing here and there at some belated moth or other, and felt almost
back again in my boyhood. Then I went back again to the boat, and
loitered there a minute or two, and then walked slowly up the meadow
towards the little house. I noted now that there were four more houses
of about the same size on the slope away from the river. The meadow in
which I was going was not up for hay; but a row of flake-hurdles ran up
the slope not far from me on each side, and in the field so parted off
from ours on the left they were making hay busily by now, in the simple
fashion of the days when I was a boy. My feet turned that way
instinctively, as I wanted to see how haymakers looked in these new and
better times, and also I rather expected to see Ellen there. I came to
the hurdles and stood looking over into the hay-field, and was close to
the end of the long line of haymakers who were spreading the low ridges
to dry off the night dew. The majority of these were young women clad
much like Ellen last night, though not mostly in silk, but in light
woollen mostly gaily embroidered; the men being all clad in white
flannel embroidered in bright colours. The meadow looked like a gigantic
tulip-bed because of them. All hands were working deliberately but well
and steadily, though they were as noisy with merry talk as a grove of
autumn starlings. Half a dozen of them, men and women, came up to me and
shook hands, gave me the sele of the morning, and asked a few questions
as to whence and whither, and wishing me good luck, went back to their
work. Ellen, to my disappointment, was not amongst them, but presently I
saw a light figure come out of the hay-field higher up the slope, and
make for our house; and that was Ellen, holding a basket in her hand.
But before she had come to the garden gate, out came Dick and Clara,
who, after a minute's pause, came down to meet me, leaving Ellen in the
garden; then we three went down to the boat, talking mere morning
prattle. We stayed there a little, Dick arranging some of the matters in
her, for we had only taken up to the house such things as we thought the
dew might damage; and then we went toward the house again; but when we
came near the garden, Dick stopped us by laying a hand on my arm and
said,---

"Just look a moment."

I looked, and over the low hedge saw Ellen, shading her eyes against the
sun as she looked toward the hay-field, a light wind stirring in her
tawny hair, her eyes like light jewels amidst her sunburnt face, which
looked as if the warmth of the sun were yet in it.

"Look, guest," said Dick; "doesn't it all look like one of those very
stories out of Grimm that we were talking about up in Bloomsbury? Here
are we two lovers wandering about the world, and we have come to a fairy
garden, and there is the very fairy herself amidst of it: I wonder what
she will do for us."

Said Clara demurely, but not stiffly: "Is she a good fairy, Dick?"

"O, yes," said he; "and according to the card, she would do better, if
it were not for the gnome or wood-spirit, our grumbling friend of last
night."

We laughed at this; and I said, "I hope you see that you have left me
out of the tale."

"Well," said he, "that's true. You had better consider that you have got
the cap of darkness, and are seeing everything, yourself invisible."

That touched me on my weak side of not feeling sure of my position in
this beautiful new country; so in order not to make matters worse, I
held my tongue, and we all went into the garden and up to the house
together. I noticed by the way that Clara must really rather have felt
the contrast between herself as a town madam and this piece of the
summer country that we all admired so, for she had rather dressed after
Ellen that morning as to thinness and scantiness, and went barefoot
also, except for light sandals.

The old man greeted us kindly in the parlour, and said: "Well, guests,
so you have been looking about to search into the nakedness of the land:
I suppose your illusions of last night have given way a bit before the
morning light? Do you still like, it, eh?"

"Very much," said I, doggedly; "it is one of the prettiest places on the
lower Thames."

"Oho!" said he; "so you know the Thames, do you?"

I reddened, for I saw Dick and Clara looking at me, and scarcely knew
what to say. However, since I had said in our early intercourse with my
Hammersmith friends that I had known Epping Forest, I thought a hasty
generalisation might be better in avoiding complications than a
downright lie; so I said---

"I have been in this country before; and I have been on the Thames in
those days."

"O," said the old man, eagerly, "so you have been in this country
before. Now really, don't you \emph{find} it (apart from all theory, you
know) much changed for the worse?"

"No, not at all," said I; "I find it much changed for the better."

"Ah," quoth he, "I fear that you have been prejudiced by some theory or
another. However, of course the time when you were here before must have
been so near our own days that the deterioration might not be very
great: as then we were, of course, still living under the same customs
as we are now. I was thinking of earlier days than that."

"In short," said Clara, "you have \emph{theories} about the change which
has taken place."

"I have facts as well," said he. "Look here! from this hill you can see
just four little houses, including this one. Well, I know for certain
that in old times, even in the summer, when the leaves were thickest,
you could see from the same place six quite big and fine houses; and
higher up the water, garden joined garden right up to Windsor; and there
were big houses in all the gardens. Ah! England was an important place
in those days."

I was getting nettled, and said: "What you mean is that you
de-cockneyised the place, and sent the damned flunkies packing, and that
everybody can live comfortably and happily, and not a few damned thieves
only, who were centres of vulgarity and corruption wherever they were,
and who, as to this lovely river, destroyed its beauty morally, and had
almost destroyed it physically, when they were thrown out of it."

There was silence after this outburst, which for the life of me I could
not help, remembering how I had suffered from cockneyism and its cause
on those same waters of old time. But at last the old man said, quite
coolly:

"My dear guest, I really don't know what you mean by either cockneys, or
flunkies, or thieves, or damned; or how only a few people could live
happily and comfortably in a wealthy country. All I can see is that you
are angry, and I fear with me: so if you like we will change the
subject."

I thought this kind and hospitable in him, considering his obstinacy
about his theory; and hastened to say that I did not mean to be angry,
only emphatic. He bowed gravely, and I thought the storm was over, when
suddenly Ellen broke in:

"Grandfather, our guest is reticent from courtesy; but really what he
has in his mind to say to you ought to be said; so as I know pretty well
what it is, I will say it for him: for as you know, I have been taught
these things by people who---"

"Yes," said the old man, "by the sage of Bloomsbury, and others."

"O," said Dick, "so you know my old kinsman Hammond?"

"Yes," said she, "and other people too, as my grandfather says, and they
have taught me things: and this is the upshot of it. We live in a little
house now, not because we have nothing grander to do than working in the
fields, but because we please; for if we liked, we could go and live in
a big house amongst pleasant companions."

Grumbled the old man: "Just so! As if I would live amongst those
conceited fellows; all of them looking down upon me!"

She smiled on him kindly, but went on as if he had not spoken. "In the
past times, when those big houses of which grandfather speaks were so
plenty, we \emph{must} have lived in a cottage whether we had liked it
or not; and the said cottage, instead of having in it everything we
want, would have been bare and empty. We should not have got enough to
eat; our clothes would have been ugly to look at, dirty and frowsy. You,
grandfather, have done no hard work for years now, but wander about and
read your books and have nothing to worry you; and as for me, I work
hard when I like it, because I like it, and think it does me good, and
knits up my muscles, and makes me prettier to look at, and healthier and
happier. But in those past days you, grandfather, would have had to work
hard after you were old; and would have been always afraid of having to
be shut up in a kind of prison along with other old men, half-starved
and without amusement. And as for me, I am twenty years old. In those
days my middle age would be beginning now, and in a few years I should
be pinched, thin, and haggard, beset with troubles and miseries, so that
no one could have guessed that I was once a beautiful girl.

"Is this what you have had in your mind, guest?" said she, the tears in
her eyes at thought of the past miseries of people like herself.

"Yes," said I, much moved; "that and more. Often---in my country I have
seen that wretched change you have spoken of, from the fresh handsome
country lass to the poor draggle-tailed country woman."

The old man sat silent for a little, but presently recovered himself and
took comfort in his old phrase of "Well, you like it so, do you?"

"Yes," said Ellen, "I love life better than death."

"O, you do, do you?" said he. "Well, for my part I like reading a good
old book with plenty of fun in it, like Thackeray's 'Vanity Fair.' Why
don't you write books like that now? Ask that question of your
Bloomsbury sage."

Seeing Dick's cheeks reddening a little at this sally, and noting that
silence followed, I thought I had better do something. So I said: "I am
only the guest, friends; but I know you want to show me your river at
its best, so don't you think we had better be moving presently, as it is
certainly going to be a hot day?"
