And now again I was busy looking about me, for we were quite clear of
Piccadilly Market, and were in a region of elegantly-built much
ornamented houses, which I should have called villas if they had been
ugly and pretentious, which was very far from being the case. Each house
stood in a garden carefully cultivated, and running over with flowers.
The blackbirds were singing their best amidst the garden-trees, which,
except for a bay here and there, and occasional groups of limes, seemed
to be all fruit-trees: there were a great many cherry-trees, now all
laden with fruit; and several times as we passed by a garden we were
offered baskets of fine fruit by children and young girls. Amidst all
these gardens and houses it was of course impossible to trace the sites
of the old streets: but it seemed to me that the main roadways were the
same as of old.

We came presently into a large open space, sloping somewhat toward the
south, the sunny site of which had been taken advantage of for planting
an orchard, mainly, as I could see, of apricot-trees, in the midst of
which was a pretty gay little structure of wood, painted and gilded,
that looked like a refreshment-stall. From the southern side of the said
orchard ran a long road, chequered over with the shadow of tall old pear
trees, at the end of which showed the high tower of the Parliament
House, or Dung Market.

A strange sensation came over me; I shut my eyes to keep out the sight
of the sun glittering on this fair abode of gardens, and for a moment
there passed before them a phantasmagoria of another day. A great space
surrounded by tall ugly houses, with an ugly church at the corner and a
nondescript ugly cupolaed building at my back; the roadway thronged with
a sweltering and excited crowd, dominated by omnibuses crowded with
spectators. In the midst a paved be-fountained square, populated only by
a few men dressed in blue, and a good many singularly ugly bronze images
(one on the top of a tall column). The said square guarded up to the
edge of the roadway by a four-fold line of big men clad in blue, and
across the southern roadway the helmets of a band of horse-soldiers,
dead white in the greyness of the chilly November afternoon---I opened
my eyes to the sunlight again and looked round me, and cried out among
the whispering trees and odorous blossoms, "Trafalgar Square!"

"Yes," said Dick, who had drawn rein again, "so it is. I don't wonder at
your finding the name ridiculous: but after all, it was nobody's
business to alter it, since the name of a dead folly doesn't bite. Yet
sometimes I think we might have given it a name which would have
commemorated the great battle which was fought on the spot itself in
1952,---that was important enough, if the historians don't lie."

"Which they generally do, or at least did," said the old man. "For
instance, what can you make of this, neighbours? I have read a muddled
account in a book---O a stupid book---called James' Social Democratic
History, of a fight which took place here in or about the year 1887 (I
am bad at dates). Some people, says this story, were going to hold a
ward-mote here, or some such thing, and the Government of London, or the
Council, or the Commission, or what not other barbarous half-hatched
body of fools, fell upon these citizens (as they were then called) with
the armed hand. That seems too ridiculous to be true; but according to
this version of the story, nothing much came of it, which certainly
\emph{is} too ridiculous to be true."

"Well," quoth I, "but after all your Mr. James is right so far, and it
\emph{is} true; except that there was no fighting, merely unarmed and
peaceable people attacked by ruffians armed with bludgeons."

"And they put up with that?" said Dick, with the first unpleasant
expression I had seen on his good-tempered face.

Said I, reddening: "We \emph{had} to put up with it; we couldn't help
it."

The old man looked at me keenly, and said: "You seem to know a great
deal about it, neighbour! And is it really true that nothing came of
it?"

"This came of it," said I, "that a good many people were sent to prison
because of it."

"What, of the bludgeoners?" said the old man. "Poor devils!"

"No, no," said I, "of the bludgeoned."

Said the old man rather severely: "Friend, I expect that you have been
reading some rotten collection of lies, and have been taken in by it too
easily."

"I assure you," said I, "what I have been saying is true."

"Well, well, I am sure you think so, neighbour," said the old man, "but
I don't see why you should be so cocksure."

As I couldn't explain why, I held my tongue. Meanwhile Dick, who had
been sitting with knit brows, cogitating, spoke at last, and said gently
and rather sadly:

"How strange to think that there have been men like ourselves, and
living in this beautiful and happy country, who I suppose had feelings
and affections like ourselves, who could yet do such dreadful things."

"Yes," said I, in a didactic tone; "yet after all, even those days were
a great improvement on the days that had gone before them. Have you not
read of the Mediaeval period, and the ferocity of its criminal laws; and
how in those days men fairly seemed to have enjoyed tormenting their
fellow men?---nay, for the matter of that, they made their God a
tormentor and a jailer rather than anything else."

"Yes," said Dick, "there are good books on that period also, some of
which I have read. But as to the great improvement of the nineteenth
century, I don't see it. After all, the Mediaeval folk acted after their
conscience, as your remark about their God (which is true) shows, and
they were ready to bear what they inflicted on others; whereas the
nineteenth century ones were hypocrites, and pretended to be humane, and
yet went on tormenting those whom they dared to treat so by shutting
them up in prison, for no reason at all, except that they were what they
themselves, the prison-masters, had forced them to be. O, it's horrible
to think of!"

"But perhaps," said I, "they did not know what the prisons were like."

Dick seemed roused, and even angry. "More shame for them," said he,
"when you and I know it all these years afterwards. Look you, neighbour,
they couldn't fail to know what a disgrace a prison is to the
Commonwealth at the best, and that their prisons were a good step on
towards being at the worst."

Quoth I: "But have you no prisons at all now?"

As soon as the words were out of my mouth, I felt that I had made a
mistake, for Dick flushed red and frowned, and the old man looked
surprised and pained; and presently Dick said angrily, yet as if
restraining himself somewhat---

"Man alive! how can you ask such a question? Have I not told you that we
know what a prison means by the undoubted evidence of really trustworthy
books, helped out by our own imaginations? And haven't you specially
called me to notice that the people about the roads and streets look
happy? and how could they look happy if they knew that their neighbours
were shut up in prison, while they bore such things quietly? And if
there were people in prison, you couldn't hide it from folk, like you
may an occasional man-slaying; because that isn't done of set purpose,
with a lot of people backing up the slayer in cold blood, as this prison
business is. Prisons, indeed! O no, no, no!"

He stopped, and began to cool down, and said in a kind voice: "But
forgive me! I needn't be so hot about it, since there are \emph{not} any
prisons: I'm afraid you will think the worse of me for losing my temper.
Of course, you, coming from the outlands, cannot be expected to know
about these things. And now I'm afraid I have made you feel
uncomfortable."

In a way he had; but he was so generous in his heat, that I liked him
the better for it, and I said:

"No, really 'tis all my fault for being so stupid. Let me change the
subject, and ask you what the stately building is on our left just
showing at the end of that grove of plane-trees?"

"Ah," he said, "that is an old building built before the middle of the
twentieth century, and as you see, in a queer fantastic style not over
beautiful; but there are some fine things inside it, too, mostly
pictures, some very old. It is called the National Gallery; I have
sometimes puzzled as to what the name means: anyhow, nowadays wherever
there is a place where pictures are kept as curiosities permanently it
is called a National Gallery, perhaps after this one. Of course there
are a good many of them up and down the country."

I didn't try to enlighten him, feeling the task too heavy; but I pulled
out my magnificent pipe and fell a-smoking, and the old horse jogged on
again. As we went, I said:

"This pipe is a very elaborate toy, and you seem so reasonable in this
country, and your architecture is so good, that I rather wonder at your
turning out such trivialities."

It struck me as I spoke that this was rather ungrateful of me, after
having received such a fine present; but Dick didn't seem to notice my
bad manners, but said:

"Well, I don't know; it is a pretty thing, and since nobody need make
such things unless they like, I don't see why they shouldn't make them,
if they like. Of course, if carvers were scarce they would all be busy
on the architecture, as you call it, and then these 'toys' (a good word)
would not be made; but since there are plenty of people who can
carve---in fact, almost everybody, and as work is somewhat scarce, or we
are afraid it may be, folk do not discourage this kind of petty work."

He mused a little, and seemed somewhat perturbed; but presently his face
cleared, and he said: "After all, you must admit that the pipe is a very
pretty thing, with the little people under the trees all cut so clean
and sweet;---too elaborate for a pipe, perhaps, but---well, it is very
pretty."

"Too valuable for its use, perhaps," said I.

"What's that?" said he; "I don't understand."

I was just going in a helpless way to try to make him understand, when
we came by the gates of a big rambling building, in which work of some
sort seemed going on. "What building is that?" said I, eagerly; for it
was a pleasure amidst all these strange things to see something a little
like what I was used to: "it seems to be a factory."

"Yes," he said, "I think I know what you mean, and that's what it is;
but we don't call them factories now, but Banded-workshops: that is,
places where people collect who want to work together."

"I suppose," said I, "power of some sort is used there?"

"No, no," said he. "Why should people collect together to use power,
when they can have it at the places where they live, or hard by, any two
or three of them; or any one, for the matter of that? No; folk collect
in these Banded-workshops to do hand-work in which working together is
necessary or convenient; such work is often very pleasant. In there, for
instance, they make pottery and glass,---there, you can see the tops of
the furnaces. Well, of course it's handy to have fair-sized ovens and
kilns and glass-pots, and a good lot of things to use them for: though
of course there are a good many such places, as it would be ridiculous
if a man had a liking for pot-making or glass-blowing that he should
have to live in one place or be obliged to forego the work he liked."

"I see no smoke coming from the furnaces," said I.

"Smoke?" said Dick; "why should you see smoke?"

I held my tongue, and he went on: "It's a nice place inside, though as
plain as you see outside. As to the crafts, throwing the clay must be
jolly work: the glass-blowing is rather a sweltering job; but some folk
like it very much indeed; and I don't much wonder: there is such a sense
of power, when you have got deft in it, in dealing with the hot metal.
It makes a lot of pleasant work," said he, smiling, "for however much
care you take of such goods, break they will, one day or another, so
there is always plenty to do."

I held my tongue and pondered.

We came just here on a gang of men road-mending which delayed us a
little; but I was not sorry for it; for all I had seen hitherto seemed a
mere part of a summer holiday; and I wanted to see how this folk would
set to on a piece of real necessary work. They had been resting, and had
only just begun work again as we came up; so that the rattle of the
picks was what woke me from my musing. There were about a dozen of them,
strong young men, looking much like a boating party at Oxford would have
looked in the days I remembered, and not more troubled with their work:
their outer raiment lay on the road-side in an orderly pile under the
guardianship of a six-year-old boy, who had his arm thrown over the neck
of a big mastiff, who was as happily lazy as if the summer-day had been
made for him alone. As I eyed the pile of clothes, I could see the gleam
of gold and silk embroidery on it, and judged that some of these workmen
had tastes akin to those of the Golden Dustman of Hammersmith. Beside
them lay a good big basket that had hints about it of cold pie and wine:
a half dozen of young women stood by watching the work or the workers,
both of which were worth watching, for the latter smote great strokes
and were very deft in their labour, and as handsome clean-built fellows
as you might find a dozen of in a summer day. They were laughing and
talking merrily with each other and the women, but presently their
foreman looked up and saw our way stopped. So he stayed his pick and
sang out, "Spell ho, mates! here are neighbours want to get past."
Whereon the others stopped also, and, drawing around us, helped the old
horse by easing our wheels over the half undone road, and then, like men
with a pleasant task on hand, hurried back to their work, only stopping
to give us a smiling good-day; so that the sound of the picks broke out
again before Greylocks had taken to his jog-trot. Dick looked back over
his shoulder at them and said:

"They are in luck to-day: it's right down good sport trying how much
pick-work one can get into an hour; and I can see those neighbours know
their business well. It is not a mere matter of strength getting on
quickly with such work; is it, guest?"

"I should think not," said I, "but to tell you the truth, I have never
tried my hand at it."

"Really?" said he gravely, "that seems a pity; it is good work for
hardening the muscles, and I like it; though I admit it is pleasanter
the second week than the first. Not that I am a good hand at it: the
fellows used to chaff me at one job where I was working, I remember, and
sing out to me, 'Well rowed, stroke!' 'Put your back into it, bow!'"

"Not much of a joke," quoth I.

"Well," said Dick, "everything seems like a joke when we have a pleasant
spell of work on, and good fellows merry about us; we feels so happy,
you know." Again I pondered silently.
