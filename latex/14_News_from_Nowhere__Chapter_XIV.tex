Said I: "How about your relations with foreign nations?"

"I will not affect not to know what you mean," said he, "but I will tell
you at once that the whole system of rival and contending nations which
played so great a part in the 'government' of the world of civilisation
has disappeared along with the inequality betwixt man and man in
society."

"Does not that make the world duller?" said I.

"Why?" said the old man.

"The obliteration of national variety," said I.

"Nonsense," he said, somewhat snappishly. "Cross the water and see. You
will find plenty of variety: the landscape, the building, the diet, the
amusements, all various. The men and women varying in looks as well as
in habits of thought; the costume far more various than in the
commercial period. How should it add to the variety or dispel the
dulness, to coerce certain families or tribes, often heterogeneous and
jarring with one another, into certain artificial and mechanical groups,
and call them nations, and stimulate their patriotism---\emph{i.e.},
their foolish and envious prejudices?"

"Well---I don't know how," said I.

"That's right," said Hammond cheerily; "you can easily understand that
now we are freed from this folly it is obvious to us that by means of
this very diversity the different strains of blood in the world can be
serviceable and pleasant to each other, without in the least wanting to
rob each other: we are all bent on the same enterprise, making the most
of our lives. And I must tell you whatever quarrels or misunderstandings
arise, they very seldom take place between people of different race; and
consequently since there is less unreason in them, they are the more
readily appeased."

"Good," said I, "but as to those matters of politics; as to general
differences of opinion in one and the same community. Do you assert that
there are none?"

"No, not at all," said he, somewhat snappishly; "but I do say that
differences of opinion about real solid things need not, and with us do
not, crystallise people into parties permanently hostile to one another,
with different theories as to the build of the universe and the progress
of time. Isn't that what politics used to mean?"

"H'm, well," said I, "I am not so sure of that."

Said he: "I take, you, neighbour; they only \emph{pretended} to this
serious difference of opinion; for if it had existed they could not have
dealt together in the ordinary business of life; couldn't have eaten
together, bought and sold together, gambled together, cheated other
people together, but must have fought whenever they met: which would not
have suited them at all. The game of the masters of politics was to
cajole or force the public to pay the expense of a luxurious life and
exciting amusement for a few cliques of ambitious persons: and the
\emph{pretence} of serious difference of opinion, belied by every action
of their lives, was quite good enough for that. What has all that got to
do with us?"

Said I: "Why, nothing, I should hope. But I fear---In short, I have been
told that political strife was a necessary result of human nature."

"Human nature!" cried the old boy, impetuously; "what human nature? The
human nature of paupers, of slaves, of slave-holders, or the human
nature of wealthy freemen? Which? Come, tell me that!"

"Well," said I, "I suppose there would be a difference according to
circumstances in people's action about these matters."

"I should think so, indeed," said he. "At all events, experience shows
that it is so. Amongst us, our differences concern matters of business,
and passing events as to them, and could not divide men permanently. As
a rule, the immediate outcome shows which opinion on a given subject is
the right one; it is a matter of fact, not of speculation. For instance,
it is clearly not easy to knock up a political party on the question as
to whether haymaking in such and such a country-side shall begin this
week or next, when all men agree that it must at latest begin the week
after next, and when any man can go down into the fields himself and see
whether the seeds are ripe enough for the cutting."

Said I: "And you settle these differences, great and small, by the will
of the majority, I suppose?"

"Certainly," said he; "how else could we settle them? You see in matters
which are merely personal which do not affect the welfare of the
community---how a man shall dress, what he shall eat and drink, what he
shall write and read, and so forth---there can be no difference of
opinion, and everybody does as he pleases. But when the matter is of
common interest to the whole community, and the doing or not doing
something affects everybody, the majority must have their way; unless
the minority were to take up arms and show by force that they were the
effective or real majority; which, however, in a society of men who are
free and equal is little likely to happen; because in such a community
the apparent majority \emph{is} the real majority, and the others, as I
have hinted before, know that too well to obstruct from mere
pigheadedness; especially as they have had plenty of opportunity of
putting forward their side of the question."

"How is that managed?" said I.

"Well," said he, "let us take one of our units of management, a commune,
or a ward, or a parish (for we have all three names, indicating little
real distinction between them now, though time was there was a good
deal). In such a district, as you would call it, some neighbours think
that something ought to be done or undone: a new town-hall built; a
clearance of inconvenient houses; or say a stone bridge substituted for
some ugly old iron one,---there you have undoing and doing in one. Well,
at the next ordinary meeting of the neighbours, or Mote, as we call it,
according to the ancient tongue of the times before bureaucracy, a
neighbour proposes the change, and of course, if everybody agrees, there
is an end of discussion, except about details. Equally, if no one backs
the proposer,---'seconds him,' it used to be called---the matter drops
for the time being; a thing not likely to happen amongst reasonable men,
however, as the proposer is sure to have talked it over with others
before the Mote. But supposing the affair proposed and seconded, if a
few of the neighbours disagree to it, if they think that the beastly
iron bridge will serve a little longer and they don't want to be
bothered with building a new one just then, they don't count heads that
time, but put off the formal discussion to the next Mote; and meantime
arguments \emph{pro} and \emph{con} are flying about, and some get
printed, so that everybody knows what is going on; and when the Mote
comes together again there is a regular discussion and at last a vote by
show of hands. If the division is a close one, the question is again put
off for further discussion; if the division is a wide one, the minority
are asked if they will yield to the more general opinion, which they
often, nay, most commonly do. If they refuse, the question is debated a
third time, when, if the minority has not perceptibly grown, they always
give way; though I believe there is some half-forgotten rule by which
they might still carry it on further; but I say, what always happens is
that they are convinced, not perhaps that their view is the wrong one,
but they cannot persuade or force the community to adopt it."

"Very good," said I; "but what happens if the divisions are still
narrow?"

Said he: "As a matter of principle and according to the rule of such
cases, the question must then lapse, and the majority, if so narrow, has
to submit to sitting down under the \emph{status quo}. But I must tell
you that in point of fact the minority very seldom enforces this rule,
but generally yields in a friendly manner."

"But do you know," said I, "that there is something in all this very
like democracy; and I thought that democracy was considered to be in a
moribund condition many, many years ago."

The old boy's eyes twinkled. "I grant you that our methods have that
drawback. But what is to be done? We can't get \emph{anyone} amongst us
to complain of his not always having his own way in the teeth of the
community, when it is clear that \emph{everybody} cannot have that
indulgence. What is to be done?"

"Well," said I, "I don't know."

Said he: "The only alternatives to our method that I can conceive of are
these. First, that we should choose out, or breed, a class of superior
persons capable of judging on all matters without consulting the
neighbours; that, in short, we should get for ourselves what used to be
called an aristocracy of intellect; or, secondly, that for the purpose
of safe-guarding the freedom of the individual will, we should revert to
a system of private property again, and have slaves and slave-holders
once more. What do you think of those two expedients?"

"Well," said I, "there is a third possibility---to wit, that every man
should be quite independent of every other, and that thus the tyranny of
society should be abolished."

He looked hard at me for a second or two, and then burst out laughing
very heartily; and I confess that I joined him. When he recovered
himself he nodded at me, and said: "Yes, yes, I quite agree with
you---and so we all do."

"Yes," I said, "and besides, it does not press hardly on the minority:
for, take this matter of the bridge, no man is obliged to work on it if
he doesn't agree to its building. At least, I suppose not."

He smiled, and said: "Shrewdly put; and yet from the point of view of
the native of another planet. If the man of the minority does find his
feelings hurt, doubtless he may relieve them by refusing to help in
building the bridge. But, dear neighbour, that is not a very effective
salve for the wound caused by the 'tyranny of a majority' in our
society; because all work that is done is either beneficial or hurtful
to every member of society. The man is benefited by the bridge-building
if it turns out a good thing, and hurt by it if it turns out a bad one,
whether he puts a hand to it or not; and meanwhile he is benefiting the
bridge-builders by his work, whatever that may be. In fact, I see no
help for him except the pleasure of saying 'I told you so' if the
bridge-building turns out to be a mistake and hurts him; if it benefits
him he must suffer in silence. A terrible tyranny our Communism, is it
not? Folk used often to be warned against this very unhappiness in times
past, when for every well-fed, contented person you saw a thousand
miserable starvelings. Whereas for us, we grow fat and well-liking on
the tyranny; a tyranny, to say the truth, not to be made visible by any
microscope I know. Don't be afraid, my friend; we are not going to seek
for troubles by calling our peace and plenty and happiness by ill names
whose very meaning we have forgotten!"

He sat musing for a little, and then started and said: "Are there any
more questions, dear guest? The morning is waning fast amidst my
garrulity?"
