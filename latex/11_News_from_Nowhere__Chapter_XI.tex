"Now," said I, "I have come to the point of asking questions which I
suppose will be dry for you to answer and difficult for you to explain;
but I have foreseen for some time past that I must ask them, will I
'nill I. What kind of a government have you? Has republicanism finally
triumphed? or have you come to a mere dictatorship, which some persons
in the nineteenth century used to prophesy as the ultimate outcome of
democracy? Indeed, this last question does not seem so very
unreasonable, since you have turned your Parliament House into a
dung-market. Or where do you house your present Parliament?"

The old man answered my smile with a hearty laugh, and said: "Well,
well, dung is not the worst kind of corruption; fertility may come of
that, whereas mere dearth came from the other kind, of which those walls
once held the great supporters. Now, dear guest, let me tell you that
our present parliament would be hard to house in one place, because the
whole people is our parliament."

"I don't understand," said I.

"No, I suppose not," said he. "I must now shock you by telling you that
we have no longer anything which you, a native of another planet, would
call a government."

"I am not so much shocked as you might think," said I, "as I know
something about governments. But tell me, how do you manage, and how
have you come to this state of things?"

Said he: "It is true that we have to make some arrangements about our
affairs, concerning which you can ask presently; and it is also true
that everybody does not always agree with the details of these
arrangements; but, further, it is true that a man no more needs an
elaborate system of government, with its army, navy, and police, to
force him to give way to the will of the majority of his \emph{equals},
than he wants a similar machinery to make him understand that his head
and a stone wall cannot occupy the same space at the same moment. Do you
want further explanation?"

"Well, yes, I do," quoth I.

Old Hammond settled himself in his chair with a look of enjoyment which
rather alarmed me, and made me dread a scientific disquisition: so I
sighed and abided. He said:

"I suppose you know pretty well what the process of government was in
the bad old times?"

"I am supposed to know," said I.

(Hammond) What was the government of those days? Was it really the
Parliament or any part of it?

(I) No.

(H.) Was not the Parliament on the one side a kind of watch-committee
sitting to see that the interests of the Upper Classes took no hurt; and
on the other side a sort of blind to delude the people into supposing
that they had some share in the management of their own affairs?

(I) History seems to show us this.

(H.) To what extent did the people manage their own affairs?

(I) I judge from what I have heard that sometimes they forced the
Parliament to make a law to legalise some alteration which had already
taken place.

(H.) Anything else?

(I) I think not. As I am informed, if the people made any attempt to
deal with the \emph{cause} of their grievances, the law stepped in and
said, this is sedition, revolt, or what not, and slew or tortured the
ringleaders of such attempts.

(H.) If Parliament was not the government then, nor the people either,
what was the government?

(I) Can you tell me?

(H.) I think we shall not be far wrong if we say that government was the
Law-Courts, backed up by the executive, which handled the brute force
that the deluded people allowed them to use for their own purposes; I
mean the army, navy, and police.

(I) Reasonable men must needs think you are right.

(H.) Now as to those Law-Courts. Were they places of fair dealing
according to the ideas of the day? Had a poor man a good chance of
defending his property and person in them?

(I) It is a commonplace that even rich men looked upon a law-suit as a
dire misfortune, even if they gained the case; and as for a poor
one---why, it was considered a miracle of justice and beneficence if a
poor man who had once got into the clutches of the law escaped prison or
utter ruin.

(H.) It seems, then, my son, that the government by law-courts and
police, which was the real government of the nineteenth century, was not
a great success even to the people of that day, living under a class
system which proclaimed inequality and poverty as the law of God and the
bond which held the world together.

(I) So it seems, indeed.

(H.) And now that all this is changed, and the "rights of property,"
which mean the clenching the fist on a piece of goods and crying out to
the neighbours, You shan't have this!---now that all this has
disappeared so utterly that it is no longer possible even to jest upon
its absurdity, is such a Government possible?

(I) It is impossible.

(H.) Yes, happily. But for what other purpose than the protection of the
rich from the poor, the strong from the weak, did this Government exist?

(I.) I have heard that it was said that their office was to defend their
own citizens against attack from other countries.

(H.) It was said; but was anyone expected to believe this? For instance,
did the English Government defend the English citizen against the
French?

(I) So it was said.

(H.) Then if the French had invaded England and conquered it, they would
not have allowed the English workmen to live well?

(I, laughing) As far as I can make out, the English masters of the
English workmen saw to that: they took from their workmen as much of
their livelihood as they dared, because they wanted it for themselves.

(H.) But if the French had conquered, would they not have taken more
still from the English workmen?

(I) I do not think so; for in that case the English workmen would have
died of starvation; and then the French conquest would have ruined the
French, just as if the English horses and cattle had died of
under-feeding. So that after all, the English \emph{workmen} would have
been no worse off for the conquest: their French Masters could have got
no more from them than their English masters did.

(H.) This is true; and we may admit that the pretensions of the
government to defend the poor (\emph{i.e.}, the useful) people against
other countries come to nothing. But that is but natural; for we have
seen already that it was the function of government to protect the rich
against the poor. But did not the government defend its rich men against
other nations?

(I) I do not remember to have heard that the rich needed defence;
because it is said that even when two nations were at war, the rich men
of each nation gambled with each other pretty much as usual, and even
sold each other weapons wherewith to kill their own countrymen.

(H.) In short, it comes to this, that whereas the so-called government
of protection of property by means of the law-courts meant destruction
of wealth, this defence of the citizens of one country against those of
another country by means of war or the threat of war meant pretty much
the same thing.

(I) I cannot deny it.

(H.) Therefore the government really existed for the destruction of
wealth?

(I) So it seems. And yet---

(H.) Yet what?

(I) There were many rich people in those times.

(H.) You see the consequences of that fact?

(I) I think I do. But tell me out what they were.

(H.) If the government habitually destroyed wealth, the country must
have been poor?

(I) Yes, certainly.

(H.) Yet amidst this poverty the persons for the sake of whom the
government existed insisted on being rich whatever might happen?

(I) So it was.

(H.) What must happen if in a poor country some people insist on being
rich at the expense of the others?

(I) Unutterable poverty for the others. All this misery, then, was
caused by the destructive government of which we have been speaking?

(H.) Nay, it would be incorrect to say so. The government itself was but
the necessary result of the careless, aimless tyranny of the times; it
was but the machinery of tyranny. Now tyranny has come to an end, and we
no longer need such machinery; we could not possibly use it since we are
free. Therefore in your sense of the word we have no government. Do you
understand this now?

(I) Yes, I do. But I will ask you some more questions as to how you as
free men manage your affairs.

(H.) With all my heart. Ask away.
