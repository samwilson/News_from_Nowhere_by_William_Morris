I said nothing, for I was not inclined for mere politeness to him after
such very serious talk; but in fact I should liked to have gone on
talking with the older man, who could understand something at least of
my wonted ways of looking at life, whereas, with the younger people, in
spite of all their kindness, I really was a being from another planet.
However, I made the best of it, and smiled as amiably as I could on the
young couple; and Dick returned the smile by saying, "Well, guest, I am
glad to have you again, and to find that you and my kinsman have not
quite talked yourselves into another world; I was half suspecting as I
was listening to the Welshmen yonder that you would presently be
vanishing away from us, and began to picture my kinsman sitting in the
hall staring at nothing and finding that he had been talking a while
past to nobody."

I felt rather uncomfortable at this speech, for suddenly the picture of
the sordid squabble, the dirty and miserable tragedy of the life I had
left for a while, came before my eyes; and I had, as it were, a vision
of all my longings for rest and peace in the past, and I loathed the
idea of going back to it again. But the old man chuckled and said:

"Don't be afraid, Dick. In any case, I have not been talking to thin
air; nor, indeed to this new friend of ours only. Who knows but I may
not have been talking to many people? For perhaps our guest may some day
go back to the people he has come from, and may take a message from us
which may bear fruit for them, and consequently for us."

Dick looked puzzled, and said: "Well, gaffer, I do not quite understand
what you mean. All I can say is, that I hope he will not leave us: for
don't you see, he is another kind of man to what we are used to, and
somehow he makes us think of all kind of things; and already I feel as
if I could understand Dickens the better for having talked with him."

"Yes," said Clara, "and I think in a few months we shall make him look
younger; and I should like to see what he was like with the wrinkles
smoothed out of his face. Don't you think he will look younger after a
little time with us?"

The old man shook his head, and looked earnestly at me, but did not
answer her, and for a moment or two we were all silent. Then Clara broke
out:

"Kinsman, I don't like this: something or another troubles me, and I
feel as if something untoward were going to happen. You have been
talking of past miseries to the guest, and have been living in past
unhappy times, and it is in the air all round us, and makes us feel as
if we were longing for something that we cannot have."

The old man smiled on her kindly, and said: "Well, my child, if that be
so, go and live in the present, and you will soon shake it off." Then he
turned to me, and said: "Do you remember anything like that, guest, in
the country from which you come?"

The lovers had turned aside now, and were talking together softly, and
not heeding us; so I said, but in a low voice: "Yes, when I was a happy
child on a sunny holiday, and had everything that I could think of."

"So it is," said he. "You remember just now you twitted me with living
in the second childhood of the world. You will find it a happy world to
live in; you will be happy there---for a while."

Again I did not like his scarcely veiled threat, and was beginning to
trouble myself with trying to remember how I had got amongst this
curious people, when the old man called out in a cheery voice: "Now, my
children, take your guest away, and make much of him; for it is your
business to make him sleek of skin and peaceful of mind: he has by no
means been as lucky as you have. Farewell, guest!" and he grasped my
hand warmly.

"Good-bye," said I, "and thank you very much for all that you have told
me. I will come and see you as soon as I come back to London. May I?"

"Yes," he said, "come by all means---if you can."

"It won't be for some time yet," quoth Dick, in his cheery voice; "for
when the hay is in up the river, I shall be for taking him a round
through the country between hay and wheat harvest, to see how our
friends live in the north country. Then in the wheat harvest we shall do
a good stroke of work, I should hope,---in Wiltshire by preference; for
he will be getting a little hard with all the open-air living, and I
shall be as tough as nails."

"But you will take me along, won't you, Dick?" said Clara, laying her
pretty hand on his shoulder.

"Will I not?" said Dick, somewhat boisterously. "And we will manage to
send you to bed pretty tired every night; and you will look so beautiful
with your neck all brown, and your hands too, and you under your gown as
white as privet, that you will get some of those strange discontented
whims out of your head, my dear. However, our week's haymaking will do
all that for you."

The girl reddened very prettily, and not for shame but for pleasure; and
the old man laughed, and said:

"Guest, I see that you will be as comfortable as need be; for you need
not fear that those two will be too officious with you: they will be so
busy with each other, that they will leave you a good deal to yourself,
I am sure, and that is a real kindness to a guest, after all. O, you
need not be afraid of being one too many, either: it is just what these
birds in a nest like, to have a good convenient friend to turn to, so
that they may relieve the ecstasies of love with the solid commonplace
of friendship. Besides, Dick, and much more Clara, likes a little
talking at times; and you know lovers do not talk unless they get into
trouble, they only prattle. Good-bye, guest; may you be happy!"

Clara went up to old Hammond, threw her arms about his neck and kissed
him heartily, and said:

"You are a dear old man, and may have your jest about me as much as you
please; and it won't be long before we see you again; and you may be
sure we shall make our guest happy; though, mind you, there is some
truth in what you say."

Then I shook hands again, and we went out of the hall and into the
cloisters, and so in the street found Greylocks in the shafts waiting
for us. He was well looked after; for a little lad of about seven years
old had his hand on the rein and was solemnly looking up into his face;
on his back, withal, was a girl of fourteen, holding a three-year old
sister on before her; while another girl, about a year older than the
boy, hung on behind. The three were occupied partly with eating
cherries, partly with patting and punching Greylocks, who took all their
caresses in good part, but pricked up his ears when Dick made his
appearance. The girls got off quietly, and going up to Clara, made much
of her and snuggled up to her. And then we got into the carriage, Dick
shook the reins, and we got under way at once, Greylocks trotting
soberly between the lovely trees of the London streets, that were
sending floods of fragrance into the cool evening air; for it was now
getting toward sunset.

We could hardly go but fair and softly all the way, as there were a
great many people abroad in that cool hour. Seeing so many people made
me notice their looks the more; and I must say, my taste, cultivated in
the sombre greyness, or rather brownness, of the nineteenth century, was
rather apt to condemn the gaiety and brightness of the raiment; and I
even ventured to say as much to Clara. She seemed rather surprised, and
even slightly indignant, and said: "Well, well, what's the matter? They
are not about any dirty work; they are only amusing themselves in the
fine evening; there is nothing to foul their clothes. Come, doesn't it
all look very pretty? It isn't gaudy, you know."

Indeed that was true; for many of the people were clad in colours that
were sober enough, though beautiful, and the harmony of the colours was
perfect and most delightful.

I said, "Yes, that is so; but how can everybody afford such costly
garments? Look! there goes a middle-aged man in a sober grey dress; but
I can see from here that it is made of very fine woollen stuff, and is
covered with silk embroidery."

Said Clara: "He could wear shabby clothes if he pleased,---that is, if
he didn't think he would hurt people's feelings by doing so."

"But please tell me," said I, "how can they afford it?"

As soon as I had spoken I perceived that I had got back to my old
blunder; for I saw Dick's shoulders shaking with laughter; but he
wouldn't say a word, but handed me over to the tender mercies of Clara,
who said---

"Why, I don't know what you mean. Of course we can afford it, or else we
shouldn't do it. It would be easy enough for us to say, we will only
spend our labour on making our clothes comfortable: but we don't choose
to stop there. Why do you find fault with us? Does it seem to you as if
we starved ourselves of food in order to make ourselves fine clothes? Or
do you think there is anything wrong in liking to see the coverings of
our bodies beautiful like our bodies are?---just as a deer's or an
otter's skin has been made beautiful from the first? Come, what is wrong
with you?"

I bowed before the storm, and mumbled out some excuse or other. I must
say, I might have known that people who were so fond of architecture
generally, would not be backward in ornamenting themselves; all the more
as the shape of their raiment, apart from its colour, was both beautiful
and reasonable---veiling the form, without either muffling or
caricaturing it.

Clara was soon mollified; and as we drove along toward the wood before
mentioned, she said to Dick---

"I tell you what, Dick: now that kinsman Hammond the Elder has seen our
guest in his queer clothes, I think we ought to find him something
decent to put on for our journey to-morrow: especially since, if we do
not, we shall have to answer all sorts of questions as to his clothes
and where they came from. Besides," she said slily, "when he is clad in
handsome garments he will not be so quick to blame us for our
childishness in wasting our time in making ourselves look pleasant to
each other."

"All right, Clara," said Dick; "he shall have everything that you---that
he wants to have. I will look something out for him before he gets up
to-morrow."
