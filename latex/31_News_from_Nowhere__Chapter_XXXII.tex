Dick brought me at once into the little field which, as I had seen from
the garden, was covered with gaily-coloured tents arranged in orderly
lanes, about which were sitting and lying on the grass some fifty or
sixty men, women, and children, all of them in the height of good temper
and enjoyment---with their holiday mood on, so to say.

"You are thinking that we don't make a great show as to numbers," said
Dick; "but you must remember that we shall have more to-morrow; because
in this haymaking work there is room for a great many people who are not
over-skilled in country matters: and there are many who lead sedentary
lives, whom it would be unkind to deprive of their pleasure in the
hay-field---scientific men and close students generally: so that the
skilled workmen, outside those who are wanted as mowers, and foremen of
the haymaking, stand aside, and take a little downright rest, which you
know is good for them, whether they like it or not: or else they go to
other countrysides, as I am doing here. You see, the scientific men and
historians, and students generally, will not be wanted till we are
fairly in the midst of the tedding, which of course will not be till the
day after to-morrow." With that he brought me out of the little field on
to a kind of causeway above the river-side meadow, and thence turning to
the left on to a path through the mowing grass, which was thick and very
tall, led on till we came to the river above the weir and its mill.
There we had a delightful swim in the broad piece of water above the
lock, where the river looked much bigger than its natural size from its
being dammed up by the weir.

"Now we are in a fit mood for dinner," said Dick, when we had dressed
and were going through the grass again; "and certainly of all the
cheerful meals in the year, this one of haysel is the cheerfullest; not
even excepting the corn-harvest feast; for then the year is beginning to
fail, and one cannot help having a feeling behind all the gaiety, of the
coming of the dark days, and the shorn fields and empty gardens; and the
spring is almost too far off to look forward to. It is, then, in the
autumn, when one almost believes in death."

"How strangely you talk," said I, "of such a constantly recurring and
consequently commonplace matter as the sequence of the seasons." And
indeed these people were like children about such things, and had what
seemed to me a quite exaggerated interest in the weather, a fine day, a
dark night, or a brilliant one, and the like.

"Strangely?" said he. "Is it strange to sympathise with the year and its
gains and losses?"

"At any rate," said I, "if you look upon the course of the year as a
beautiful and interesting drama, which is what I think you do, you
should be as much pleased and interested with the winter and its trouble
and pain as with this wonderful summer luxury."

"And am I not?" said Dick, rather warmly; "only I can't look upon it as
if I were sitting in a theatre seeing the play going on before me,
myself taking no part of it. It is difficult," said he, smiling
good-humouredly, "for a non-literary man like me to explain myself
properly, like that dear girl Ellen would; but I mean that I am part of
it all, and feel the pain as well as the pleasure in my own person. It
is not done for me by somebody else, merely that I may eat and drink and
sleep; but I myself do my share of it."

In his way also, as Ellen in hers, I could see that Dick had that
passionate love of the earth which was common to but few people at
least, in the days I knew; in which the prevailing feeling amongst
intellectual persons was a kind of sour distaste for the changing drama
of the year, for the life of earth and its dealings with men. Indeed, in
those days it was thought poetic and imaginative to look upon life as a
thing to be borne, rather than enjoyed.

So I mused till Dick's laugh brought me back into the Oxfordshire
hay-fields. "One thing seems strange to me," said he---"that I must
needs trouble myself about the winter and its scantiness, in the midst
of the summer abundance. If it hadn't happened to me before, I should
have thought it was your doing, guest; that you had thrown a kind of
evil charm over me. Now, you know," said he, suddenly, "that's only a
joke, so you mustn't take it to heart."

"All right," said I; "I don't." Yet I did feel somewhat uneasy at his
words, after all.

We crossed the causeway this time, and did not turn back to the house,
but went along a path beside a field of wheat now almost ready to
blossom. I said:

"We do not dine in the house or garden, then?---as indeed I did not
expect to do. Where do we meet, then? For I can see that the houses are
mostly very small."

"Yes," said Dick, "you are right, they are small in this country-side:
there are so many good old houses left, that people dwell a good deal in
such small detached houses. As to our dinner, we are going to have our
feast in the church. I wish, for your sake, it were as big and handsome
as that of the old Roman town to the west, or the forest town to the
north;\textsuperscript{\protect\hyperlink{cite_note-1}{{[}1{]}}} but,
however, it will hold us all; and though it is a little thing, it is
beautiful in its way."

This was somewhat new to me, this dinner in a church, and I thought of
the church-ales of the Middle Ages; but I said nothing, and presently we
came out into the road which ran through the village. Dick looked up and
down it, and seeing only two straggling groups before us, said: "It
seems as if we must be somewhat late; they are all gone on; and they
will be sure to make a point of waiting for you, as the guest of guests,
since you come from so far."

He hastened as he spoke, and I kept up with him, and presently we came
to a little avenue of lime-trees which led us straight to the church
porch, from whose open door came the sound of cheerful voices and
laughter, and varied merriment.

"Yes," said Dick, "it's the coolest place for one thing, this hot
evening. Come along; they will be glad to see you."

Indeed, in spite of my bath, I felt the weather more sultry and
oppressive than on any day of our journey yet.

We went into the church, which was a simple little building with one
little aisle divided from the nave by three round arches, a chancel, and
a rather roomy transept for so small a building, the windows mostly of
the graceful Oxfordshire fourteenth century type. There was no modern
architectural decoration in it; it looked, indeed, as if none had been
attempted since the Puritans whitewashed the mediaeval saints and
histories on the wall. It was, however, gaily dressed up for this
latter-day festival, with festoons of flowers from arch to arch, and
great pitchers of flowers standing about on the floor; while under the
west window hung two cross scythes, their blades polished white, and
gleaming from out of the flowers that wreathed them. But its best
ornament was the crowd of handsome, happy-looking men and women that
were set down to table, and who, with their bright faces and rich hair
over their gay holiday raiment, looked, as the Persian poet puts it,
like a bed of tulips in the sun. Though the church was a small one,
there was plenty of room; for a small church makes a biggish house; and
on this evening there was no need to set cross tables along the
transepts; though doubtless these would be wanted next day, when the
learned men of whom Dick has been speaking should be come to take their
more humble part in the haymaking.

I stood on the threshold with the expectant smile on my face of a man
who is going to take part in a festivity which he is really prepared to
enjoy. Dick, standing by me was looking round the company with an air of
proprietorship in them, I thought. Opposite me sat Clara and Ellen, with
Dick's place open between them: they were smiling, but their beautiful
faces were each turned towards the neighbours on either side, who were
talking to them, and they did not seem to see me. I turned to Dick,
expecting him to lead me forward, and he turned his face to me; but
strange to say, though it was as smiling and cheerful as ever, it made
no response to my glance---nay, he seemed to take no heed at all of my
presence, and I noticed that none of the company looked at me. A pang
shot through me, as of some disaster long expected and suddenly
realised. Dick moved on a little without a word to me. I was not three
yards from the two women who, though they had been my companions for
such a short time, had really, as I thought, become my friends. Clara's
face was turned full upon me now, but she also did not seem to see me,
though I know I was trying to catch her eye with an appealing look. I
turned to Ellen, and she \emph{did} seem to recognise me for an instant;
but her bright face turned sad directly, and she shook her head with a
mournful look, and the next moment all consciousness of my presence had
faded from her face.

I felt lonely and sick at heart past the power of words to describe. I
hung about a minute longer, and then turned and went out of the porch
again and through the lime-avenue into the road, while the blackbirds
sang their strongest from the bushes about me in the hot June evening.

Once more without any conscious effort of will I set my face toward the
old house by the ford, but as I turned round the corner which led to the
remains of the village cross, I came upon a figure strangely contrasting
with the joyous, beautiful people I had left behind in the church. It
was a man who looked old, but whom I knew from habit, now half
forgotten, was really not much more than fifty. His face was rugged, and
grimed rather than dirty; his eyes dull and bleared; his body bent, his
calves thin and spindly, his feet dragging and limping. His clothing was
a mixture of dirt and rags long over-familiar to me. As I passed him he
touched his hat with some real goodwill and courtesy, and much
servility.

Inexpressibly shocked, I hurried past him and hastened along the road
that led to the river and the lower end of the village; but suddenly I
saw as it were a black cloud rolling along to meet me, like a nightmare
of my childish days; and for a while I was conscious of nothing else
than being in the dark, and whether I was walking, or sitting, or lying
down, I could not tell.

*{⁠}*{⁠}*

I lay in my bed in my house at dingy Hammersmith thinking about it all;
and trying to consider if I was overwhelmed with despair at finding I
had been dreaming a dream; and strange to say, I found that I was not so
despairing.

Or indeed \emph{was} it a dream? If so, why was I so conscious all along
that I was really seeing all that new life from the outside, still
wrapped up in the prejudices, the anxieties, the distrust of this time
of doubt and struggle?

All along, though those friends were so real to me, I had been feeling
as if I had no business amongst them: as though the time would come when
they would reject me, and say, as Ellen's last mournful look seemed to
say, "No, it will not do; you cannot be of us; you belong so entirely to
the unhappiness of the past that our happiness even would weary you. Go
back again, now you have seen us, and your outward eyes have learned
that in spite of all the infallible maxims of your day there is yet a
time of rest in store for the world, when mastery has changed into
fellowship---but not before. Go back again, then, and while you live you
will see all round you people engaged in making others live lives which
are not their own, while they themselves care nothing for their own real
lives---men who hate life though they fear death. Go back and be the
happier for having seen us, for having added a little hope to your
struggle. Go on living while you may, striving, with whatsoever pain and
labour needs must be, to build up little by little the new day of
fellowship, and rest, and happiness."

Yes, surely! and if others can see it as I have seen it, then it may be
called a vision rather than a dream.

\hypertarget{footnotesedit}{%
\subsection[{{{[}}\href{/w/index.php?title=News_from_Nowhere/Chapter_XXXII\&action=edit\&section=1}{edit}{{]}}}]{\texorpdfstring{\protect\hypertarget{Footnotes}{}{Footnotes}{{{[}}\href{/w/index.php?title=News_from_Nowhere/Chapter_XXXII\&action=edit\&section=1}{edit}{{]}}}}{Footnotes{[}edit{]}}}\label{footnotesedit}}

\begin{enumerate}
\tightlist
\item
  \protect\hypertarget{cite_note-1}{}{{\protect\hyperlink{cite_ref-1}{↑}}
  {Cirencester and Burford he must have meant.}}
\end{enumerate}
