"Well," said I, "so you got clear out of all your trouble. Were people
satisfied with the new order of things when it came?"

"People?" he said. "Well, surely all must have been glad of peace when
it came; especially when they found, as they must have found, that after
all, they---even the once rich---were not living very badly. As to those
who had been poor, all through the war, which lasted about two years,
their condition had been bettering, in spite of the struggle; and when
peace came at last, in a very short time they made great strides towards
a decent life. The great difficulty was that the once-poor had such a
feeble conception of the real pleasure of life: so to say, they did not
ask enough, did not know how to ask enough, from the new state of
things. It was perhaps rather a good than an evil thing that the
necessity for restoring the wealth destroyed during the war forced them
into working at first almost as hard as they had been used to before the
Revolution. For all historians are agreed that there never was a war in
which there was so much destruction of wares, and instruments for making
them as in this civil war."

"I am rather surprised at that," said I.

"Are you? I don't see why," said Hammond.

"Why," I said, "because the party of order would surely look upon the
wealth as their own property, no share of which, if they could help it,
should go to their slaves, supposing they conquered. And on the other
hand, it was just for the possession of that wealth that 'the rebels'
were fighting, and I should have thought, especially when they saw that
they were winning, that they would have been careful to destroy as
little as possible of what was so soon to be their own."

"It was as I have told you, however," said he. "The party of order, when
they recovered from their first cowardice of surprise---or, if you
please, when they fairly saw that, whatever happened, they would be
ruined, fought with great bitterness, and cared little what they did, so
long as they injured the enemies who had destroyed the sweets of life
for them. As to 'the rebels,' I have told you that the outbreak of
actual war made them careless of trying to save the wretched scraps of
wealth that they had. It was a common saying amongst them, Let the
country be cleared of everything except valiant living men, rather than
that we fall into slavery again!"

He sat silently thinking a little while, and then said:

"When the conflict was once really begun, it was seen how little of any
value there was in the old world of slavery and inequality. Don't you
see what it means? In the times which you are thinking of, and of which
you seem to know so much, there was no hope; nothing but the dull jog of
the mill-horse under compulsion of collar and whip; but in that
fighting-time that followed, all was hope: 'the rebels' at least felt
themselves strong enough to build up the world again from its dry
bones,---and they did it, too!" said the old man, his eyes glittering
under his beetling brows. He went on: "And their opponents at least and
at last learned something about the reality of life, and its sorrows,
which they---their class, I mean---had once known nothing of. In short,
the two combatants, the workman and the gentleman, between them---"

"Between them," said I, quickly, "they destroyed commercialism!"

"Yes, yes, yes," said he; "that is it. Nor could it have been destroyed
otherwise; except, perhaps, by the whole of society gradually falling
into lower depths, till it should at last reach a condition as rude as
barbarism, but lacking both the hope and the pleasures of barbarism.
Surely the sharper, shorter remedy was the happiest."

"Most surely," said I.

"Yes," said the old man, "the world was being brought to its second
birth; how could that take place without a tragedy? Moreover, think of
it. The spirit of the new days, of our days, was to be delight in the
life of the world; intense and overweening love of the very skin and
surface of the earth on which man dwells, such as a lover has in the
fair flesh of the woman he loves; this, I say, was to be the new spirit
of the time. All other moods save this had been exhausted: the unceasing
criticism, the boundless curiosity in the ways and thoughts of man,
which was the mood of the ancient Greek, to whom these things were not
so much a means, as an end, was gone past recovery; nor had there been
really any shadow of it in the so-called science of the nineteenth
century, which, as you must know, was in the main an appendage to the
commercial system; nay, not seldom an appendage to the police of that
system. In spite of appearances, it was limited and cowardly, because it
did not really believe in itself. It was the outcome, as it was the sole
relief, of the unhappiness of the period which made life so bitter even
to the rich, and which, as you may see with your bodily eyes, the great
change has swept away. More akin to our way of looking at life was the
spirit of the Middle Ages, to whom heaven and the life of the next world
was such a reality, that it became to them a part of the life upon the
earth; which accordingly they loved and adorned, in spite of the ascetic
doctrines of their formal creed, which bade them contemn it.

"But that also, with its assured belief in heaven and hell as two
countries in which to live, has gone, and now we do, both in word and in
deed, believe in the continuous life of the world of men, and as it
were, add every day of that common life to the little stock of days
which our own mere individual experience wins for us: and consequently
we are happy. Do you wonder at it? In times past, indeed, men were told
to love their kind, to believe in the religion of humanity, and so
forth. But look you, just in the degree that a man had elevation of mind
and refinement enough to be able to value this idea, was he repelled by
the obvious aspect of the individuals composing the mass which he was to
worship; and he could only evade that repulsion by making a conventional
abstraction of mankind that had little actual or historical relation to
the race; which to his eyes was divided into blind tyrants on the one
hand and apathetic degraded slaves on the other. But now, where is the
difficulty in accepting the religion of humanity, when the men and women
who go to make up humanity are free, happy, and energetic at least, and
most commonly beautiful of body also, and surrounded by beautiful things
of their own fashioning, and a nature bettered and not worsened by
contact with mankind? This is what this age of the world has reserved
for us."

"It seems true," said I, "or ought to be, if what my eyes have seen is a
token of the general life you lead. Can you now tell me anything of your
progress after the years of the struggle?"

Said he: "I could easily tell you more than you have time to listen to;
but I can at least hint at one of the chief difficulties which had to be
met: and that was, that when men began to settle down after the war, and
their labour had pretty much filled up the gap in wealth caused by the
destruction of that war, a kind of disappointment seemed coming over us,
and the prophecies of some of the reactionists of past times seemed as
if they would come true, and a dull level of utilitarian comfort be the
end for a while of our aspirations and success. The loss of the
competitive spur to exertion had not, indeed, done anything to interfere
with the necessary production of the community, but how if it should
make men dull by giving them too much time for thought or idle musing?
But, after all, this dull thunder-cloud only threatened us, and then
passed over. Probably, from what I have told you before, you will have a
guess at the remedy for such a disaster; remembering always that many of
the things which used to be produced---slave-wares for the poor and mere
wealth-wasting wares for the rich---ceased to be made. That remedy was,
in short, the production of what used to be called art, but which has no
name amongst us now, because it has become a necessary part of the
labour of every man who produces."

Said I: "What! had men any time or opportunity for cultivating the fine
arts amidst the desperate struggle for life and freedom that you have
told me of?"

Said Hammond: "You must not suppose that the new form of art was founded
chiefly on the memory of the art of the past; although, strange to say,
the civil war was much less destructive of art than of other things, and
though what of art existed under the old forms, revived in a wonderful
way during the latter part of the struggle, especially as regards music
and poetry. The art or work-pleasure, as one ought to call it, of which
I am now speaking, sprung up almost spontaneously, it seems, from a kind
of instinct amongst people, no longer driven desperately to painful and
terrible over-work, to do the best they could with the work in hand---to
make it excellent of its kind; and when that had gone on for a little, a
craving for beauty seemed to awaken in men's minds, and they began
rudely and awkwardly to ornament the wares which they made; and when
they had once set to work at that, it soon began to grow. All this was
much helped by the abolition of the squalor which our immediate
ancestors put up with so coolly; and by the leisurely, but not stupid,
country-life which now grew (as I told you before) to be common amongst
us. Thus at last and by slow degrees we got pleasure into our work; then
we became conscious of that pleasure, and cultivated it, and took care
that we had our fill of it; and then all was gained, and we were happy.
So may it be for ages and ages!"

The old man fell into a reverie, not altogether without melancholy I
thought; but I would not break it. Suddenly he started, and said: "Well,
dear guest, here are come Dick and Clara to fetch you away, and there is
an end of my talk; which I daresay you will not be sorry for; the long
day is coming to an end, and you will have a pleasant ride back to
Hammersmith."
