So on we went, Dick rowing in an easy tireless way, and Clara sitting by
my side admiring his manly beauty and heartily good-natured face, and
thinking, I fancy, of nothing else. As we went higher up the river,
there was less difference between the Thames of that day and Thames as I
remembered it; for setting aside the hideous vulgarity of the cockney
villas of the well-to-do, stockbrokers and other such, which in older
time marred the beauty of the bough-hung banks, even this beginning of
the country Thames was always beautiful; and as we slipped between the
lovely summer greenery, I almost felt my youth come back to me, and as
if I were on one of those water excursions which I used to enjoy so much
in days when I was too happy to think that there could be much amiss
anywhere.

At last we came to a reach of the river where on the left hand a very
pretty little village with some old houses in it came down to the edge
of the water, over which was a ferry; and beyond these houses the
elm-beset meadows ended in a fringe of tall willows, while on the right
hand went the tow-path and a clear space before a row of trees, which
rose up behind huge and ancient, the ornaments of a great park: but
these drew back still further from the river at the end of the reach to
make way for a little town of quaint and pretty houses, some new, some
old, dominated by the long walls and sharp gables of a great red-brick
pile of building, partly of the latest Gothic, partly of the court-style
of Dutch William, but so blended together by the bright sun and
beautiful surroundings, including the bright blue river, which it looked
down upon, that even amidst the beautiful buildings of that new happy
time it had a strange charm about it. A great wave of fragrance, amidst
which the lime-tree blossom was clearly to be distinguished, came down
to us from its unseen gardens, as Clara sat up in her place, and said:

"O Dick, dear, couldn't we stop at Hampton Court for to-day, and take
the guest about the park a little, and show him those sweet old
buildings? Somehow, I suppose because you have lived so near it, you
have seldom taken me to Hampton Court."

Dick rested on his oars a little, and said: "Well, well, Clara, you are
lazy to-day. I didn't feel like stopping short of Shepperton for the
night; suppose we just go and have our dinner at the Court, and go on
again about five o'clock?"

"Well," she said, "so be it; but I should like the guest to have spent
an hour or two in the Park."

"The Park!" said Dick; "why, the whole Thames-side is a park this time
of the year; and for my part, I had rather lie under an elm-tree on the
borders of a wheat-field, with the bees humming about me and the
corn-crake crying from furrow to furrow, than in any park in England.
Besides---"

"Besides," said she, "you want to get on to your dearly-loved upper
Thames, and show your prowess down the heavy swathes of the mowing
grass."

She looked at him fondly, and I could tell that she was seeing him in
her mind's eye showing his splendid form at its best amidst the rhymed
strokes of the scythes; and she looked down at her own pretty feet with
a half sigh, as though she were contrasting her slight woman's beauty
with his man's beauty; as women will when they are really in love, and
are not spoiled with conventional sentiment.

As for Dick, he looked at her admiringly a while, and then said at last:
"Well, Clara, I do wish we were there! But, hilloa! we are getting back
way." And he set to work sculling again, and in two minutes we were all
standing on the gravelly strand below the bridge, which, as you may
imagine, was no longer the old hideous iron abortion, but a handsome
piece of very solid oak framing.

We went into the Court and straight into the great hall, so well
remembered, where there were tables spread for dinner, and everything
arranged much as in Hammersmith Guest-Hall. Dinner over, we sauntered
through the ancient rooms, where the pictures and tapestry were still
preserved, and nothing was much changed, except that the people whom we
met there had an indefinable kind of look of being at home and at ease,
which communicated itself to me, so that I felt that the beautiful old
place was mine in the best sense of the word; and my pleasure of past
days seemed to add itself to that of to-day, and filled my whole soul
with content.

Dick (who, in spite of Clara's gibe, knew the place very well) told me
that the beautiful old Tudor rooms, which I remembered had been the
dwellings of the lesser fry of Court flunkies, were now much used by
people coming and going; for, beautiful as architecture had now become,
and although the whole face of the country had quite recovered its
beauty, there was still a sort of tradition of pleasure and beauty which
clung to that group of buildings, and people thought going to Hampton
Court a necessary summer outing, as they did in the days when London was
so grimy and miserable. We went into some of the rooms looking into the
old garden, and were well received by the people in them, who got
speedily into talk with us, and looked with politely half-concealed
wonder at my strange face. Besides these birds of passage, and a few
regular dwellers in the place, we saw out in the meadows near the
garden, down "the Long Water," as it used to be called, many gay tents
with men, women, and children round about them. As it seemed, this
pleasure-loving people were fond of tent-life, with all its
inconveniences, which, indeed, they turned into pleasure also.

We left this old friend by the time appointed, and I made some feeble
show of taking the sculls; but Dick repulsed me, not much to my grief, I
must say, as I found I had quite enough to do between the enjoyment of
the beautiful time and my own lazily blended thoughts.

As to Dick, it was quite right to let him pull, for he was as strong as
a horse, and had the greatest delight in bodily exercise, whatever it
was. We really had some difficulty in getting him to stop when it was
getting rather more than dusk, and the moon was brightening just as we
were off Runnymede. We landed there, and were looking about for a place
whereon to pitch our tents (for we had brought two with us), when an old
man came up to us, bade us good evening, and asked if we were housed for
that that night; and finding that we were not, bade us home to his
house. Nothing loth, we went with him, and Clara took his hand in a
coaxing way which I noticed she used with old men; and as we went on our
way, made some commonplace remark about the beauty of the day. The old
man stopped short, and looked at her and said: "You really like it
then?"

"Yes," she said, looking very much astonished, "Don't you?"

"Well," said he, "perhaps I do. I did, at any rate, when I was younger;
but now I think I should like it cooler."

She said nothing, and went on, the night growing about as dark as it
would be; till just at the rise of the hill we came to a hedge with a
gate in it, which the old man unlatched and led us into a garden, at the
end of which we could see a little house, one of whose little windows
was already yellow with candlelight. We could see even under the
doubtful light of the moon and the last of the western glow that the
garden was stuffed full of flowers; and the fragrance it gave out in the
gathering coolness was so wonderfully sweet, that it seemed the very
heart of the delight of the June dusk; so that we three stopped
instinctively, and Clara gave forth a little sweet "O," like a bird
beginning to sing.

"What's the matter?" said the old man, a little testily, and pulling at
her hand. "There's no dog; or have you trodden on a thorn and hurt your
foot?"

"No, no, neighbour," she said; "but how sweet, how sweet it is!"

"Of course it is," said he, "but do you care so much for that?"

She laughed out musically, and we followed suit in our gruffer voices;
and then she said: "Of course I do, neighbour; don't you?"

"Well, I don't know," quoth the old fellow; then he added, as if
somewhat ashamed of himself: "Besides, you know, when the waters are out
and all Runnymede is flooded, it's none so pleasant."

"\emph{I} should like it," quoth Dick. "What a jolly sail one would get
about here on the floods on a bright frosty January morning!"

"\emph{Would} you like it?" said our host. "Well, I won't argue with
you, neighbour; it isn't worth while. Come in and have some supper."

We went up a paved path between the roses, and straight into a very
pretty room, panelled and carved, and as clean as a new pin; but the
chief ornament of which was a young woman, light-haired and grey-eyed,
but with her face and hands and bare feet tanned quite brown with the
sun. Though she was very lightly clad, that was clearly from choice, not
from poverty, though these were the first cottage-dwellers I had come
across; for her gown was of silk, and on her wrists were bracelets that
seemed to me of great value. She was lying on a sheep-skin near the
window, but jumped up as soon as we entered, and when she saw the guests
behind the old man, she clapped her hands and cried out with pleasure,
and when she got us into the middle of the room, fairly danced round us
in delight of our company.

"What!" said the old man, "you are pleased, are you, Ellen?"

The girl danced up to him and threw her arms round him, and said: "Yes I
am, and so ought you to be grandfather."

"Well, well, I am," said he, "as much as I can be pleased. Guests,
please be seated."

This seemed rather strange to us; stranger, I suspect, to my friends
than to me; but Dick took the opportunity of both the host and his
grand-daughter being out of the room to say to me, softly: "A grumbler:
there are a few of them still. Once upon a time, I am told, they were
quite a nuisance."

The old man came in as he spoke and sat down beside us with a sigh,
which, indeed, seemed fetched up as if he wanted us to take notice of
it; but just then the girl came in with the victuals, and the carle
missed his mark, what between our hunger generally and that I was pretty
busy watching the grand-daughter moving about as beautiful as a picture.

Everything to eat and drink, though it was somewhat different to what we
had had in London, was better than good, but the old man eyed rather
sulkily the chief dish on the table, on which lay a leash of fine perch,
and said:

"H'm, perch! I am sorry we can't do better for you, guests. The time was
when we might have had a good piece of salmon up from London for you;
but the times have grown mean and petty."

"Yes, but you might have had it now," said the girl, giggling, "if you
had known that they were coming."

"It's our fault for not bringing it with us, neighbours," said Dick,
good-humouredly. "But if the times have grown petty, at any rate the
perch haven't; that fellow in the middle there must have weighed a good
two pounds when he was showing his dark stripes and red fins to the
minnows yonder. And as to the salmon, why, neighbour, my friend here,
who comes from the outlands, was quite surprised yesterday morning when
I told him we had plenty of salmon at Hammersmith. I am sure I have
heard nothing of the times worsening."

He looked a little uncomfortable. And the old man, turning to me, said
very courteously:

"Well, sir, I am happy to see a man from over the water; but I really
must appeal to you to say whether on the whole you are not better off in
your country; where I suppose, from what our guest says, you are brisker
and more alive, because you have not wholly got rid of competition. You
see, I have read not a few books of the past days, and certainly
\emph{they} are much more alive than those which are written now; and
good sound unlimited competition was the condition under which they were
written,---if we didn't know that from the record of history, we should
know it from the books themselves. There is a spirit of adventure in
them, and signs of a capacity to extract good out of evil which our
literature quite lacks now; and I cannot help thinking that our
moralists and historians exaggerate hugely the unhappiness of the past
days, in which such splendid works of imagination and intellect were
produced."

Clara listened to him with restless eyes, as if she were excited and
pleased; Dick knitted his brow and looked still more uncomfortable, but
said nothing. Indeed, the old man gradually, as he warmed to his
subject, dropped his sneering manner, and both spoke and looked very
seriously. But the girl broke out before I could deliver myself of the
answer I was framing:

"Books, books! always books, grandfather! When will you understand that
after all it is the world we live in which interests us; the world of
which we are a part, and which we can never love too much? Look!" she
said, throwing open the casement wider and showing us the white light
sparkling between the black shadows of the moonlit garden, through which
ran a little shiver of the summer night-wind, "look! these are our books
in these days!---and these," she said, stepping lightly up to the two
lovers and laying a hand on each of their shoulders; "and the guest
there, with his over-sea knowledge and experience;---yes, and even you,
grandfather" (a smile ran over her face as she spoke), "with all your
grumbling and wishing yourself back again in the good old days,---in
which, as far as I can make out, a harmless and lazy old man like you
would either have pretty nearly starved, or have had to pay soldiers and
people to take the folk's victuals and clothes and houses away from them
by force. Yes, these are our books; and if we want more, can we not find
work to do in the beautiful buildings that we raise up all over the
country (and I know there was nothing like them in past times), wherein
a man can put forth whatever is in him, and make his hands set forth his
mind and his soul."

She paused a little, and I for my part could not help staring at her,
and thinking that if she were a book, the pictures in it were most
lovely. The colour mantled in her delicate sunburnt cheeks; her grey
eyes, light amidst the tan of her face, kindly looked on us all as she
spoke. She paused, and said again:

"As for your books, they were well enough for times when intelligent
people had but little else in which they could take pleasure, and when
they must needs supplement the sordid miseries of their own lives with
imaginations of the lives of other people. But I say flatly that in
spite of all their cleverness and vigour, and capacity for
story-telling, there is something loathsome about them. Some of them,
indeed, do here and there show some feeling for those whom the
history-books call 'poor,' and of the misery of whose lives we have some
inkling; but presently they give it up, and towards the end of the story
we must be contented to see the hero and heroine living happily in an
island of bliss on other people's troubles; and that after a long series
of sham troubles (or mostly sham) of their own making, illustrated by
dreary introspective nonsense about their feelings and aspirations, and
all the rest of it; while the world must even then have gone on its way,
and dug and sewed and baked and built and carpentered round about these
useless---animals."

"There!" said the old man, reverting to his dry sulky manner again.
"There's eloquence! I suppose you like it?"

"Yes," said I, very emphatically.

"Well," said he, "now the storm of eloquence has lulled for a little,
suppose you answer my question?---that is, if you like, you know," quoth
he, with a sudden access of courtesy.

"What question?" said I. For I must confess that Ellen's strange and
almost wild beauty had put it out of my head.

Said he: "First of all (excuse my catechising), is there competition in
life, after the old kind, in the country whence you come?"

"Yes," said I, "it is the rule there." And I wondered as I spoke what
fresh complications I should get into as a result of this answer.

"Question two," said the carle: "Are you not on the whole much freer,
more energetic---in a word, healthier and happier---for it?"

I smiled. "You wouldn't talk so if you had any idea of our life. To me
you seem here as if you were living in heaven compared with us of the
country from which I came."

"Heaven?" said he: "you like heaven, do you?"

"Yes," said I---snappishly, I am afraid; for I was beginning rather to
resent his formula.

"Well, I am far from sure that I do," quoth he. "I think one may do more
with one's life than sitting on a damp cloud and singing hymns."

I was rather nettled by this inconsequence, and said: "Well, neighbour,
to be short, and without using metaphors, in the land whence I come,
where the competition which produced those literary works which you
admire so much is still the rule, most people are thoroughly unhappy;
here, to me at least most people seem thoroughly happy."

"No offence, guest---no offence," said he; "but let me ask you; you like
that, do you?"

His formula, put with such obstinate persistence, made us all laugh
heartily; and even the old man joined in the laughter on the sly.
However, he was by no means beaten, and said presently:

"From all I can hear, I should judge that a young woman so beautiful as
my dear Ellen yonder would have been a lady, as they called it in the
old time, and wouldn't have had to wear a few rags of silk as she does
now, or to have browned herself in the sun as she has to do now. What do
you say to that, eh?"

Here Clara, who had been pretty much silent hitherto, struck in, and
said: "Well, really, I don't think that you would have mended matters,
or that they want mending. Don't you see that she is dressed deliciously
for this beautiful weather? And as for the sun-burning of your
hay-fields, why, I hope to pick up some of that for myself when we get a
little higher up the river. Look if I don't need a little sun on my
pasty white skin!"

And she stripped up the sleeve from her arm and laid it beside Ellen's
who was now sitting next her. To say the truth, it was rather amusing to
me to see Clara putting herself forward as a town-bred fine lady, for
she was as well-knit and clean-skinned a girl as might be met with
anywhere at the best. Dick stroked the beautiful arm rather shyly, and
pulled down the sleeve again, while she blushed at his touch; and the
old man said laughingly: "Well, I suppose you \emph{do} like that; don't
you?"

Ellen kissed her new friend, and we all sat silent for a little, till
she broke out into a sweet shrill song, and held us all entranced with
the wonder of her clear voice; and the old grumbler sat looking at her
lovingly. The other young people sang also in due time; and then Ellen
showed us to our beds in small cottage chambers, fragrant and clean as
the ideal of the old pastoral poets; and the pleasure of the evening
quite extinguished my fear of the last night, that I should wake up in
the old miserable world of worn-out pleasures, and hopes that were half
fears.
