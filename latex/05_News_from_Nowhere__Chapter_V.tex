Past the Broadway there were fewer houses on either side. We presently
crossed a pretty little brook that ran across a piece of land dotted
over with trees, and awhile after came to another market and town-hall,
as we should call it. Although there was nothing familiar to me in its
surroundings, I knew pretty well where we were, and was not surprised
when my guide said briefly, "Kensington Market."

Just after this we came into a short street of houses: or rather, one
long house on either side of the way, built of timber and plaster, and
with a pretty arcade over the footway before it.

Quoth Dick: "This is Kensington proper. People are apt to gather here
rather thick, for they like the romance of the wood; and naturalists
haunt it, too; for it is a wild spot even here, what there is of it; for
it does not go far to the south: it goes from here northward and west
right over Paddington and a little way down Notting Hill: thence it runs
north-east to Primrose Hill, and so on; rather a narrow strip of it gets
through Kingsland to Stoke-Newington and Clapton, where it spreads out
along the heights above the Lea marshes; on the other side of which, as
you know, is Epping Forest holding out a hand to it. This part we are
just coming to is called Kensington Gardens; though why 'gardens' I
don't know."

I rather longed to say, "Well, \emph{I} know"; but there were so many
things about me which I did \emph{not} know, in spite of his
assumptions, that I thought it better to hold my tongue.

The road plunged at once into a beautiful wood spreading out on either
side, but obviously much further on the north side, where even the oaks
and sweet chestnuts were of a good growth; while the quicker-growing
trees (amongst which I thought the planes and sycamores too numerous)
were very big and fine-grown.

It was exceedingly pleasant in the dappled shadow, for the day was
growing as hot as need be, and the coolness and shade soothed my excited
mind into a condition of dreamy pleasure, so that I felt as if I should
like to go on for ever through that balmy freshness. My companion seemed
to share in my feelings, and let the horse go slower and slower as he
sat inhaling the green forest scents, chief amongst which was the smell
of the trodden bracken near the wayside.

Romantic as this Kensington wood was, however, it was not lonely. We
came on many groups both coming and going, or wandering in the edges of
the wood. Amongst these were many children from six or eight years old
up to sixteen or seventeen. They seemed to me to be especially fine
specimens of their race, and enjoying themselves to the utmost; some of
them were hanging about little tents pitched on the greensward, and by
some of these fires were burning, with pots hanging over them gipsy
fashion. Dick explained to me that there were scattered houses in the
forest, and indeed we caught a glimpse of one or two. He said they were
mostly quite small, such as used to be called cottages when there were
slaves in the land, but they were pleasant enough and fitting for the
wood.

"They must be pretty well stocked with children," said I, pointing to
the many youngsters about the way.

"O," said he, "these children do not all come from the near houses, the
woodland houses, but from the country-side generally. They often make up
parties, and come to play in the woods for weeks together in
summer-time, living in tents, as you see. We rather encourage them to
it; they learn to do things for themselves, and get to notice the wild
creatures; and, you see, the less they stew inside houses the better for
them. Indeed, I must tell you that many grown people will go to live in
the forests through the summer; though they for the most part go to the
bigger ones, like Windsor, or the Forest of Dean, or the northern
wastes. Apart from the other pleasures of it, it gives them a little
rough work, which I am sorry to say is getting somewhat scarce for these
last fifty years."

He broke off, and then said, "I tell you all this, because I see that if
I talk I must be answering questions, which you are thinking, even if
you are not speaking them out; but my kinsman will tell you more about
it."

I saw that I was likely to get out of my depth again, and so merely for
the sake of tiding over an awkwardness and to say something, I said---

"Well, the youngsters here will be all the fresher for school when the
summer gets over and they have to go back again."

"School?" he said; "yes, what do you mean by that word? I don't see how
it can have anything to do with children. We talk, indeed, of a school
of herring, and a school of painting, and in the former sense we might
talk of a school of children---but otherwise," said he, laughing, "I
must own myself beaten."

Hang it! thought I, I can't open my mouth without digging up some new
complexity. I wouldn't try to set my friend right in his etymology; and
I thought I had best say nothing about the boy-farms which I had been
used to call schools, as I saw pretty clearly that they had disappeared;
so I said after a little fumbling, "I was using the word in the sense of
a system of education."

"Education?" said he, meditatively, "I know enough Latin to know that
the word must come from \emph{educere}, to lead out; and I have heard it
used; but I have never met anybody who could give me a clear explanation
of what it means."

You may imagine how my new friends fell in my esteem when I heard this
frank avowal; and I said, rather contemptuously, "Well, education means
a system of teaching young people."

"Why not old people also?" said he with a twinkle in his eye. "But," he
went on, "I can assure you our children learn, whether they go through a
'system of teaching' or not. Why, you will not find one of these
children about here, boy or girl, who cannot swim; and every one of them
has been used to tumbling about the little forest ponies---there's one
of them now! They all of them know how to cook; the bigger lads can mow;
many can thatch and do odd jobs at carpentering; or they know how to
keep shop. I can tell you they know plenty of things."

"Yes, but their mental education, the teaching of their minds," said I,
kindly translating my phrase.

"Guest," said he, "perhaps you have not learned to do these things I
have been speaking about; and if that's the case, don't you run away
with the idea that it doesn't take some skill to do them, and doesn't
give plenty of work for one's mind: you would change your opinion if you
saw a Dorsetshire lad thatching, for instance. But, however, I
understand you to be speaking of book-learning; and as to that, it is a
simple affair. Most children, seeing books lying about, manage to read
by the time they are four years old; though I am told it has not always
been so. As to writing, we do not encourage them to scrawl too early
(though scrawl a little they will), because it gets them into a habit of
ugly writing; and what's the use of a lot of ugly writing being done,
when rough printing can be done so easily. You understand that handsome
writing we like, and many people will write their books out when they
make them, or get them written; I mean books of which only a few copies
are needed---poems, and such like, you know. However, I am wandering
from my lambs; but you must excuse me, for I am interested in this
matter of writing, being myself a fair-writer."

"Well," said I, "about the children; when they know how to read and
write, don't they learn something else---languages, for instance?"

"Of course," he said; "sometimes even before they can read, they can
talk French, which is the nearest language talked on the other side of
the water; and they soon get to know German also, which is talked by a
huge number of communes and colleges on the mainland. These are the
principal languages we speak in these islands, along with English or
Welsh, or Irish, which is another form of Welsh; and children pick them
up very quickly, because their elders all know them; and besides our
guests from over sea often bring their children with them, and the
little ones get together, and rub their speech into one another."

"And the older languages?" said I.

"O, yes," said he, "they mostly learn Latin and Greek along with the
modern ones, when they do anything more than merely pick up the latter."

"And history?" said I; "how do you teach history?"

"Well," said he, "when a person can read, of course he reads what he
likes to; and he can easily get someone to tell him what are the best
books to read on such or such a subject, or to explain what he doesn't
understand in the books when he is reading them."

"Well," said I, "what else do they learn? I suppose they don't all learn
history?"

"No, no," said he; "some don't care about it; in fact, I don't think
many do. I have heard my great-grandfather say that it is mostly in
periods of turmoil and strife and confusion that people care much about
history; and you know," said my friend, with an amiable smile, "we are
not like that now. No; many people study facts about the make of things
and the matters of cause and effect, so that knowledge increases on us,
if that be good; and some, as you heard about friend Bob yonder, will
spend time over mathematics. 'Tis no use forcing people's tastes."

Said I: "But you don't mean that children learn all these things?"

Said he: "That depends on what you mean by children; and also you must
remember how much they differ. As a rule, they don't do much reading,
except for a few story-books, till they are about fifteen years old; we
don't encourage early bookishness: though you will find some children
who \emph{will} take to books very early; which perhaps is not good for
them; but it's no use thwarting them; and very often it doesn't last
long with them, and they find their level before they are twenty years
old. You see, children are mostly given to imitating their elders, and
when they see most people about them engaged in genuinely amusing work,
like house-building and street-paving, and gardening, and the like, that
is what they want to be doing; so I don't think we need fear having too
many book-learned men."

What could I say? I sat and held my peace, for fear of fresh
entanglements. Besides, I was using my eyes with all my might, wondering
as the old horse jogged on, when I should come into London proper, and
what it would be like now.

But my companion couldn't let his subject quite drop, and went on
meditatively:

"After all, I don't know that it does them much harm, even if they do
grow up book-students. Such people as that, 'tis a great pleasure seeing
them so happy over work which is not much sought for. And besides, these
students are generally such pleasant people; so kind and sweet tempered;
so humble, and at the same time so anxious to teach everybody all that
they know. Really, I like those that I have met prodigiously."

This seemed to me such very queer talk that I was on the point of asking
him another question; when just as we came to the top of a rising
ground, down a long glade of the wood on my right I caught sight of a
stately building whose outline was familiar to me, and I cried out,
"Westminster Abbey!"

"Yes," said Dick, "Westminster Abbey---what there is left of it."

"Why, what have you done with it?" quoth I in terror.

"What have \emph{we} done with it?" said he; "nothing much, save clean
it. But you know the whole outside was spoiled centuries ago: as to the
inside, that remains in its beauty after the great clearance, which took
place over a hundred years ago, of the beastly monuments to fools and
knaves, which once blocked it up, as great-grandfather says."

We went on a little further, and I looked to the right again, and said,
in rather a doubtful tone of voice, "Why, there are the Houses of
Parliament! Do you still use them?"

He burst out laughing, and was some time before he could control
himself; then he clapped me on the back and said:

"I take you, neighbour; you may well wonder at our keeping them
standing, and I know something about that, and my old kinsman has given
me books to read about the strange game that they played there. Use
them! Well, yes, they are used for a sort of subsidiary market, and a
storage place for manure, and they are handy for that, being on the
waterside. I believe it was intended to pull them down quite at the
beginning of our days; but there was, I am told, a queer antiquarian
society, which had done some service in past times, and which
straightway set up its pipe against their destruction, as it has done
with many other buildings, which most people looked upon as worthless,
and public nuisances; and it was so energetic, and had such good reasons
to give, that it generally gained its point; and I must say that when
all is said I am glad of it: because you know at the worst these silly
old buildings serve as a kind of foil to the beautiful ones which we
build now. You will see several others in these parts; the place my
great-grandfather lives in, for instance, and a big building called St.
Paul's. And you see, in this matter we need not grudge a few poorish
buildings standing, because we can always build elsewhere; nor need we
be anxious as to the breeding of pleasant work in such matters, for
there is always room for more and more work in a new building, even
without making it pretentious. For instance, elbow-room \emph{within}
doors is to me so delightful that if I were driven to it I would most
sacrifice outdoor space to it. Then, of course, there is the ornament,
which, as we must all allow, may easily be overdone in mere living
houses, but can hardly be in mote-halls and markets, and so forth. I
must tell you, though, that my great-grandfather sometimes tells me I am
a little cracked on this subject of fine building; and indeed I
\emph{do} think that the energies of mankind are chiefly of use to them
for such work; for in that direction I can see no end to the work, while
in many others a limit does seem possible."
