"Your kinsman doesn't much care for beautiful building, then," said I,
as we entered the rather dreary classical house; which indeed was as
bare as need be, except for some big pots of the June flowers which
stood about here and there; though it was very clean and nicely
whitewashed.

"O I don't know," said Dick, rather absently. "He is getting old,
certainly, for he is over a hundred and five, and no doubt he doesn't
care about moving. But of course he could live in a prettier house if he
liked: he is not obliged to live in one place any more than any one
else. This way, Guest."

And he led the way upstairs, and opening a door we went into a
fair-sized room of the old type, as plain as the rest of the house, with
a few necessary pieces of furniture, and those very simple and even
rude, but solid and with a good deal of carving about them, well
designed but rather crudely executed. At the furthest corner of the
room, at a desk near the window, sat a little old man in a roomy oak
chair, well becushioned. He was dressed in a sort of Norfolk jacket of
blue serge worn threadbare, with breeches of the same, and grey worsted
stockings. He jumped up from his chair, and cried out in a voice of
considerable volume for such an old man, "Welcome, Dick, my lad; Clara
is here, and will be more than glad to see you; so keep your heart up."

"Clara here?" quoth Dick; "if I had known, I would not have brought---At
least, I mean I would---"

He was stuttering and confused, clearly because he was anxious to say
nothing to make me feel one too many. But the old man, who had not seen
me at first, helped him out by coming forward and saying to me in a kind
tone:

"Pray pardon me, for I did not notice that Dick, who is big enough to
hide anybody, you know, had brought a friend with him. A most hearty
welcome to you! All the more, as I almost hope that you are going to
amuse an old man by giving him news from over sea, for I can see that
you are come from over the water and far off countries."

He looked at me thoughtfully, almost anxiously, as he said in a changed
voice, "Might I ask you where you come from, as you are so clearly a
stranger?"

I said in an absent way: "I used to live in England, and now I am come
back again; and I slept last night at the Hammersmith Guest House."

He bowed gravely, but seemed, I thought, a little disappointed with my
answer. As for me, I was now looking at him harder than good manners
allowed of; perhaps; for in truth his face, dried-apple-like as it was,
seemed strangely familiar to me; as if I had seen it before---in a
looking-glass it might be, said I to myself.

"Well," said the old man, "wherever you come from, you are come among
friends. And I see my kinsman Richard Hammond has an air about him as if
he had brought you here for me to do something for you. Is that so,
Dick?"

Dick, who was getting still more absent-minded and kept looking uneasily
at the door, managed to say, "Well, yes, kinsman: our guest finds things
much altered, and cannot understand it; nor can I; so I thought I would
bring him to you, since you know more of all that has happened within
the last two hundred years than any body else does.---What's that?"

And he turned toward the door again. We heard footsteps outside; the
door opened, and in came a very beautiful young woman, who stopped short
on seeing Dick, and flushed as red as a rose, but faced him
nevertheless. Dick looked at her hard, and half reached out his hand
toward her, and his whole face quivered with emotion.

The old man did not leave them long in this shy discomfort, but said,
smiling with an old man's mirth:

"Dick, my lad, and you, my dear Clara, I rather think that we two
oldsters are in your way; for I think you will have plenty to say to
each other. You had better go into Nelson's room up above; I know he has
gone out; and he has just been covering the walls all over with
mediaeval books, so it will be pretty enough even for you two and your
renewed pleasure."

The girl reached out her hand to Dick, and taking his led him out of the
room, looking straight before her; but it was easy to see that her
blushes came from happiness, not anger; as, indeed, love is far more
self-conscious than wrath.

When the door had shut on them the old man turned to me, still smiling,
and said:

"Frankly, my dear guest, you will do me a great service if you are come
to set my old tongue wagging. My love of talk still abides with me, or
rather grows on me; and though it is pleasant enough to see these
youngsters moving about and playing together so seriously, as if the
whole world depended on their kisses (as indeed it does somewhat), yet I
don't think my tales of the past interest them much. The last harvest,
the last baby, the last knot of carving in the market-place, is history
enough for them. It was different, I think, when I was a lad, when we
were not so assured of peace and continuous plenty as we are now---Well,
well! Without putting you to the question, let me ask you this: Am I to
consider you as an enquirer who knows a little of our modern ways of
life, or as one who comes from some place where the very foundations of
life are different from ours,---do you know anything or nothing about
us?"

He looked at me keenly and with growing wonder in his eyes as he spoke;
and I answered in a low voice:

"I know only so much of your modern life as I could gather from using my
eyes on the way here from Hammersmith, and from asking some questions of
Richard Hammond, most of which he could hardly understand."

The old man smiled at this. "Then," said he, "I am to speak to you
as---"

"As if I were a being from another planet," said I.

The old man, whose name, by the bye, like his kinsman's, was Hammond,
smiled and nodded, and wheeling his seat round to me, bade me sit in a
heavy oak chair, and said, as he saw my eyes fix on its curious carving:

"Yes, I am much tied to the past, my past, you understand. These very
pieces of furniture belong to a time before my early days; it was my
father who got them made; if they had been done within the last fifty
years they would have been much cleverer in execution; but I don't think
I should have liked them the better. We were almost beginning again in
those days: and they were brisk, hot-headed times. But you hear how
garrulous I am: ask me questions, ask me questions about anything, dear
guest; since I must talk, make my talk profitable to you."

I was silent for a minute, and then I said, somewhat nervously: "Excuse
me if I am rude; but I am so much interested in Richard, since he has
been so kind to me, a perfect stranger, that I should like to ask a
question about him."

"Well," said old Hammond, "if he were not 'kind', as you call it, to a
perfect stranger he would be thought a strange person, and people would
be apt to shun him. But ask on, ask on! don't be shy of asking."

Said I: "That beautiful girl, is he going to be married to her?"

"Well," said he, "yes, he is. He has been married to her once already,
and now I should say it is pretty clear that he will be married to her
again."

"Indeed," quoth I, wondering what that meant.

"Here is the whole tale," said old Hammond; "a short one enough; and now
I hope a happy one: they lived together two years the first time; were
both very young; and then she got it into her head that she was in love
with somebody else. So she left poor Dick; I say \emph{poor} Dick,
because he had not found any one else. But it did not last long, only
about a year. Then she came to me, as she was in the habit of bringing
her troubles to the old carle, and asked me how Dick was, and whether he
was happy, and all the rest of it. So I saw how the land lay, and said
that he was very unhappy, and not at all well; which last at any rate
was a lie. There, you can guess the rest. Clara came to have a long talk
with me to-day, but Dick will serve her turn much better. Indeed, if he
hadn't chanced in upon me to-day I should have had to have sent for him
to-morrow."

"Dear me," said I. "Have they any children?"

"Yes," said he, "two; they are staying with one of my daughters at
present, where, indeed, Clara has mostly been. I wouldn't lose sight of
her, as I felt sure they would come together again: and Dick, who is the
best of good fellows, really took the matter to heart. You see, he had
no other love to run to, as she had. So I managed it all; as I have done
with such-like matters before."

"Ah," said I, "no doubt you wanted to keep them out of the Divorce
Court: but I suppose it often has to settle such matters."

"Then you suppose nonsense," said he. "I know that there used to be such
lunatic affairs as divorce-courts: but just consider; all the cases that
came into them were matters of property quarrels: and I think, dear
guest," said he, smiling, "that though you do come from another planet,
you can see from the mere outside look of our world that quarrels about
private property could not go on amongst us in our days."

Indeed, my drive from Hammersmith to Bloomsbury, and all the quiet happy
life I had seen so many hints of; even apart from my shopping, would
have been enough to tell me that "the sacred rights of property," as we
used to think of them, were now no more. So I sat silent while the old
man took up the thread of the discourse again, and said:

"Well, then, property quarrels being no longer possible, what remains in
these matters that a court of law could deal with? Fancy a court for
enforcing a contract of passion or sentiment! If such a thing were
needed as a \emph{reductio ad absurdum} of the enforcement of contract,
such a folly would do that for us."

He was silent again a little, and then said: "You must understand once
for all that we have changed these matters; or rather, that our way of
looking at them has changed, as we have changed within the last two
hundred years. We do not deceive ourselves, indeed, or believe that we
can get rid of all the trouble that besets the dealings between the
sexes. We know that we must face the unhappiness that comes of man and
woman confusing the relations between natural passion, and sentiment,
and the friendship which, when things go well, softens the awakening
from passing illusions: but we are not so mad as to pile up degradation
on that unhappiness by engaging in sordid squabbles about livelihood and
position, and the power of tyrannising over the children who have been
the results of love or lust."

Again he paused awhile, and again went on: "Calf love, mistaken for a
heroism that shall be lifelong, yet early waning into disappointment;
the inexplicable desire that comes on a man of riper years to be the
all-in-all to some one woman, whose ordinary human kindness and human
beauty he has idealised into superhuman perfection, and made the one
object of his desire; or lastly the reasonable longing of a strong and
thoughtful man to become the most intimate friend of some beautiful and
wise woman, the very type of the beauty and glory of the world which we
love so well,---as we exult in all the pleasure and exaltation of spirit
which goes with these things, so we set ourselves to bear the sorrow
which not unseldom goes with them also; remembering those lines of the
ancient poet (I quote roughly from memory one of the many translations
of the nineteenth century):

\begin{verbatim}
  'For this the Gods have fashioned man's grief and evil day
  That still for man hereafter might be the tale and the lay.'
\end{verbatim}

Well, well, 'tis little likely anyhow that all tales shall be lacking,
or all sorrow cured."

He was silent for some time, and I would not interrupt him. At last he
began again: "But you must know that we of these generations are strong
and healthy of body, and live easily; we pass our lives in reasonable
strife with nature, exercising not one side of ourselves only, but all
sides, taking the keenest pleasure in all the life of the world. So it
is a point of honour with us not to be self-centred; not to suppose that
the world must cease because one man is sorry; therefore we should think
it foolish, or if you will, criminal, to exaggerate these matters of
sentiment and sensibility: we are no more inclined to eke out our
sentimental sorrows than to cherish our bodily pains; and we recognise
that there are other pleasures besides love-making. You must remember,
also, that we are long-lived, and that therefore beauty both in man and
woman is not so fleeting as it was in the days when we were burdened so
heavily by self-inflicted diseases. So we shake off these griefs in a
way which perhaps the sentimentalists of other times would think
contemptible and unheroic, but which we think necessary and manlike. As
on the other hand, therefore, we have ceased to be commercial in our
love-matters, so also we have ceased to be \emph{artificially} foolish.
The folly which comes by nature, the unwisdom of the immature man, or
the older man caught in a trap, we must put up with that, nor are we
much ashamed of it; but to be conventionally sensitive or
sentimental---my friend, I am old and perhaps disappointed, but at least
I think we have cast off \emph{some} of the follies of the older world."

He paused, as if for some words of mine; but I held my peace: then he
went on: "At least, if we suffer from the tyranny and fickleness of
nature or our own want of experience, we neither grimace about it, nor
lie. If there must be sundering betwixt those who meant never to sunder,
so it must be: but there need be no pretext of unity when the reality of
it is gone: nor do we drive those who well know that they are incapable
of it to profess an undying sentiment which they cannot really feel:
thus it is that as that monstrosity of venal lust is no longer possible,
so also it is no longer needed. Don't misunderstand me. You did not
seemed shocked when I told you that there were no law-courts to enforce
contracts of sentiment or passion; but so curiously are men made, that
perhaps you will be shocked when I tell you that there is no code of
public opinion which takes the place of such courts, and which might be
as tyrannical and unreasonable as they were. I do not say that people
don't judge their neighbours' conduct, sometimes, doubtless, unfairly.
But I do say that there is no unvarying conventional set of rules by
which people are judged; no bed of Procrustes to stretch or cramp their
minds and lives; no hypocritical excommunication which people are
\emph{forced} to pronounce, either by unconsidered habit, or by the
unexpressed threat of the lesser interdict if they are lax in their
hypocrisy. Are you shocked now?"

"N-o---no," said I, with some hesitation. "It is all so different."

"At any rate," said he, "one thing I think I can answer for: whatever
sentiment there is, it is real---and general; it is not confined to
people very specially refined. I am also pretty sure, as I hinted to you
just now, that there is not by a great way as much suffering involved in
these matters either to men or to women as there used to be. But excuse
me for being so prolix on this question! You know you asked to be
treated like a being from another planet."

"Indeed I thank you very much," said I. "Now may I ask you about the
position of women in your society?"

He laughed very heartily for a man of his years, and said: "It is not
without reason that I have got a reputation as a careful student of
history. I believe I really do understand 'the Emancipation of Women
movement' of the nineteenth century. I doubt if any other man now alive
does."

"Well?" said I, a little bit nettled by his merriment.

"Well," said he, "of course you will see that all that is a dead
controversy now. The men have no longer any opportunity of tyrannising
over the women, or the women over the men; both of which things took
place in those old times. The women do what they can do best, and what
they like best, and the men are neither jealous of it or injured by it.
This is such a commonplace that I am almost ashamed to state it."

I said, "O; and legislation? do they take any part in that?"

Hammond smiled and said: "I think you may wait for an answer to that
question till we get on to the subject of legislation. There may be
novelties to you in that subject also."

"Very well," I said; "but about this woman question? I saw at the Guest
House that the women were waiting on the men: that seems a little like
reaction doesn't it?"

"Does it?" said the old man; "perhaps you think housekeeping an
unimportant occupation, not deserving of respect. I believe that was the
opinion of the 'advanced' women of the nineteenth century, and their
male backers. If it is yours, I recommend to your notice an old
Norwegian folk-lore tale called How the Man minded the House, or some
such title; the result of which minding was that, after various
tribulations, the man and the family cow balanced each other at the end
of a rope, the man hanging halfway up the chimney, the cow dangling from
the roof, which, after the fashion of the country, was of turf and
sloping down low to the ground. Hard on the cow, \emph{I} think. Of
course no such mishap could happen to such a superior person as
yourself," he added, chuckling.

I sat somewhat uneasy under this dry gibe. Indeed, his manner of
treating this latter part of the question seemed to me a little
disrespectful.

"Come, now, my friend," quoth he, "don't you know that it is a great
pleasure to a clever woman to manage a house skilfully, and to do it so
that all the house-mates about her look pleased, and are grateful to
her? And then, you know, everybody likes to be ordered about by a pretty
woman: why, it is one of the pleasantest forms of flirtation. You are
not so old that you cannot remember that. Why, I remember it well."

And the old fellow chuckled again, and at last fairly burst out
laughing.

"Excuse me," said he, after a while; "I am not laughing at anything you
could be thinking of; but at that silly nineteenth-century fashion,
current amongst rich so-called cultivated people, of ignoring all the
steps by which their daily dinner was reached, as matters too low for
their lofty intelligence. Useless idiots! Come, now, I am a 'literary
man,' as we queer animals used to be called, yet I am a pretty good cook
myself."

"So am I," said I.

"Well, then," said he, "I really think you can understand me better than
you would seem to do, judging by your words and your silence."

Said I: "Perhaps that is so; but people putting in practice commonly
this sense of interest in the ordinary occupations of life rather
startles me. I will ask you a question or two presently about that. But
I want to return to the position of women amongst you. You have studied
the 'emancipation of women' business of the nineteenth century: don't
you remember that some of the 'superior' women wanted to emancipate the
more intelligent part of their sex from the bearing of children?"

The old man grew quite serious again. Said he: "I \emph{do} remember
about that strange piece of baseless folly, the result, like all other
follies of the period, of the hideous class tyranny which then obtained.
What do we think of it now? you would say. My friend, that is a question
easy to answer. How could it possibly be but that maternity should be
highly honoured amongst us? Surely it is a matter of course that the
natural and necessary pains which the mother must go through form a bond
of union between man and woman, an extra stimulus to love and affection
between them, and that this is universally recognised. For the rest,
remember that all the \emph{artificial} burdens of motherhood are now
done away with. A mother has no longer any mere sordid anxieties for the
future of her children. They may indeed turn out better or worse; they
may disappoint her highest hopes; such anxieties as these are a part of
the mingled pleasure and pain which goes to make up the life of mankind.
But at least she is spared the fear (it was most commonly the certainty)
that artificial disabilities would make her children something less than
men and women: she knows that they will live and act according to the
measure of their own faculties. In times past, it is clear that the
'Society' of the day helped its Judaic god, and the 'Man of Science' of
the time, in visiting the sins of the fathers upon the children. How to
reverse this process, how to take the sting out of heredity, has for
long been one of the most constant cares of the thoughtful men amongst
us. So that, you see, the ordinarily healthy woman (and almost all our
women are both healthy and at least comely), respected as a child-bearer
and rearer of children, desired as a woman, loved as a companion,
unanxious for the future of her children, has far more instinct for
maternity than the poor drudge and mother of drudges of past days could
ever have had; or than her sister of the upper classes, brought up in
affected ignorance of natural facts, reared in an atmosphere of mingled
prudery and prurience."

"You speak warmly," I said, "but I can see that you are right."

"Yes," he said, "and I will point out to you a token of all the benefits
which we have gained by our freedom. What did you think of the looks of
the people whom you have come across to-day?"

Said I: "I could hardly have believed that there could be so many
good-looking people in any civilised country."

He crowed a little, like the old bird he was. "What! are we still
civilised?" said he. "Well, as to our looks, the English and Jutish
blood, which on the whole is predominant here, used not to produce much
beauty. But I think we have improved it. I know a man who has a large
collection of portraits printed from photographs of the nineteenth
century, and going over those and comparing them with the everyday faces
in these times, puts the improvement in our good looks beyond a doubt.
Now, there are some people who think it not too fantastic to connect
this increase of beauty directly with our freedom and good sense in the
matters we have been speaking of: they believe that a child born from
the natural and healthy love between a man and a woman, even if that be
transient, is likely to turn out better in all ways, and especially in
bodily beauty, than the birth of the respectable commercial marriage
bed, or of the dull despair of the drudge of that system. They say,
Pleasure begets pleasure. What do you think?"

"I am much of that mind," said I.
