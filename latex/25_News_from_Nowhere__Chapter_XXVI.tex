Before we parted from these girls we saw two sturdy young men and a
woman putting off from the Berkshire shore, and then Dick bethought him
of a little banter of the girls, and asked them how it was that there
was nobody of the male kind to go with them across the water, and where
their boats were gone to. Said one, the youngest of the party: "O, they
have got the big punt to lead stone from up the water."

"Who do you mean by 'they,' dear child?" said Dick.

Said an older girl, laughing: "You had better go and see them. Look
there," and she pointed northwest, "don't you see building going on
there?"

"Yes," said Dick, "and I am rather surprised at this time of the year;
why are they not haymaking with you?"

The girls all laughed at this, and before their laugh was over, the
Berkshire boat had run on to the grass and the girls stepped in lightly,
still sniggering, while the new comers gave us the sele of the day. But
before they were under way again, the tall girl said:

"Excuse us for laughing, dear neighbours, but we have had some friendly
bickering with the builders up yonder, and as we have no time to tell
you the story, you had better go and ask them: they will be glad to see
you---if you don't hinder their work."

They all laughed again at that, and waved us a pretty farewell as the
punters set them over toward the other shore, and left us standing on
the bank beside our boat.

"Let us go and see them," said Clara; "that is, if you are not in a
hurry to get to Streatley, Walter?"

"O no," said Walter, "I shall be glad of the excuse to have a little
more of your company."

So we left the boat moored there, and went on up the slow slope of the
hill; but I said to Dick on the way, being somewhat mystified: "What was
all that laughing about? what was the joke!"

"I can guess pretty well," said Dick; "some of them up there have got a
piece of work which interests them, and they won't go to the haymaking,
which doesn't matter at all, because there are plenty of people to do
such easy-hard work as that; only, since haymaking is a regular
festival, the neighbours find it amusing to jeer good-humouredly at
them."

"I see," said I, "much as if in Dickens's time some young people were so
wrapped up in their work that they wouldn't keep Christmas."

"Just so," said Dick, "only these people need not be young either."

"But what did you mean by easy-hard work?" said I.

Quoth Dick: "Did I say that? I mean work that tries the muscles and
hardens them and sends you pleasantly weary to bed, but which isn't
trying in other ways: doesn't harass you in short. Such work is always
pleasant if you don't overdo it. Only, mind you, good mowing requires
some little skill. I'm a pretty good mower."

This talk brought us up to the house that was a-building, not a large
one, which stood at the end of a beautiful orchard surrounded by an old
stone wall. "O yes, I see," said Dick; "I remember, a beautiful place
for a house: but a starveling of a nineteenth century house stood there:
I am glad they are rebuilding: it's all stone, too, though it need not
have been in this part of the country: my word, though, they are making
a neat job of it: but I wouldn't have made it all ashlar."

Walter and Clara were already talking to a tall man clad in his mason's
blouse, who looked about forty, but was I daresay older, who had his
mallet and chisel in hand; there were at work in the shed and on the
scaffold about half a dozen men and two women, blouse-clad like the
carles, while a very pretty woman who was not in the work but was
dressed in an elegant suit of blue linen came sauntering up to us with
her knitting in her hand. She welcomed us and said, smiling: "So you are
come up from the water to see the Obstinate Refusers: where are you
going haymaking, neighbours?"

"O, right up above Oxford," said Dick; "it is rather a late country. But
what share have you got with the Refusers, pretty neighbour?"

Said she, with a laugh: "O, I am the lucky one who doesn't want to work;
though sometimes I get it, for I serve as model to Mistress Philippa
there when she wants one: she is our head carver; come and see her."

She led us up to the door of the unfinished house, where a rather little
woman was working with mallet and chisel on the wall near by. She seemed
very intent on what she was doing, and did not turn round when we came
up; but a taller woman, quite a girl she seemed, who was at work near
by, had already knocked off, and was standing looking from Clara to Dick
with delighted eyes. None of the others paid much heed to us.

The blue-clad girl laid her hand on the carver's shoulder and said: "Now
Philippa, if you gobble up your work like that, you will soon have none
to do; and what will become of you then?"

The carver turned round hurriedly and showed us the face of a woman of
forty (or so she seemed), and said rather pettishly, but in a sweet
voice:

"Don't talk nonsense, Kate, and don't interrupt me if you can help it."
She stopped short when she saw us, then went on with the kind smile of
welcome which never failed us. "Thank you for coming to see us,
neighbours; but I am sure that you won't think me unkind if I go on with
my work, especially when I tell you that I was ill and unable to do
anything all through April and May; and this open-air and the sun and
the work together, and my feeling well again too, make a mere delight of
every hour to me; and excuse me, I must go on."

She fell to work accordingly on a carving in low relief of flowers and
figures, but talked on amidst her mallet strokes: "You see, we all think
this the prettiest place for a house up and down these reaches; and the
site has been so long encumbered with an unworthy one, that we masons
were determined to pay off fate and destiny for once, and build the
prettiest house we could compass here---and so---and so---"

Here she lapsed into mere carving, but the tall foreman came up and
said: "Yes, neighbours, that is it: so it is going to be all ashlar
because we want to carve a kind of a wreath of flowers and figures all
round it; and we have been much hindered by one thing or
other---Philippa's illness amongst others,---and though we could have
managed our wreath without her---"

"Could you, though?" grumbled the last-named from the face of the wall.

"Well, at any rate, she is our best carver, and it would not have been
kind to begin the carving without her. So you see," said he, looking at
Dick and me, "we really couldn't go haymaking, could we, neighbours? But
you see, we are getting on so fast now with this splendid weather, that
I think we may well spare a week or ten days at wheat-harvest; and won't
we go at that work then! Come down then to the acres that lie north and
by west here at our backs and you shall see good harvesters, neighbours.

"Hurrah, for a good brag!" called a voice from the scaffold above us;
"our foreman thinks that an easier job than putting one stone on
another!"

There was a general laugh at this sally, in which the tall foreman
joined; and with that we saw a lad bringing out a little table into the
shadow of the stone-shed, which he set down there, and then going back,
came out again with the inevitable big wickered flask and tall glasses,
whereon the foreman led us up to due seats on blocks of stone, and said:

"Well, neighbours, drink to my brag coming true, or I shall think you
don't believe me! Up there!" said he, hailing the scaffold, "are you
coming down for a glass?" Three of the workmen came running down the
ladder as men with good "building legs" will do; but the others didn't
answer, except the joker (if he must so be called), who called out
without turning round: "Excuse me, neighbours for not getting down. I
must get on: my work is not superintending, like the gaffer's yonder;
but, you fellows, send us up a glass to drink the haymakers' health." Of
course, Philippa would not turn away from her beloved work; but the
other woman carver came; she turned out to be Philippa's daughter, but
was a tall strong girl, black-haired and gipsey-like of face and
curiously solemn of manner. The rest gathered round us and clinked
glasses, and the men on the scaffold turned about and drank to our
healths; but the busy little woman by the door would have none of it
all, but only shrugged her shoulders when her daughter came up to her
and touched her.

So we shook hands and turned our backs on the Obstinate Refusers, went
down the slope to our boat, and before we had gone many steps heard the
full tune of tinkling trowels mingle with the humming of the bees and
the singing of the larks above the little plain of Basildon.
