\chapter{A market by the way}

We turned away from the river at once, and were soon in the main road
that runs through Hammersmith. But I should have had no guess as to
where I was, if I had not started from the waterside; for King Street
was gone, and the highway ran through wide sunny meadows and garden-like
tillage. The Creek, which we crossed at once, had been rescued from its
culvert, and as we went over its pretty bridge we saw its waters, yet
swollen by the tide, covered with gay boats of different sizes. There
were houses about, some on the road, some amongst the fields with
pleasant lanes leading down to them, and each surrounded by a teeming
garden. They were all pretty in design, and as solid as might be, but
countryfied in appearance, like yeomen's dwellings; some of them of red
brick like those by the river, but more of timber and plaster, which
were by the necessity of their construction so like mediaeval houses of
the same materials that I fairly felt as if I were alive in the
fourteenth century; a sensation helped out by the costume of the people
that we met or passed, in whose dress there was nothing "modern." Almost
everybody was gaily dressed, but especially the women, who were so
well-looking, or even so handsome, that I could scarcely refrain my
tongue from calling my companion's attention to the fact. Some faces I
saw that were thoughtful, and in these I noticed great nobility of
expression, but none that had a glimmer of unhappiness, and the greater
part (we came upon a good many people) were frankly and openly joyous.

I thought I knew the Broadway by the lie of the roads that still met
there. On the north side of the road was a range of buildings and
courts, low, but very handsomely built and ornamented, and in that way
forming a great contrast to the unpretentiousness of the houses round
about; while above this lower building rose the steep lead-covered roof
and the buttresses and higher part of the wall of a great hall, of a
splendid and exuberant style of architecture, of which one can say
little more than that it seemed to me to embrace the best qualities of
the Gothic of northern Europe with those of the Saracenic and Byzantine,
though there was no copying of any one of these styles. On the other,
the south side, of the road was an octagonal building with a high roof,
not unlike the Baptistry at Florence in outline, except that it was
surrounded by a lean-to that clearly made an arcade or cloisters to it:
it also was most delicately ornamented.

This whole mass of architecture which we had come upon so suddenly from
amidst the pleasant fields was not only exquisitely beautiful in itself,
but it bore upon it the expression of such generosity and abundance of
life that I was exhilarated to a pitch that I had never yet reached. I
fairly chuckled for pleasure. My friend seemed to understand it, and sat
looking on me with a pleased and affectionate interest. We had pulled up
amongst a crowd of carts, wherein sat handsome healthy-looking people,
men, women, and children very gaily dressed, and which were clearly
market carts, as they were full of very tempting-looking country
produce.

I said, "I need not ask if this is a market, for I see clearly that it
is; but what market is it that it is so splendid? And what is the
glorious hall there, and what is the building on the south side?"

"O," said he, "it is just our Hammersmith market; and I am glad you like
it so much, for we are really proud of it. Of course the hall inside is
our winter Mote-House; for in summer we mostly meet in the fields down
by the river opposite Barn Elms. The building on our right hand is our
theatre: I hope you like it."

"I should be a fool if I didn't," said I.

He blushed a little as he said: "I am glad of that, too, because I had a
hand in it; I made the great doors, which are of damascened bronze. We
will look at them later in the day, perhaps: but we ought to be getting
on now. As to the market, this is not one of our busy days; so we shall
do better with it another time, because you will see more people."

I thanked him, and said: "Are these the regular country people? What
very pretty girls there are amongst them."

As I spoke, my eye caught the face of a beautiful woman, tall,
dark-haired, and white-skinned, dressed in a pretty light-green dress in
honour of the season and the hot day, who smiled kindly on me, and more
kindly still, I thought on Dick; so I stopped a minute, but presently
went on:

"I ask because I do not see any of the country-looking people I should
have expected to see at a market---I mean selling things there."

"I don't understand," said he, "what kind of people you would expect to
see; nor quite what you mean by 'country' people. These are the
neighbours, and that like they run in the Thames valley. There are parts
of these islands which are rougher and rainier than we are here, and
there people are rougher in their dress; and they themselves are tougher
and more hard-bitten than we are to look at. But some people like their
looks better than ours; they say they have more character in
them---that's the word. Well, it's a matter of taste.---Anyhow, the
cross between us and them generally turns out well," added he,
thoughtfully.

I heard him, though my eyes were turned away from him, for that pretty
girl was just disappearing through the gate with her big basket of early
peas, and I felt that disappointed kind of feeling which overtakes one
when one has seen an interesting or lovely face in the streets which one
is never likely to see again; and I was silent a little. At last I said:
"What I mean is, that I haven't seen any poor people about---not one."

He knit his brows, looked puzzled, and said: "No, naturally; if anybody
is poorly, he is likely to be within doors, or at best crawling about
the garden: but I don't know of any one sick at present. Why should you
expect to see poorly people on the road?"

"No, no," I said; "I don't mean sick people. I mean poor people, you
know; rough people."

"No," said he, smiling merrily, "I really do not know. The fact is, you
must come along quick to my great-grandfather, who will understand you
better than I do. Come on, Greylocks!" Therewith he shook the reins, and
we jogged along merrily eastward.
