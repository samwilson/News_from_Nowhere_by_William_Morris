I lingered a little behind the others to have a stare at this house,
which, as I have told you, stood on the site of my old dwelling.

It was a longish building with its gable ends turned away from the road,
and long traceried windows coming rather low down set in the wall that
faced us. It was very handsomely built of red brick with a lead roof;
and high up above the windows there ran a frieze of figure subjects in
baked clay, very well executed, and designed with a force and directness
which I had never noticed in modern work before. The subjects I
recognised at once, and indeed was very particularly familiar with them.

However, all this I took in in a minute; for we were presently within
doors, and standing in a hall with a floor of marble mosaic and an open
timber roof. There were no windows on the side opposite to the river,
but arches below leading into chambers, one of which showed a glimpse of
a garden beyond, and above them a long space of wall gaily painted (in
fresco, I thought) with similar subjects to those of the frieze outside;
everything about the place was handsome and generously solid as to
material; and though it was not very large (somewhat smaller than Crosby
Hall perhaps), one felt in it that exhilarating sense of space and
freedom which satisfactory architecture always gives to an unanxious man
who is in the habit of using his eyes.

In this pleasant place, which of course I knew to be the hall of the
Guest House, three young women were flitting to and fro. As they were
the first of the sex I had seen on this eventful morning, I naturally
looked at them very attentively, and found them at least as good as the
gardens, the architecture, and the male men. As to their dress, which of
course I took note of, I should say that they were decently veiled with
drapery, and not bundled up with millinery; that they were clothed like
women, not upholstered like armchairs, as most women of our time are. In
short, their dress was somewhat between that of the ancient classical
costume and the simpler forms of the fourteenth century garments, though
it was clearly not an imitation of either: the materials were light and
gay to suit the season. As to the women themselves, it was pleasant
indeed to see them, they were so kind and happy-looking in expression of
face, so shapely and well-knit of body, and thoroughly healthy-looking
and strong. All were at least comely, and one of them very handsome and
regular of feature. They came up to us at once merrily and without the
least affectation of shyness, and all three shook hands with me as if I
were a friend newly come back from a long journey: though I could not
help noticing that they looked askance at my garments; for I had on my
clothes of last night, and at the best was never a dressy person.

A word or two from Robert the weaver, and they bustled about on our
behoof, and presently came and took us by the hands and led us to a
table in the pleasantest corner of the hall, where our breakfast was
spread for us; and, as we sat down, one of them hurried out by the
chambers aforesaid, and came back again in a little while with a great
bunch of roses, very different in size and quality to what Hammersmith
had been wont to grow, but very like the produce of an old country
garden. She hurried back thence into the buttery, and came back once
more with a delicately made glass, into which she put the flowers and
set them down in the midst of our table. One of the others, who had run
off also, then came back with a big cabbage-leaf filled with
strawberries, some of them barely ripe, and said as she set them on the
table, "There, now; I thought of that before I got up this morning; but
looking at the stranger here getting into your boat, Dick, put it out of
my head; so that I was not before \emph{all} the blackbirds: however,
there are a few about as good as you will get them anywhere in
Hammersmith this morning."

Robert patted her on the head in a friendly manner; and we fell to on
our breakfast, which was simple enough, but most delicately cooked, and
set on the table with much daintiness. The bread was particularly good,
and was of several different kinds, from the big, rather close,
dark-coloured, sweet-tasting farmhouse loaf, which was most to my
liking, to the thin pipe-stems of wheaten crust, such as I have eaten in
Turin.

As I was putting the first mouthfuls into my mouth my eye caught a
carved and gilded inscription on the panelling, behind what we should
have called the High Table in an Oxford college hall, and a familiar
name in it forced me to read it through. Thus it ran:

\begin{verbatim}
  "Guests and neighbours, on the site of this Guest-hall once stood
  the lecture-room of the Hammersmith Socialists.  Drink a glass to
  the memory!  May 1962."
\end{verbatim}

It is difficult to tell you how I felt as I read these words, and I
suppose my face showed how much I was moved, for both my friends looked
curiously at me, and there was silence between us for a little while.

Presently the weaver, who was scarcely so well mannered a man as the
ferryman, said to me rather awkwardly:

"Guest, we don't know what to call you: is there any indiscretion in
asking you your name?"

"Well," said I, "I have some doubts about it myself; so suppose you call
me Guest, which is a family name, you know, and add William to it if you
please."

Dick nodded kindly to me; but a shade of anxiousness passed over the
weaver's face, and he said---"I hope you don't mind my asking, but would
you tell me where you come from? I am curious about such things for good
reasons, literary reasons."

Dick was clearly kicking him underneath the table; but he was not much
abashed, and awaited my answer somewhat eagerly. As for me, I was just
going to blurt out "Hammersmith," when I bethought me what an
entanglement of cross purposes that would lead us into; so I took time
to invent a lie with circumstance, guarded by a little truth, and said:

"You see, I have been such a long time away from Europe that things seem
strange to me now; but I was born and bred on the edge of Epping Forest;
Walthamstow and Woodford, to wit."

"A pretty place, too," broke in Dick; "a very jolly place, now that the
trees have had time to grow again since the great clearing of houses in
1955."

Quoth the irrepressible weaver: "Dear neighbour, since you knew the
Forest some time ago, could you tell me what truth there is in the
rumour that in the nineteenth century the trees were all pollards?"

This was catching me on my archaeological natural-history side, and I
fell into the trap without any thought of where and when I was; so I
began on it, while one of the girls, the handsome one, who had been
scattering little twigs of lavender and other sweet-smelling herbs about
the floor, came near to listen, and stood behind me with her hand on my
shoulder, in which she held some of the plant that I used to call balm:
its strong sweet smell brought back to my mind my very early days in the
kitchen-garden at Woodford, and the large blue plums which grew on the
wall beyond the sweet-herb patch,---a connection of memories which all
boys will see at once.

I started off: "When I was a boy, and for long after, except for a piece
about Queen Elizabeth's Lodge, and for the part about High Beech, the
Forest was almost wholly made up of pollard hornbeams mixed with holly
thickets. But when the Corporation of London took it over about
twenty-five years ago, the topping and lopping, which was a part of the
old commoners' rights, came to an end, and the trees were let to grow.
But I have not seen the place now for many years, except once, when we
Leaguers went a pleasuring to High Beech. I was very much shocked then
to see how it was built-over and altered; and the other day we heard
that the philistines were going to landscape-garden it. But what you
were saying about the building being stopped and the trees growing is
only too good news;---only you know---"

At that point I suddenly remembered Dick's date, and stopped short
rather confused. The eager weaver didn't notice my confusion, but said
hastily, as if he were almost aware of his breach of good manners, "But,
I say, how old are you?"

Dick and the pretty girl both burst out laughing, as if Robert's conduct
were excusable on the grounds of eccentricity; and Dick said amidst his
laughter:

"Hold hard, Bob; this questioning of guests won't do. Why, much learning
is spoiling you. You remind me of the radical cobblers in the silly old
novels, who, according to the authors, were prepared to trample down all
good manners in the pursuit of utilitarian knowledge. The fact is, I
begin to think that you have so muddled your head with mathematics, and
with grubbing into those idiotic old books about political economy (he
he!), that you scarcely know how to behave. Really, it is about time for
you to take to some open-air work, so that you may clear away the
cobwebs from your brain."

The weaver only laughed good-humouredly; and the girl went up to him and
patted his cheek and said laughingly, "Poor fellow! he was born so."

As for me, I was a little puzzled, but I laughed also, partly for
company's sake, and partly with pleasure at their unanxious happiness
and good temper; and before Robert could make the excuse to me which he
was getting ready, I said:

"But neighbours" (I had caught up that word), "I don't in the least mind
answering questions, when I can do so: ask me as many as you please;
it's fun for me. I will tell you all about Epping Forest when I was a
boy, if you please; and as to my age, I'm not a fine lady, you know, so
why shouldn't I tell you? I'm hard on fifty-six."

In spite of the recent lecture on good manners, the weaver could not
help giving a long "whew" of astonishment, and the others were so amused
by his \emph{naivete} that the merriment flitted all over their faces,
though for courtesy's sake they forbore actual laughter; while I looked
from one to the other in a puzzled manner, and at last said:

"Tell me, please, what is amiss: you know I want to learn from you. And
please laugh; only tell me."

Well, they \emph{did} laugh, and I joined them again, for the
above-stated reasons. But at last the pretty woman said coaxingly---

"Well, well, he \emph{is} rude, poor fellow! but you see I may as well
tell you what he is thinking about: he means that you look rather old
for your age. But surely there need be no wonder in that, since you have
been travelling; and clearly from all you have been saying, in unsocial
countries. It has often been said, and no doubt truly, that one ages
very quickly if one lives amongst unhappy people. Also they say that
southern England is a good place for keeping good looks." She blushed
and said: "How old am I, do you think?"

"Well," quoth I, "I have always been told that a woman is as old as she
looks, so without offence or flattery, I should say that you were
twenty."

She laughed merrily, and said, "I am well served out for fishing for
compliments, since I have to tell you the truth, to wit, that I am
forty-two."

I stared at her, and drew musical laughter from her again; but I might
well stare, for there was not a careful line on her face; her skin was
as smooth as ivory, her cheeks full and round, her lips as red as the
roses she had brought in; her beautiful arms, which she had bared for
her work, firm and well-knit from shoulder to wrist. She blushed a
little under my gaze, though it was clear that she had taken me for a
man of eighty; so to pass it off I said---

"Well, you see, the old saw is proved right again, and I ought not to
have let you tempt me into asking you a rude question."

She laughed again, and said: "Well, lads, old and young, I must get to
my work now. We shall be rather busy here presently; and I want to clear
it off soon, for I began to read a pretty old book yesterday, and I want
to get on with it this morning: so good-bye for the present."

She waved a hand to us, and stepped lightly down the hall, taking (as
Scott says) at least part of the sun from our table as she went.

When she was gone, Dick said "Now guest, won't you ask a question or two
of our friend here? It is only fair that you should have your turn."

"I shall be very glad to answer them," said the weaver.

"If I ask you any questions, sir," said I, "they will not be very
severe; but since I hear that you are a weaver, I should like to ask you
something about that craft, as I am---or was---interested in it."

"Oh," said he, "I shall not be of much use to you there, I'm afraid. I
only do the most mechanical kind of weaving, and am in fact but a poor
craftsman, unlike Dick here. Then besides the weaving, I do a little
with machine printing and composing, though I am little use at the finer
kinds of printing; and moreover machine printing is beginning to die
out, along with the waning of the plague of book-making, so I have had
to turn to other things that I have a taste for, and have taken to
mathematics; and also I am writing a sort of antiquarian book about the
peaceable and private history, so to say, of the end of the nineteenth
century,---more for the sake of giving a picture of the country before
the fighting began than for anything else. That was why I asked you
those questions about Epping Forest. You have rather puzzled me, I
confess, though your information was so interesting. But later on, I
hope, we may have some more talk together, when our friend Dick isn't
here. I know he thinks me rather a grinder, and despises me for not
being very deft with my hands: that's the way nowadays. From what I have
read of the nineteenth century literature (and I have read a good deal),
it is clear to me that this is a kind of revenge for the stupidity of
that day, which despised everybody who \emph{could} use his hands. But
Dick, old fellow, \emph{Ne quid nimis}! Don't overdo it!"

"Come now," said Dick, "am I likely to? Am I not the most tolerant man
in the world? Am I not quite contented so long as you don't make me
learn mathematics, or go into your new science of aesthetics, and let me
do a little practical aesthetics with my gold and steel, and the
blowpipe and the nice little hammer? But, hillo! here comes another
questioner for you, my poor guest. I say, Bob, you must help me to
defend him now."

"Here, Boffin," he cried out, after a pause; "here we are, if you must
have it!"

I looked over my shoulder, and saw something flash and gleam in the
sunlight that lay across the hall; so I turned round, and at my ease saw
a splendid figure slowly sauntering over the pavement; a man whose
surcoat was embroidered most copiously as well as elegantly, so that the
sun flashed back from him as if he had been clad in golden armour. The
man himself was tall, dark-haired, and exceedingly handsome, and though
his face was no less kindly in expression than that of the others, he
moved with that somewhat haughty mien which great beauty is apt to give
to both men and women. He came and sat down at our table with a smiling
face, stretching out his long legs and hanging his arm over the chair in
the slowly graceful way which tall and well-built people may use without
affectation. He was a man in the prime of life, but looked as happy as a
child who has just got a new toy. He bowed gracefully to me and said---

"I see clearly that you are the guest, of whom Annie has just told me,
who have come from some distant country that does not know of us, or our
ways of life. So I daresay you would not mind answering me a few
questions; for you see---"

Here Dick broke in: "No, please, Boffin! let it alone for the present.
Of course you want the guest to be happy and comfortable; and how can
that be if he has to trouble himself with answering all sorts of
questions while he is still confused with the new customs and people
about him? No, no: I am going to take him where he can ask questions
himself, and have them answered; that is, to my great-grandfather in
Bloomsbury: and I am sure you can't have anything to say against that.
So instead of bothering, you had much better go out to James Allen's and
get a carriage for me, as I shall drive him up myself; and please tell
Jim to let me have the old grey, for I can drive a wherry much better
than a carriage. Jump up, old fellow, and don't be disappointed; our
guest will keep himself for you and your stories."

I stared at Dick; for I wondered at his speaking to such a
dignified-looking personage so familiarly, not to say curtly; for I
thought that this Mr. Boffin, in spite of his well-known name out of
Dickens, must be at the least a senator of these strange people.
However, he got up and said, "All right, old oar-wearer, whatever you
like; this is not one of my busy days; and though" (with a condescending
bow to me) "my pleasure of a talk with this learned guest is put off, I
admit that he ought to see your worthy kinsman as soon as possible.
Besides, perhaps he will be the better able to answer \emph{my}
questions after his own have been answered."

And therewith he turned and swung himself out of the hall.

When he was well gone, I said: "Is it wrong to ask what Mr. Boffin is?
whose name, by the way, reminds me of many pleasant hours passed in
reading Dickens."

Dick laughed. "Yes, yes," said he, "as it does us. I see you take the
allusion. Of course his real name is not Boffin, but Henry Johnson; we
only call him Boffin as a joke, partly because he is a dustman, and
partly because he will dress so showily, and get as much gold on him as
a baron of the Middle Ages. As why should he not if he likes? only we
are his special friends, you know, so of course we jest with him."

I held my tongue for some time after that; but Dick went on:

"He is a capital fellow, and you can't help liking him; but he has a
weakness: he will spend his time in writing reactionary novels, and is
very proud of getting the local colour right, as he calls it; and as he
thinks you come from some forgotten corner of the earth, where people
are unhappy, and consequently interesting to a story-teller, he thinks
he might get some information out of you. O, he will be quite
straightforward with you, for that matter. Only for your own comfort
beware of him!"

"Well, Dick," said the weaver, doggedly, "I think his novels are very
good."

"Of course you do," said Dick; "birds of a feather flock together;
mathematics and antiquarian novels stand on much the same footing. But
here he comes again."

And in effect the Golden Dustman hailed us from the hall-door; so we all
got up and went into the porch, before which, with a strong grey horse
in the shafts, stood a carriage ready for us which I could not help
noticing. It was light and handy, but had none of that sickening
vulgarity which I had known as inseparable from the carriages of our
time, especially the "elegant" ones, but was as graceful and pleasant in
line as a Wessex waggon. We got in, Dick and I. The girls, who had come
into the porch to see us off, waved their hands to us; the weaver nodded
kindly; the dustman bowed as gracefully as a troubadour; Dick shook the
reins, and we were off.
