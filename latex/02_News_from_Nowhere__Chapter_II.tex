\chapter{A morning bath}

Well, I awoke, and found that I had kicked my bedclothes off; and no
wonder, for it was hot and the sun shining brightly. I jumped up and
washed and hurried on my clothes, but in a hazy and half-awake
condition, as if I had slept for a long, long while, and could not shake
off the weight of slumber. In fact, I rather took it for granted that I
was at home in my own room than saw that it was so.

When I was dressed, I felt the place so hot that I made haste to get out
of the room and out of the house; and my first feeling was a delicious
relief caused by the fresh air and pleasant breeze; my second, as I
began to gather my wits together, mere measureless wonder: for it was
winter when I went to bed the last night, and now, by witness of the
river-side trees, it was summer, a beautiful bright morning seemingly of
early June. However, there was still the Thames sparkling under the sun,
and near high water, as last night I had seen it gleaming under the
moon.

I had by no means shaken off the feeling of oppression, and wherever I
might have been should scarce have been quite conscious of the place; so
it was no wonder that I felt rather puzzled in despite of the familiar
face of the Thames. Withal I felt dizzy and queer; and remembering that
people often got a boat and had a swim in mid-stream, I thought I would
do no less. It seems very early, quoth I to myself, but I daresay I
shall find someone at Biffin's to take me. However, I didn't get as far
as Biffin's, or even turn to my left thitherward, because just then I
began to see that there was a landing-stage right before me in front of
my house: in fact, on the place where my next-door neighbour had rigged
one up, though somehow it didn't look like that either. Down I went on
to it, and sure enough among the empty boats moored to it lay a man on
his sculls in a solid-looking tub of a boat clearly meant for bathers.
He nodded to me, and bade me good-morning as if he expected me, so I
jumped in without any words, and he paddled away quietly as I peeled for
my swim. As we went, I looked down on the water, and couldn't help
saying---

"How clear the water is this morning!"

"Is it?" said he; "I didn't notice it. You know the flood-tide always
thickens it a bit."

"H'm," said I, "I have seen it pretty muddy even at half-ebb."

He said nothing in answer, but seemed rather astonished; and as he now
lay just stemming the tide, and I had my clothes off, I jumped in
without more ado. Of course when I had my head above water again I
turned towards the tide, and my eyes naturally sought for the bridge,
and so utterly astonished was I by what I saw, that I forgot to strike
out, and went spluttering under water again, and when I came up made
straight for the boat; for I felt that I must ask some questions of my
waterman, so bewildering had been the half-sight I had seen from the
face of the river with the water hardly out of my eyes; though by this
time I was quit of the slumbrous and dizzy feeling, and was wide-awake
and clear-headed.

As I got in up the steps which he had lowered, and he held out his hand
to help me, we went drifting speedily up towards Chiswick; but now he
caught up the sculls and brought her head round again, and said---"A
short swim, neighbour; but perhaps you find the water cold this morning,
after your journey. Shall I put you ashore at once, or would you like to
go down to Putney before breakfast?"

He spoke in a way so unlike what I should have expected from a
Hammersmith waterman, that I stared at him, as I answered, "Please to
hold her a little; I want to look about me a bit."

"All right," he said; "it's no less pretty in its way here than it is
off Barn Elms; it's jolly everywhere this time in the morning. I'm glad
you got up early; it's barely five o'clock yet."

If I was astonished with my sight of the river banks, I was no less
astonished at my waterman, now that I had time to look at him and see
him with my head and eyes clear.

He was a handsome young fellow, with a peculiarly pleasant and friendly
look about his eyes,---an expression which was quite new to me then,
though I soon became familiar with it. For the rest, he was dark-haired
and berry-brown of skin, well-knit and strong, and obviously used to
exercising his muscles, but with nothing rough or coarse about him, and
clean as might be. His dress was not like any modern work-a-day clothes
I had seen, but would have served very well as a costume for a picture
of fourteenth century life: it was of dark blue cloth, simple enough,
but of fine web, and without a stain on it. He had a brown leather belt
round his waist, and I noticed that its clasp was of damascened steel
beautifully wrought. In short, he seemed to be like some specially manly
and refined young gentleman, playing waterman for a spree, and I
concluded that this was the case.

I felt that I must make some conversation; so I pointed to the Surrey
bank, where I noticed some light plank stages running down the
foreshore, with windlasses at the landward end of them, and said, "What
are they doing with those things here? If we were on the Tay, I should
have said that they were for drawing the salmon nets; but here---"

"Well," said he, smiling, "of course that is what they \emph{are} for.
Where there are salmon, there are likely to be salmon-nets, Tay or
Thames; but of course they are not always in use; we don't want salmon
\emph{every} day of the season."

I was going to say, "But is this the Thames?" but held my peace in my
wonder, and turned my bewildered eyes eastward to look at the bridge
again, and thence to the shores of the London river; and surely there
was enough to astonish me. For though there was a bridge across the
stream and houses on its banks, how all was changed from last night! The
soap-works with their smoke-vomiting chimneys were gone; the engineer's
works gone; the lead-works gone; and no sound of rivetting and hammering
came down the west wind from Thorneycroft's. Then the bridge! I had
perhaps dreamed of such a bridge, but never seen such an one out of an
illuminated manuscript; for not even the Ponte Vecchio at Florence came
anywhere near it. It was of stone arches, splendidly solid, and as
graceful as they were strong; high enough also to let ordinary river
traffic through easily. Over the parapet showed quaint and fanciful
little buildings, which I supposed to be booths or shops, beset with
painted and gilded vanes and spirelets. The stone was a little
weathered, but showed no marks of the grimy sootiness which I was used
to on every London building more than a year old. In short, to me a
wonder of a bridge.

The sculler noted my eager astonished look, and said, as if in answer to
my thoughts---

"Yes, it \emph{is} a pretty bridge, isn't it? Even the up-stream
bridges, which are so much smaller, are scarcely daintier, and the
down-stream ones are scarcely more dignified and stately."

I found myself saying, almost against my will, "How old is it?"

"Oh, not very old," he said; "it was built or at least opened, in 2003.
There used to be a rather plain timber bridge before then."

The date shut my mouth as if a key had been turned in a padlock fixed to
my lips; for I saw that something inexplicable had happened, and that if
I said much, I should be mixed up in a game of cross questions and
crooked answers. So I tried to look unconcerned, and to glance in a
matter-of-course way at the banks of the river, though this is what I
saw up to the bridge and a little beyond; say as far as the site of the
soap-works. Both shores had a line of very pretty houses, low and not
large, standing back a little way from the river; they were mostly built
of red brick and roofed with tiles, and looked, above all, comfortable,
and as if they were, so to say, alive, and sympathetic with the life of
the dwellers in them. There was a continuous garden in front of them,
going down to the water's edge, in which the flowers were now blooming
luxuriantly, and sending delicious waves of summer scent over the
eddying stream. Behind the houses, I could see great trees rising,
mostly planes, and looking down the water there were the reaches towards
Putney almost as if they were a lake with a forest shore, so thick were
the big trees; and I said aloud, but as if to myself---

"Well, I'm glad that they have not built over Barn Elms."

I blushed for my fatuity as the words slipped out of my mouth, and my
companion looked at me with a half smile which I thought I understood;
so to hide my confusion I said, "Please take me ashore now: I want to
get my breakfast."

He nodded, and brought her head round with a sharp stroke, and in a
trice we were at the landing-stage again. He jumped out and I followed
him; and of course I was not surprised to see him wait, as if for the
inevitable after-piece that follows the doing of a service to a
fellow-citizen. So I put my hand into my waistcoat-pocket, and said,
"How much?" though still with the uncomfortable feeling that perhaps I
was offering money to a gentleman.

He looked puzzled, and said, "How much? I don't quite understand what
you are asking about. Do you mean the tide? If so, it is close on the
turn now."

I blushed, and said, stammering, "Please don't take it amiss if I ask
you; I mean no offence: but what ought I to pay you? You see I am a
stranger, and don't know your customs---or your coins."

And therewith I took a handful of money out of my pocket, as one does in
a foreign country. And by the way, I saw that the silver had oxydised,
and was like a blackleaded stove in colour.

He still seemed puzzled, but not at all offended; and he looked at the
coins with some curiosity. I thought, Well after all, he \emph{is} a
waterman, and is considering what he may venture to take. He seems such
a nice fellow that I'm sure I don't grudge him a little over-payment. I
wonder, by the way, whether I couldn't hire him as a guide for a day or
two, since he is so intelligent.

Therewith my new friend said thoughtfully:

"I think I know what you mean. You think that I have done you a service;
so you feel yourself bound to give me something which I am not to give
to a neighbour, unless he has done something special for me. I have
heard of this kind of thing; but pardon me for saying, that it seems to
us a troublesome and roundabout custom; and we don't know how to manage
it. And you see this ferrying and giving people casts about the water is
my \emph{business}, which I would do for anybody; so to take gifts in
connection with it would look very queer. Besides, if one person gave me
something, then another might, and another, and so on; and I hope you
won't think me rude if I say that I shouldn't know where to stow away so
many mementos of friendship."

And he laughed loud and merrily, as if the idea of being paid for his
work was a very funny joke. I confess I began to be afraid that the man
was mad, though he looked sane enough; and I was rather glad to think
that I was a good swimmer, since we were so close to a deep swift
stream. However, he went on by no means like a madman:

"As to your coins, they are curious, but not very old; they seem to be
all of the reign of Victoria; you might give them to some
scantily-furnished museum. Ours has enough of such coins, besides a fair
number of earlier ones, many of which are beautiful, whereas these
nineteenth century ones are so beastly ugly, ain't they? We have a piece
of Edward III., with the king in a ship, and little leopards and
fleurs-de-lys all along the gunwale, so delicately worked. You see," he
said, with something of a smirk, "I am fond of working in gold and fine
metals; this buckle here is an early piece of mine."

No doubt I looked a little shy of him under the influence of that doubt
as to his sanity. So he broke off short, and said in a kind voice:

"But I see that I am boring you, and I ask your pardon. For, not to
mince matters, I can tell that you \emph{are} a stranger, and must come
from a place very unlike England. But also it is clear that it won't do
to overdose you with information about this place, and that you had best
suck it in little by little. Further, I should take it as very kind in
you if you would allow me to be the showman of our new world to you,
since you have stumbled on me first. Though indeed it will be a mere
kindness on your part, for almost anybody would make as good a guide,
and many much better."

There certainly seemed no flavour in him of Colney Hatch; and besides I
thought I could easily shake him off if it turned out that he really was
mad; so I said:

"It is a very kind offer, but it is difficult for me to accept it,
unless---" I was going to say, Unless you will let me pay you properly;
but fearing to stir up Colney Hatch again, I changed the sentence into,
"I fear I shall be taking you away from your work---or your amusement."

"O," he said, "don't trouble about that, because it will give me an
opportunity of doing a good turn to a friend of mine, who wants to take
my work here. He is a weaver from Yorkshire, who has rather overdone
himself between his weaving and his mathematics, both indoor work, you
see; and being a great friend of mine, he naturally came to me to get
him some outdoor work. If you think you can put up with me, pray take me
as your guide."

He added presently: "It is true that I have promised to go up-stream to
some special friends of mine, for the hay-harvest; but they won't be
ready for us for more than a week: and besides, you might go with me,
you know, and see some very nice people, besides making notes of our
ways in Oxfordshire. You could hardly do better if you want to see the
country."

I felt myself obliged to thank him, whatever might come of it; and he
added eagerly:

"Well, then, that's settled. I will give my friend call; he is living in
the Guest House like you, and if he isn't up yet, he ought to be this
fine summer morning."

Therewith he took a little silver bugle-horn from his girdle and blew
two or three sharp but agreeable notes on it; and presently from the
house which stood on the site of my old dwelling (of which more
hereafter) another young man came sauntering towards us. He was not so
well-looking or so strongly made as my sculler friend, being
sandy-haired, rather pale, and not stout-built; but his face was not
wanting in that happy and friendly expression which I had noticed in his
friend. As he came up smiling towards us, I saw with pleasure that I
must give up the Colney Hatch theory as to the waterman, for no two
madmen ever behaved as they did before a sane man. His dress also was of
the same cut as the first man's, though somewhat gayer, the surcoat
being light green with a golden spray embroidered on the breast, and his
belt being of filagree silver-work.

He gave me good-day very civilly, and greeting his friend joyously,
said:

"Well, Dick, what is it this morning? Am I to have my work, or rather
your work? I dreamed last night that we were off up the river fishing."

"All right, Bob," said my sculler; "you will drop into my place, and if
you find it too much, there is George Brightling on the look out for a
stroke of work, and he lives close handy to you. But see, here is a
stranger who is willing to amuse me to-day by taking me as his guide
about our country-side, and you may imagine I don't want to lose the
opportunity; so you had better take to the boat at once. But in any case
I shouldn't have kept you out of it for long, since I am due in the
hay-fields in a few days."

The newcomer rubbed his hands with glee, but turning to me, said in a
friendly voice:

"Neighbour, both you and friend Dick are lucky, and will have a good
time to-day, as indeed I shall too. But you had better both come in with
me at once and get something to eat, lest you should forget your dinner
in your amusement. I suppose you came into the Guest House after I had
gone to bed last night?"

I nodded, not caring to enter into a long explanation which would have
led to nothing, and which in truth by this time I should have begun to
doubt myself. And we all three turned toward the door of the Guest
House.
