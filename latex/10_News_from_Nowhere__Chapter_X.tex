"Well," said the old man, shifting in his chair, "you must get on with
your questions, Guest; I have been some time answering this first one."

Said I: "I want an extra word or two about your ideas of education;
although I gathered from Dick that you let your children run wild and
didn't teach them anything; and in short, that you have so refined your
education, that now you have none."

"Then you gathered left-handed," quoth he. "But of course I understand
your point of view about education, which is that of times past, when
'the struggle for life,' as men used to phrase it (\emph{i.e.}, the
struggle for a slave's rations on one side, and for a bouncing share of
the slave-holders' privilege on the other), pinched 'education' for most
people into a niggardly dole of not very accurate information; something
to be swallowed by the beginner in the art of living whether he liked it
or not, and was hungry for it or not: and which had been chewed and
digested over and over again by people who didn't care about it in order
to serve it out to other people who didn't care about it."

I stopped the old man's rising wrath by a laugh, and said: "Well,
\emph{you} were not taught that way, at any rate, so you may let your
anger run off you a little."

"True, true," said he, smiling. "I thank you for correcting my
ill-temper: I always fancy myself as living in any period of which we
may be speaking. But, however, to put it in a cooler way: you expected
to see children thrust into schools when they had reached an age
conventionally supposed to be the due age, whatever their varying
faculties and dispositions might be, and when there, with like disregard
to facts to be subjected to a certain conventional course of 'learning.'
My friend, can't you see that such a proceeding means ignoring the fact
of \emph{growth}, bodily and mental? No one could come out of such a
mill uninjured; and those only would avoid being crushed by it who would
have the spirit of rebellion strong in them. Fortunately most children
have had that at all times, or I do not know that we should ever have
reached our present position. Now you see what it all comes to. In the
old times all this was the result of \emph{poverty}. In the nineteenth
century, society was so miserably poor, owing to the systematised
robbery on which it was founded, that real education was impossible for
anybody. The whole theory of their so-called education was that it was
necessary to shove a little information into a child, even if it were by
means of torture, and accompanied by twaddle which it was well known was
of no use, or else he would lack information lifelong: the hurry of
poverty forbade anything else. All that is past; we are no longer
hurried, and the information lies ready to each one's hand when his own
inclinations impel him to seek it. In this as in other matters we have
become wealthy: we can afford to give ourselves time to grow."

"Yes," said I, "but suppose the child, youth, man, never wants the
information, never grows in the direction you might hope him to do:
suppose, for instance, he objects to learning arithmetic or mathematics;
you can't force him when he \emph{is} grown; can't you force him while
he is growing, and oughtn't you to do so?"

"Well," said he, "were you forced to learn arithmetic and mathematics?"

"A little," said I.

"And how old are you now?"

"Say fifty-six," said I.

"And how much arithmetic and mathematics do you know now?" quoth the old
man, smiling rather mockingly.

Said I: "None whatever, I am sorry to say."

Hammond laughed quietly, but made no other comment on my admission, and
I dropped the subject of education, perceiving him to be hopeless on
that side.

I thought a little, and said: "You were speaking just now of households:
that sounded to me a little like the customs of past times; I should
have thought you would have lived more in public."

"Phalangsteries, eh?" said he. "Well, we live as we like, and we like to
live as a rule with certain house-mates that we have got used to.
Remember, again, that poverty is extinct, and that the Fourierist
phalangsteries and all their kind, as was but natural at the time,
implied nothing but a refuge from mere destitution. Such a way of life
as that, could only have been conceived of by people surrounded by the
worst form of poverty. But you must understand therewith, that though
separate households are the rule amongst us, and though they differ in
their habits more or less, yet no door is shut to any good-tempered
person who is content to live as the other house-mates do: only of
course it would be unreasonable for one man to drop into a household and
bid the folk of it to alter their habits to please him, since he can go
elsewhere and live as he pleases. However, I need not say much about all
this, as you are going up the river with Dick, and will find out for
yourself by experience how these matters are managed."

After a pause, I said: "Your big towns, now; how about them? London,
which---which I have read about as the modern Babylon of civilization,
seems to have disappeared."

"Well, well," said old Hammond, "perhaps after all it is more like
ancient Babylon now than the 'modern Babylon' of the nineteenth century
was. But let that pass. After all, there is a good deal of population in
places between here and Hammersmith; nor have you seen the most populous
part of the town yet."

"Tell me, then," said I, "how is it towards the east?"

Said he: "Time was when if you mounted a good horse and rode straight
away from my door here at a round trot for an hour and a half; you would
still be in the thick of London, and the greater part of that would be
'slums,' as they were called; that is to say, places of torture for
innocent men and women; or worse, stews for rearing and breeding men and
women in such degradation that that torture should seem to them mere
ordinary and natural life."

"I know, I know," I said, rather impatiently. "That was what was; tell
me something of what is. Is any of that left?"

"Not an inch," said he; "but some memory of it abides with us, and I am
glad of it. Once a year, on May-day, we hold a solemn feast in those
easterly communes of London to commemorate The Clearing of Misery, as it
is called. On that day we have music and dancing, and merry games and
happy feasting on the site of some of the worst of the old slums, the
traditional memory of which we have kept. On that occasion the custom is
for the prettiest girls to sing some of the old revolutionary songs, and
those which were the groans of the discontent, once so hopeless, on the
very spots where those terrible crimes of class-murder were committed
day by day for so many years. To a man like me, who have studied the
past so diligently, it is a curious and touching sight to see some
beautiful girl, daintily clad, and crowned with flowers from the
neighbouring meadows, standing amongst the happy people, on some mound
where of old time stood the wretched apology for a house, a den in which
men and women lived packed amongst the filth like pilchards in a cask;
lived in such a way that they could only have endured it, as I said just
now, by being degraded out of humanity---to hear the terrible words of
threatening and lamentation coming from her sweet and beautiful lips,
and she unconscious of their real meaning: to hear her, for instance,
singing Hood's Song of the Shirt, and to think that all the time she
does not understand what it is all about---a tragedy grown inconceivable
to her and her listeners. Think of that, if you can, and of how glorious
life is grown!"

"Indeed," said I, "it is difficult for me to think of it."

And I sat watching how his eyes glittered, and how the fresh life seemed
to glow in his face, and I wondered how at his age he should think of
the happiness of the world, or indeed anything but his coming dinner.

"Tell me in detail," said I, "what lies east of Bloomsbury now?"

Said he: "There are but few houses between this and the outer part of
the old city; but in the city we have a thickly-dwelling population. Our
forefathers, in the first clearing of the slums, were not in a hurry to
pull down the houses in what was called at the end of the nineteenth
century the business quarter of the town, and what later got to be known
as the Swindling Kens. You see, these houses, though they stood
hideously thick on the ground, were roomy and fairly solid in building,
and clean, because they were not used for living in, but as mere
gambling booths; so the poor people from the cleared slums took them for
lodgings and dwelt there, till the folk of those days had time to think
of something better for them; so the buildings were pulled down so
gradually that people got used to living thicker on the ground there
than in most places; therefore it remains the most populous part of
London, or perhaps of all these islands. But it is very pleasant there,
partly because of the splendour of the architecture, which goes further
than what you will see elsewhere. However, this crowding, if it may be
called so, does not go further than a street called Aldgate, a name
which perhaps you may have heard of. Beyond that the houses are
scattered wide about the meadows there, which are very beautiful,
especially when you get on to the lovely river Lea (where old Isaak
Walton used to fish, you know) about the places called Stratford and Old
Ford, names which of course you will not have heard of, though the
Romans were busy there once upon a time."

Not heard of them! thought I to myself. How strange! that I who had seen
the very last remnant of the pleasantness of the meadows by the Lea
destroyed, should have heard them spoken of with pleasantness come back
to them in full measure.

Hammond went on: "When you get down to the Thames side you come on the
Docks, which are works of the nineteenth century, and are still in use,
although not so thronged as they once were, since we discourage
centralisation all we can, and we have long ago dropped the pretension
to be the market of the world. About these Docks are a good few houses,
which, however, are not inhabited by many people permanently; I mean,
those who use them come and go a good deal, the place being too low and
marshy for pleasant dwelling. Past the Docks eastward and landward it is
all flat pasture, once marsh, except for a few gardens, and there are
very few permanent dwellings there: scarcely anything but a few sheds,
and cots for the men who come to look after the great herds of cattle
pasturing there. But however, what with the beasts and the men, and the
scattered red-tiled roofs and the big hayricks, it does not make a bad
holiday to get a quiet pony and ride about there on a sunny afternoon of
autumn, and look over the river and the craft passing up and down, and
on to Shooters' Hill and the Kentish uplands, and then turn round to the
wide green sea of the Essex marsh-land, with the great domed line of the
sky, and the sun shining down in one flood of peaceful light over the
long distance. There is a place called Canning's Town, and further out,
Silvertown, where the pleasant meadows are at their pleasantest:
doubtless they were once slums, and wretched enough."

The names grated on my ear, but I could not explain why to him. So I
said: "And south of the river, what is it like?"

He said: "You would find it much the same as the land about Hammersmith.
North, again, the land runs up high, and there is an agreeable and
well-built town called Hampstead, which fitly ends London on that side.
It looks down on the north-western end of the forest you passed
through."

I smiled. "So much for what was once London," said I. "Now tell me about
the other towns of the country."

He said: "As to the big murky places which were once, as we know, the
centres of manufacture, they have, like the brick and mortar desert of
London, disappeared; only, since they were centres of nothing but
'manufacture,' and served no purpose but that of the gambling market,
they have left less signs of their existence than London. Of course, the
great change in the use of mechanical force made this an easy matter,
and some approach to their break-up as centres would probably have taken
place, even if we had not changed our habits so much: but they being
such as they were, no sacrifice would have seemed too great a price to
pay for getting rid of the 'manufacturing districts,' as they used to be
called. For the rest, whatever coal or mineral we need is brought to
grass and sent whither it is needed with as little as possible of dirt,
confusion, and the distressing of quiet people's lives. One is tempted
to believe from what one has read of the condition of those districts in
the nineteenth century, that those who had them under their power
worried, befouled, and degraded men out of malice prepense: but it was
not so; like the mis-education of which we were talking just now, it
came of their dreadful poverty. They were obliged to put up with
everything, and even pretend that they liked it; whereas we can now deal
with things reasonably, and refuse to be saddled with what we do not
want."

I confess I was not sorry to cut short with a question his
glorifications of the age he lived in. Said I: "How about the smaller
towns? I suppose you have swept those away entirely?"

"No, no," said he, "it hasn't gone that way. On the contrary, there has
been but little clearance, though much rebuilding, in the smaller towns.
Their suburbs, indeed, when they had any, have melted away into the
general country, and space and elbow-room has been got in their centres:
but there are the towns still with their streets and squares and
market-places; so that it is by means of these smaller towns that we of
to-day can get some kind of idea of what the towns of the older world
were like;---I mean to say at their best."

"Take Oxford, for instance," said I.

"Yes," said he, "I suppose Oxford was beautiful even in the nineteenth
century. At present it has the great interest of still preserving a
great mass of pre-commercial building, and is a very beautiful place,
yet there are many towns which have become scarcely less beautiful."

Said I: "In passing, may I ask if it is still a place of learning?"

"Still?" said he, smiling. "Well, it has reverted to some of its best
traditions; so you may imagine how far it is from its nineteenth-century
position. It is real learning, knowledge cultivated for its own
sake---the Art of Knowledge, in short---which is followed there, not the
Commercial learning of the past. Though perhaps you do not know that in
the nineteenth century Oxford and its less interesting sister Cambridge
became definitely commercial. They (and especially Oxford) were the
breeding places of a peculiar class of parasites, who called themselves
cultivated people; they were indeed cynical enough, as the so-called
educated classes of the day generally were; but they affected an
exaggeration of cynicism in order that they might be thought knowing and
worldly-wise. The rich middle classes (they had no relation with the
working classes) treated them with the kind of contemptuous toleration
with which a mediaeval baron treated his jester; though it must be said
that they were by no means so pleasant as the old jesters were, being,
in fact, \emph{the} bores of society. They were laughed at,
despised---and paid. Which last was what they aimed at."

Dear me! thought I, how apt history is to reverse contemporary
judgments. Surely only the worst of them were as bad as that. But I must
admit that they were mostly prigs, and that they \emph{were} commercial.
I said aloud, though more to myself than to Hammond, "Well, how could
they be better than the age that made them?"

"True," he said, "but their pretensions were higher."

"Were they?" said I, smiling.

"You drive me from corner to corner," said he, smiling in turn. "Let me
say at least that they were a poor sequence to the aspirations of Oxford
of 'the barbarous Middle Ages.'"

"Yes, that will do," said I.

"Also," said Hammond, "what I have been saying of them is true in the
main. But ask on!"

I said: "We have heard about London and the manufacturing districts and
the ordinary towns: how about the villages?"

Said Hammond: "You must know that toward the end of the nineteenth
century the villages were almost destroyed, unless where they became
mere adjuncts to the manufacturing districts, or formed a sort of minor
manufacturing districts themselves. Houses were allowed to fall into
decay and actual ruin; trees were cut down for the sake of the few
shillings which the poor sticks would fetch; the building became
inexpressibly mean and hideous. Labour was scarce; but wages fell
nevertheless. All the small country arts of life which once added to the
little pleasures of country people were lost. The country produce which
passed through the hands of the husbandmen never got so far as their
mouths. Incredible shabbiness and niggardly pinching reigned over the
fields and acres which, in spite of the rude and careless husbandry of
the times, were so kind and bountiful. Had you any inkling of all this?"

"I have heard that it was so," said I "but what followed?"

"The change," said Hammond, "which in these matters took place very
early in our epoch, was most strangely rapid. People flocked into the
country villages, and, so to say, flung themselves upon the freed land
like a wild beast upon his prey; and in a very little time the villages
of England were more populous than they had been since the fourteenth
century, and were still growing fast. Of course, this invasion of the
country was awkward to deal with, and would have created much misery, if
the folk had still been under the bondage of class monopoly. But as it
was, things soon righted themselves. People found out what they were fit
for, and gave up attempting to push themselves into occupations in which
they must needs fail. The town invaded the country; but the invaders,
like the warlike invaders of early days, yielded to the influence of
their surroundings, and became country people; and in their turn, as
they became more numerous than the townsmen, influenced them also; so
that the difference between town and country grew less and less; and it
was indeed this world of the country vivified by the thought and
briskness of town-bred folk which has produced that happy and leisurely
but eager life of which you have had a first taste. Again I say, many
blunders were made, but we have had time to set them right. Much was
left for the men of my earlier life to deal with. The crude ideas of the
first half of the twentieth century, when men were still oppressed by
the fear of poverty, and did not look enough to the present pleasure of
ordinary daily life, spoilt a great deal of what the commercial age had
left us of external beauty: and I admit that it was but slowly that men
recovered from the injuries that they inflicted on themselves even after
they became free. But slowly as the recovery came, it \emph{did} come;
and the more you see of us, the clearer it will be to you that we are
happy. That we live amidst beauty without any fear of becoming
effeminate; that we have plenty to do, and on the whole enjoy doing it.
What more can we ask of life?"

He paused, as if he were seeking for words with which to express his
thought. Then he said:

"This is how we stand. England was once a country of clearings amongst
the woods and wastes, with a few towns interspersed, which were
fortresses for the feudal army, markets for the folk, gathering places
for the craftsmen. It then became a country of huge and foul workshops
and fouler gambling-dens, surrounded by an ill-kept, poverty-stricken
farm, pillaged by the masters of the workshops. It is now a garden,
where nothing is wasted and nothing is spoilt, with the necessary
dwellings, sheds, and workshops scattered up and down the country, all
trim and neat and pretty. For, indeed, we should be too much ashamed of
ourselves if we allowed the making of goods, even on a large scale, to
carry with it the appearance, even, of desolation and misery. Why, my
friend, those housewives we were talking of just now would teach us
better than that."

Said I: "This side of your change is certainly for the better. But
though I shall soon see some of these villages, tell me in a word or two
what they are like, just to prepare me."

"Perhaps," said he, "you have seen a tolerable picture of these villages
as they were before the end of the nineteenth century. Such things
exist."

"I have seen several of such pictures," said I.

"Well," said Hammond, "our villages are something like the best of such
places, with the church or mote-house of the neighbours for their chief
building. Only note that there are no tokens of poverty about them: no
tumble-down picturesque; which, to tell you the truth, the artist
usually availed himself of to veil his incapacity for drawing
architecture. Such things do not please us, even when they indicate no
misery. Like the mediaevals, we like everything trim and clean, and
orderly and bright; as people always do when they have any sense of
architectural power; because then they know that they can have what they
want, and they won't stand any nonsense from Nature in their dealings
with her."

"Besides the villages, are there any scattered country houses?" said I.

"Yes, plenty," said Hammond; "in fact, except in the wastes and forests
and amongst the sand-hills (like Hindhead in Surrey), it is not easy to
be out of sight of a house; and where the houses are thinly scattered
they run large, and are more like the old colleges than ordinary houses
as they used to be. That is done for the sake of society, for a good
many people can dwell in such houses, as the country dwellers are not
necessarily husbandmen; though they almost all help in such work at
times. The life that goes on in these big dwellings in the country is
very pleasant, especially as some of the most studious men of our time
live in them, and altogether there is a great variety of mind and mood
to be found in them which brightens and quickens the society there."

"I am rather surprised," said I, "by all this, for it seems to me that
after all the country must be tolerably populous."

"Certainly," said he; "the population is pretty much the same as it was
at the end of the nineteenth century; we have spread it, that is all. Of
course, also, we have helped to populate other countries---where we were
wanted and were called for."

Said I: "One thing, it seems to me, does not go with your word of
'garden' for the country. You have spoken of wastes and forests, and I
myself have seen the beginning of your Middlesex and Essex forest. Why
do you keep such things in a garden? and isn't it very wasteful to do
so?"

"My friend," he said, "we like these pieces of wild nature, and can
afford them, so we have them; let alone that as to the forests, we need
a great deal of timber, and suppose that our sons and sons' sons will do
the like. As to the land being a garden, I have heard that they used to
have shrubberies and rockeries in gardens once; and though I might not
like the artificial ones, I assure you that some of the natural
rockeries of our garden are worth seeing. Go north this summer and look
at the Cumberland and Westmoreland ones,---where, by the way, you will
see some sheep-feeding, so that they are not so wasteful as you think;
not so wasteful as forcing-grounds for fruit out of season, \emph{I}
think. Go and have a look at the sheep-walks high up the slopes between
Ingleborough and Pen-y-gwent, and tell me if you think we \emph{waste}
the land there by not covering it with factories for making things that
nobody wants, which was the chief business of the nineteenth century."

"I will try to go there," said I.

"It won't take much trying," said he.
