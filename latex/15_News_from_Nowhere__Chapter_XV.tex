"Yes," said I. "I was expecting Dick and Clara to make their appearance
any moment: but is there time to ask just one or two questions before
they come?"

"Try it, dear neighbour---try it," said old Hammond. "For the more you
ask me the better I am pleased; and at any rate if they do come and find
me in the middle of an answer, they must sit quiet and pretend to listen
till I come to an end. It won't hurt them; they will find it quite
amusing enough to sit side by side, conscious of their proximity to each
other."

I smiled, as I was bound to, and said: "Good; I will go on talking
without noticing them when they come in. Now, this is what I want to ask
you about---to wit, how you get people to work when there is no reward
of labour, and especially how you get them to work strenuously?"

"No reward of labour?" said Hammond, gravely. "The reward of labour is
\emph{life}. Is that not enough?"

"But no reward for especially good work," quoth I.

"Plenty of reward," said he---"the reward of creation. The wages which
God gets, as people might have said time agone. If you are going to ask
to be paid for the pleasure of creation, which is what excellence in
work means, the next thing we shall hear of will be a bill sent in for
the begetting of children."

"Well, but," said I, "the man of the nineteenth century would say there
is a natural desire towards the procreation of children, and a natural
desire not to work."

"Yes, yes," said he, "I know the ancient platitude,---wholly untrue;
indeed, to us quite meaningless. Fourier, whom all men laughed at,
understood the matter better."

"Why is it meaningless to you?" said I.

He said: "Because it implies that all work is suffering, and we are so
far from thinking that, that, as you may have noticed, whereas we are
not short of wealth, there is a kind of fear growing up amongst us that
we shall one day be short of work. It is a pleasure which we are afraid
of losing, not a pain."

"Yes," said I, "I have noticed that, and I was going to ask you about
that also. But in the meantime, what do you positively mean to assert
about the pleasurableness of work amongst you?"

"This, that \emph{all} work is now pleasurable; either because of the
hope of gain in honour and wealth with which the work is done, which
causes pleasurable excitement, even when the actual work is not
pleasant; or else because it has grown into a pleasurable \emph{habit},
as in the case with what you may call mechanical work; and lastly (and
most of our work is of this kind) because there is conscious sensuous
pleasure in the work itself; it is done, that is, by artists."

"I see," said I. "Can you now tell me how you have come to this happy
condition? For, to speak plainly, this change from the conditions of the
older world seems to me far greater and more important than all the
other changes you have told me about as to crime, politics, property,
marriage."

"You are right there," said he. "Indeed, you may say rather that it is
this change which makes all the others possible. What is the object of
Revolution? Surely to make people happy. Revolution having brought its
foredoomed change about, how can you prevent the counter-revolution from
setting in except by making people happy? What! shall we expect peace
and stability from unhappiness? The gathering of grapes from thorns and
figs from thistles is a reasonable expectation compared with that! And
happiness without happy daily work is impossible."

"Most obviously true," said I: for I thought the old boy was preaching a
little. "But answer my question, as to how you gained this happiness."

"Briefly," said he, "by the absence of artificial coercion, and the
freedom for every man to do what he can do best, joined to the knowledge
of what productions of labour we really wanted. I must admit that this
knowledge we reached slowly and painfully."

"Go on," said I, "give me more detail; explain more fully. For this
subject interests me intensely."

"Yes, I will," said he; "but in order to do so I must weary you by
talking a little about the past. Contrast is necessary for this
explanation. Do you mind?"

"No, no," said I.

Said he, settling himself in his chair again for a long talk: "It is
clear from all that we hear and read, that in the last age of
civilisation men had got into a vicious circle in the matter of
production of wares. They had reached a wonderful facility of
production, and in order to make the most of that facility they had
gradually created (or allowed to grow, rather) a most elaborate system
of buying and selling, which has been called the World-Market; and that
World-Market, once set a-going, forced them to go on making more and
more of these wares, whether they needed them or not. So that while (of
course) they could not free themselves from the toil of making real
necessaries, they created in a never-ending series sham or artificial
necessaries, which became, under the iron rule of the aforesaid
World-Market, of equal importance to them with the real necessaries
which supported life. By all this they burdened themselves with a
prodigious mass of work merely for the sake of keeping their wretched
system going."

"Yes---and then?" said I.

"Why, then, since they had forced themselves to stagger along under this
horrible burden of unnecessary production, it became impossible for them
to look upon labour and its results from any other point of view than
one---to wit, the ceaseless endeavour to expend the least possible
amount of labour on any article made, and yet at the same time to make
as many articles as possible. To this 'cheapening of production', as it
was called, everything was sacrificed: the happiness of the workman at
his work, nay, his most elementary comfort and bare health, his food,
his clothes, his dwelling, his leisure, his amusement, his
education---his life, in short---did not weigh a grain of sand in the
balance against this dire necessity of 'cheap production' of things, a
great part of which were not worth producing at all. Nay, we are told,
and we must believe it, so overwhelming is the evidence, though many of
our people scarcely \emph{can} believe it, that even rich and powerful
men, the masters of the poor devils aforesaid, submitted to live amidst
sights and sounds and smells which it is in the very nature of man to
abhor and flee from, in order that their riches might bolster up this
supreme folly. The whole community, in fact, was cast into the jaws of
this ravening monster, 'the cheap production' forced upon it by the
World-Market."

"Dear me!" said I. "But what happened? Did not their cleverness and
facility in production master this chaos of misery at last? Couldn't
they catch up with the World-Market, and then set to work to devise
means for relieving themselves from this fearful task of extra labour?"

He smiled bitterly. "Did they even try to?" said he. "I am not sure. You
know that according to the old saw the beetle gets used to living in
dung; and these people, whether they found the dung sweet or not,
certainly lived in it."

His estimate of the life of the nineteenth century made me catch my
breath a little; and I said feebly, "But the labour-saving machines?"

"Heyday!" quoth he. "What's that you are saying? the labour-saving
machines? Yes, they were made to 'save labour' (or, to speak more
plainly, the lives of men) on one piece of work in order that it might
be expended---I will say wasted---on another, probably useless, piece of
work. Friend, all their devices for cheapening labour simply resulted in
increasing the burden of labour. The appetite of the World-Market grew
with what it fed on: the countries within the ring of 'civilisation'
(that is, organised misery) were glutted with the abortions of the
market, and force and fraud were used unsparingly to 'open up' countries
\emph{outside} that pale. This process of 'opening up' is a strange one
to those who have read the professions of the men of that period and do
not understand their practice; and perhaps shows us at its worst the
great vice of the nineteenth century, the use of hypocrisy and cant to
evade the responsibility of vicarious ferocity. When the civilised
World-Market coveted a country not yet in its clutches, some transparent
pretext was found---the suppression of a slavery different from and not
so cruel as that of commerce; the pushing of a religion no longer
believed in by its promoters; the 'rescue' of some desperado or
homicidal madman whose misdeeds had got him into trouble amongst the
natives of the 'barbarous' country---any stick, in short, which would
beat the dog at all. Then some bold, unprincipled, ignorant adventurer
was found (no difficult task in the days of competition), and he was
bribed to 'create a market' by breaking up whatever traditional society
there might be in the doomed country, and by destroying whatever leisure
or pleasure he found there. He forced wares on the natives which they
did not want, and took their natural products in 'exchange,' as this
form of robbery was called, and thereby he 'created new wants,' to
supply which (that is, to be allowed to live by their new masters) the
hapless, helpless people had to sell themselves into the slavery of
hopeless toil so that they might have something wherewith to purchase
the nullities of 'civilisation.' Ah," said the old man, pointing the
dealings of to the Museum, "I have read books and papers in there,
telling strange stories indeed of civilisation (or organised misery)
with 'non-civilisation'; from the time when the British Government
deliberately sent blankets infected with small-pox as choice gifts to
inconvenient tribes of Red-skins, to the time when Africa was infested
by a man named Stanley, who---"

"Excuse me," said I, "but as you know, time presses; and I want to keep
our question on the straightest line possible; and I want at once to ask
this about these wares made for the World-Market---how about their
quality; these people who were so clever about making goods, I suppose
they made them well?"

"Quality!" said the old man crustily, for he was rather peevish at being
cut short in his story; "how could they possibly attend to such trifles
as the quality of the wares they sold? The best of them were of a lowish
average, the worst were transparent make-shifts for the things asked
for, which nobody would have put up with if they could have got anything
else. It was a current jest of the time that the wares were made to sell
and not to use; a jest which you, as coming from another planet, may
understand, but which our folk could not."

Said I: "What! did they make nothing well?"

"Why, yes," said he, "there was one class of goods which they did make
thoroughly well, and that was the class of machines which were used for
making things. These were usually quite perfect pieces of workmanship,
admirably adapted to the end in view. So that it may be fairly said that
the great achievement of the nineteenth century was the making of
machines which were wonders of invention, skill, and patience, and which
were used for the production of measureless quantities of worthless
make-shifts. In truth, the owners of the machines did not consider
anything which they made as wares, but simply as means for the
enrichment of themselves. Of course the only admitted test of utility in
wares was the finding of buyers for them---wise men or fools, as it
might chance."

"And people put up with this?" said I.

"For a time," said he.

"And then?"

"And then the overturn," said the old man, smiling, "and the nineteenth
century saw itself as a man who has lost his clothes whilst bathing, and
has to walk naked through the town."

"You are very bitter about that unlucky nineteenth century," said I.

"Naturally," said he, "since I know so much about it."

He was silent a little, and then said: "There are traditions---nay, real
histories---in our family about it: my grandfather was one of its
victims. If you know something about it, you will understand what he
suffered when I tell you that he was in those days a genuine artist, a
man of genius, and a revolutionist."

"I think I do understand," said I: "but now, as it seems, you have
reversed all this?"

"Pretty much so," said he. "The wares which we make are made because
they are needed: men make for their neighbours' use as if they were
making for themselves, not for a vague market of which they know
nothing, and over which they have no control: as there is no buying and
selling, it would be mere insanity to make goods on the chance of their
being wanted; for there is no longer anyone who can be compelled to buy
them. So that whatever is made is good, and thoroughly fit for its
purpose. Nothing can be made except for genuine use; therefore no
inferior goods are made. Moreover, as aforesaid, we have now found out
what we want, so we make no more than we want; and as we are not driven
to make a vast quantity of useless things we have time and resources
enough to consider our pleasure in making them. All work which would be
irksome to do by hand is done by immensely improved machinery; and in
all work which it is a pleasure to do by hand machinery is done without.
There is no difficulty in finding work which suits the special turn of
mind of everybody; so that no man is sacrificed to the wants of another.
From time to time, when we have found out that some piece of work was
too disagreeable or troublesome, we have given it up and done altogether
without the thing produced by it. Now, surely you can see that under
these circumstances all the work that we do is an exercise of the mind
and body more or less pleasant to be done: so that instead of avoiding
work everybody seeks it: and, since people have got defter in doing the
work generation after generation, it has become so easy to do, that it
seems as if there were less done, though probably more is produced. I
suppose this explains that fear, which I hinted at just now, of a
possible scarcity in work, which perhaps you have already noticed, and
which is a feeling on the increase, and has been for a score of years."

"But do you think," said I, "that there is any fear of a work-famine
amongst you?"

"No, I do not," said he, "and I will tell why; it is each man's business
to make his own work pleasanter and pleasanter, which of course tends
towards raising the standard of excellence, as no man enjoys turning out
work which is not a credit to him, and also to greater deliberation in
turning it out; and there is such a vast number of things which can be
treated as works of art, that this alone gives employment to a host of
deft people. Again, if art be inexhaustible, so is science also; and
though it is no longer the only innocent occupation which is thought
worth an intelligent man spending his time upon, as it once was, yet
there are, and I suppose will be, many people who are excited by its
conquest of difficulties, and care for it more than for anything else.
Again, as more and more of pleasure is imported into work, I think we
shall take up kinds of work which produce desirable wares, but which we
gave up because we could not carry them on pleasantly. Moreover, I think
that it is only in parts of Europe which are more advanced than the rest
of the world that you will hear this talk of the fear of a work-famine.
Those lands which were once the colonies of Great Britain, for instance,
and especially America---that part of it, above all, which was once the
United states---are now and will be for a long while a great resource to
us. For these lands, and, I say, especially the northern parts of
America, suffered so terribly from the full force of the last days of
civilisation, and became such horrible places to live in, that they are
now very backward in all that makes life pleasant. Indeed, one may say
that for nearly a hundred years the people of the northern parts of
America have been engaged in gradually making a dwelling-place out of a
stinking dust-heap; and there is still a great deal to do, especially as
the country is so big."

"Well," said I, "I am exceedingly glad to think that you have such a
prospect of happiness before you. But I should like to ask a few more
questions, and then I have done for to-day."
