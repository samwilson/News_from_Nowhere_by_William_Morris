We now turned into a pleasant lane where the branches of great
plane-trees nearly met overhead, but behind them lay low houses standing
rather close together.

"This is Long Acre," quoth Dick; "so there must once have been a
cornfield here. How curious it is that places change so, and yet keep
their old names! Just look how thick the houses stand! and they are
still going on building, look you!"

"Yes," said the old man, "but I think the cornfields must have been
built over before the middle of the nineteenth century. I have heard
that about here was one of the thickest parts of the town. But I must
get down here, neighbours; I have got to call on a friend who lives in
the gardens behind this Long Acre. Good-bye and good luck, Guest!"

And he jumped down and strode away vigorously, like a young man.

"How old should you say that neighbour will be?" said I to Dick as we
lost sight of him; for I saw that he was old, and yet he looked dry and
sturdy like a piece of old oak; a type of old man I was not used to
seeing.

"O, about ninety, I should say," said Dick.

"How long-lived your people must be!" said I.

"Yes," said Dick, "certainly we have beaten the threescore-and-ten of
the old Jewish proverb-book. But then you see that was written of Syria,
a hot dry country, where people live faster than in our temperate
climate. However, I don't think it matters much, so long as a man is
healthy and happy while he \emph{is} alive. But now, Guest, we are so
near to my old kinsman's dwelling-place that I think you had better keep
all future questions for him."

I nodded a yes; and therewith we turned to the left, and went down a
gentle slope through some beautiful rose-gardens, laid out on what I
took to be the site of Endell Street. We passed on, and Dick drew rein
an instant as we came across a long straightish road with houses
scantily scattered up and down it. He waved his hand right and left, and
said, "Holborn that side, Oxford Road that. This was once a very
important part of the crowded city outside the ancient walls of the
Roman and Mediaeval burg: many of the feudal nobles of the Middle Ages,
we are told, had big houses on either side of Holborn. I daresay you
remember that the Bishop of Ely's house is mentioned in Shakespeare's
play of King Richard III.; and there are some remains of that still
left. However, this road is not of the same importance, now that the
ancient city is gone, walls and all."

He drove on again, while I smiled faintly to think how the nineteenth
century, of which such big words have been said, counted for nothing in
the memory of this man, who read Shakespeare and had not forgotten the
Middle Ages.

We crossed the road into a short narrow lane between the gardens, and
came out again into a wide road, on one side of which was a great and
long building, turning its gables away from the highway, which I saw at
once was another public group. Opposite to it was a wide space of
greenery, without any wall or fence of any kind. I looked through the
trees and saw beyond them a pillared portico quite familiar to me---no
less old a friend, in fact, than the British Museum. It rather took my
breath away, amidst all the strange things I had seen; but I held my
tongue and let Dick speak. Said he:

"Yonder is the British Museum, where my great-grandfather mostly lives;
so I won't say much about it. The building on the left is the Museum
Market, and I think we had better turn in there for a minute or two; for
Greylocks will be wanting his rest and his oats; and I suppose you will
stay with my kinsman the greater part of the day; and to say the truth,
there may be some one there whom I particularly want to see, and perhaps
have a long talk with."

He blushed and sighed, not altogether with pleasure, I thought; so of
course I said nothing, and he turned the horse under an archway which
brought us into a very large paved quadrangle, with a big sycamore tree
in each corner and a plashing fountain in the midst. Near the fountain
were a few market stalls, with awnings over them of gay striped linen
cloth, about which some people, mostly women and children, were moving
quietly, looking at the goods exposed there. The ground floor of the
building round the quadrangle was occupied by a wide arcade or cloister,
whose fanciful but strong architecture I could not enough admire. Here
also a few people were sauntering or sitting reading on the benches.

Dick said to me apologetically: "Here as elsewhere there is little doing
to-day; on a Friday you would see it thronged, and gay with people, and
in the afternoon there is generally music about the fountain. However, I
daresay we shall have a pretty good gathering at our mid-day meal."

We drove through the quadrangle and by an archway, into a large handsome
stable on the other side, where we speedily stalled the old nag and made
him happy with horse-meat, and then turned and walked back again through
the market, Dick looking rather thoughtful, as it seemed to me.

I noticed that people couldn't help looking at me rather hard, and
considering my clothes and theirs, I didn't wonder; but whenever they
caught my eye they made me a very friendly sign of greeting.

We walked straight into the forecourt of the Museum, where, except that
the railings were gone, and the whispering boughs of the trees were all
about, nothing seemed changed; the very pigeons were wheeling about the
building and clinging to the ornaments of the pediment as I had seen
them of old.

Dick seemed grown a little absent, but he could not forbear giving me an
architectural note, and said:

"It is rather an ugly old building, isn't it? Many people have wanted to
pull it down and rebuild it: and perhaps if work does really get scarce
we may yet do so. But, as my great grandfather will tell you, it would
not be quite a straightforward job; for there are wonderful collections
in there of all kinds of antiquities, besides an enormous library with
many exceedingly beautiful books in it, and many most useful ones as
genuine records, texts of ancient works and the like; and the worry and
anxiety, and even risk, there would be in moving all this has saved the
buildings themselves. Besides, as we said before, it is not a bad thing
to have some record of what our forefathers thought a handsome building.
For there is plenty of labour and material in it."

"I see there is," said I, "and I quite agree with you. But now hadn't we
better make haste to see your great-grandfather?"

In fact, I could not help seeing that he was rather dallying with the
time. He said, "Yes, we will go into the house in a minute. My kinsman
is too old to do much work in the Museum, where he was a custodian of
the books for many years; but he still lives here a good deal; indeed I
think," said he, smiling, "that he looks upon himself as a part of the
books, or the books a part of him, I don't know which."

He hesitated a little longer, then flushing up, took my hand, and
saying, "Come along, then!" led me toward the door of one of the old
official dwellings.
