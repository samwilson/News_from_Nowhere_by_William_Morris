\chapter{Discussion and bed}

Up at the League, says a friend, there had been one night a brisk
conversational discussion, as to what would happen on the Morrow of the
Revolution, finally shading off into a vigorous statement by various
friends of their views on the future of the fully-developed new society.

Says our friend: Considering the subject, the discussion was
good-tempered; for those present being used to public meetings and
after-lecture debates, if they did not listen to each others' opinions
(which could scarcely be expected of them), at all events did not always
attempt to speak all together, as is the custom of people in ordinary
polite society when conversing on a subject which interests them. For
the rest, there were six persons present, and consequently six sections
of the party were represented, four of which had strong but divergent
Anarchist opinions. One of the sections, says our friend, a man whom he
knows very well indeed, sat almost silent at the beginning of the
discussion, but at last got drawn into it, and finished by roaring out
very loud, and damning all the rest for fools; after which befel a
period of noise, and then a lull, during which the aforesaid section,
having said good-night very amicably, took his way home by himself to a
western suburb, using the means of travelling which civilisation has
forced upon us like a habit. As he sat in that vapour-bath of hurried
and discontented humanity, a carriage of the underground railway, he,
like others, stewed discontentedly, while in self-reproachful mood he
turned over the many excellent and conclusive arguments which, though
they lay at his fingers' ends, he had forgotten in the just past
discussion. But this frame of mind he was so used to, that it didn't
last him long, and after a brief discomfort, caused by disgust with
himself for having lost his temper (which he was also well used to), he
found himself musing on the subject-matter of discussion, but still
discontentedly and unhappily. "If I could but see a day of it," he said
to himself; "if I could but see it!"

As he formed the words, the train stopped at his station, five minutes'
walk from his own house, which stood on the banks of the Thames, a
little way above an ugly suspension bridge. He went out of the station,
still discontented and unhappy, muttering "If I could but see it! if I
could but see it!" but had not gone many steps towards the river before
(says our friend who tells the story) all that discontent and trouble
seemed to slip off him.

It was a beautiful night of early winter, the air just sharp enough to
be refreshing after the hot room and the stinking railway carriage. The
wind, which had lately turned a point or two north of west, had blown
the sky clear of all cloud save a light fleck or two which went swiftly
down the heavens. There was a young moon halfway up the sky, and as the
home-farer caught sight of it, tangled in the branches of a tall old
elm, he could scarce bring to his mind the shabby London suburb where he
was, and he felt as if he were in a pleasant country place---pleasanter,
indeed, than the deep country was as he had known it.

He came right down to the river-side, and lingered a little, looking
over the low wall to note the moonlit river, near upon high water, go
swirling and glittering up to Chiswick Eyot: as for the ugly bridge
below, he did not notice it or think of it, except when for a moment
(says our friend) it struck him that he missed the row of lights down
stream. Then he turned to his house door and let himself in; and even as
he shut the door to, disappeared all remembrance of that brilliant logic
and foresight which had so illuminated the recent discussion; and of the
discussion itself there remained no trace, save a vague hope, that was
now become a pleasure, for days of peace and rest, and cleanness and
smiling goodwill.

In this mood he tumbled into bed, and fell asleep after his wont, in two
minutes' time; but (contrary to his wont) woke up again not long after
in that curiously wide-awake condition which sometimes surprises even
good sleepers; a condition under which we feel all our wits
preternaturally sharpened, while all the miserable muddles we have ever
got into, all the disgraces and losses of our lives, will insist on
thrusting themselves forward for the consideration of those sharpened
wits.

In this state he lay (says our friend) till he had almost begun to enjoy
it: till the tale of his stupidities amused him, and the entanglements
before him, which he saw so clearly, began to shape themselves into an
amusing story for him.

He heard one o'clock strike, then two and then three; after which he
fell asleep again. Our friend says that from that sleep he awoke once
more, and afterwards went through such surprising adventures that he
thinks that they should be told to our comrades, and indeed the public
in general, and therefore proposes to tell them now. But, says he, I
think it would be better if I told them in the first person, as if it
were myself who had gone through them; which, indeed, will be the easier
and more natural to me, since I understand the feelings and desires of
the comrade of whom I am telling better than any one else in the world
does.
