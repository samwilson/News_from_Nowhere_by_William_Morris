Presently at a place where the river flowed round a headland of the
meadows, we stopped a while for rest and victuals, and settled ourselves
on a beautiful bank which almost reached the dignity of a hill-side: the
wide meadows spread before us, and already the scythe was busy amidst
the hay. One change I noticed amidst the quiet beauty of the fields---to
wit, that they were planted with trees here and there, often
fruit-trees, and that there was none of the niggardly begrudging of
space to a handsome tree which I remembered too well; and though the
willows were often polled (or shrowded, as they call it in that
country-side), this was done with some regard to beauty: I mean that
there was no polling of rows on rows so as to destroy the pleasantness
of half a mile of country, but a thoughtful sequence in the cutting,
that prevented a sudden bareness anywhere. To be short, the fields were
everywhere treated as a garden made for the pleasure as well as the
livelihood of all, as old Hammond told me was the case.

On this bank or bent of the hill, then, we had our mid-day meal;
somewhat early for dinner, if that mattered, but we had been stirring
early: the slender stream of the Thames winding below us between the
garden of a country I have been telling of; a furlong from us was a
beautiful little islet begrown with graceful trees; on the slopes
westward of us was a wood of varied growth overhanging the narrow meadow
on the south side of the river; while to the north was a wide stretch of
mead rising very gradually from the river's edge. A delicate spire of an
ancient building rose up from out of the trees in the middle distance,
with a few grey houses clustered about it; while nearer to us, in fact
not half a furlong from the water, was a quite modern stone house---a
wide quadrangle of one story, the buildings that made it being quite
low. There was no garden between it and the river, nothing but a row of
pear-trees still quite young and slender; and though there did not seem
to be much ornament about it, it had a sort of natural elegance, like
that of the trees themselves.

As we sat looking down on all this in the sweet June day, rather happy
than merry, Ellen, who sat next me, her hand clasped about one knee,
leaned sideways to me, and said in a low voice which Dick and Clara
might have noted if they had not been busy in happy wordless
love-making: "Friend, in your country were the houses of your
field-labourers anything like that?"

I said: "Well, at any rate the houses of our rich men were not; they
were mere blots upon the face of the land."

"I find that hard to understand," she said. "I can see why the workmen,
who were so oppressed, should not have been able to live in beautiful
houses; for it takes time and leisure, and minds not over-burdened with
care, to make beautiful dwellings; and I quite understand that these
poor people were not allowed to live in such a way as to have these (to
us) necessary good things. But why the rich men, who had the time and
the leisure and the materials for building, as it would be in this case,
should not have housed themselves well, I do not understand as yet. I
know what you are meaning to say to me," she said, looking me full in
the eyes and blushing, "to wit that their houses and all belonging to
them were generally ugly and base, unless they chanced to be ancient
like yonder remnant of our forefathers' work" (pointing to the spire);
"that they were---let me see; what is the word?"

"Vulgar," said I. "We used to say," said I, "that the ugliness and
vulgarity of the rich men's dwellings was a necessary reflection from
the sordidness and bareness of life which they forced upon the poor
people."

She knit her brows as in thought; then turned a brightened face on me,
as if she had caught the idea, and said: "Yes, friend, I see what you
mean. We have sometimes---those of us who look into these
things---talked this very matter over; because, to say the truth, we
have plenty of record of the so-called arts of the time before Equality
of Life; and there are not wanting people who say that the state of that
society was not the cause of all that ugliness; that they were ugly in
their life because they liked to be, and could have had beautiful things
about them if they had chosen; just as a man or body of men now may, if
they please, make things more or less beautiful---Stop! I know what you
are going to say."

"Do you?" said I, smiling, yet with a beating heart.

"Yes," she said; "you are answering me, teaching me, in some way or
another, although you have not spoken the words aloud. You were going to
say that in times of inequality it was an essential condition of the
life of these rich men that they should not themselves make what they
wanted for the adornment of their lives, but should force those to make
them whom they forced to live pinched and sordid lives; and that as a
necessary consequence the sordidness and pinching, the ugly barrenness
of those ruined lives, were worked up into the adornment of the lives of
the rich, and art died out amongst men? Was that what you would say, my
friend?"

"Yes, yes," I said, looking at her eagerly; for she had risen and was
standing on the edge of the bent, the light wind stirring her dainty
raiment, one hand laid on her bosom, the other arm stretched downward
and clenched in her earnestness.

"It is true," she said, "it is true! We have proved it true!"

I think amidst my---something more than interest in her, and admiration
for her, I was beginning to wonder how it would all end. I had a
glimmering of fear of what might follow; of anxiety as to the remedy
which this new age might offer for the missing of something one might
set one's heart on. But now Dick rose to his feet and cried out in his
hearty manner: "Neighbour Ellen, are you quarrelling with the guest, or
are you worrying him to tell you things which he cannot properly explain
to our ignorance?"

"Neither, dear neighbour," she said. "I was so far from quarrelling with
him that I think I have been making him good friends both with himself
and me. Is it so, dear guest?" she said, looking down at me with a
delightful smile of confidence in being understood.

"Indeed it is," said I.

"Well, moreover," she said, "I must say for him that he has explained
himself to me very well indeed, so that I quite understand him."

"All right," quoth Dick. "When I first set eyes on you at Runnymede I
knew that there was something wonderful in your keenness of wits. I
don't say that as a mere pretty speech to please you," said he quickly,
"but because it is true; and it made me want to see more of you. But,
come, we ought to be going; for we are not half way, and we ought to be
in well before sunset."

And therewith he took Clara's hand, and led her down the bent. But Ellen
stood thoughtfully looking down for a little, and as I took her hand to
follow Dick, she turned round to me and said:

"You might tell me a great deal and make many things clear to me, if you
would."

"Yes," said I, "I am pretty well fit for that,---and for nothing
else---an old man like me."

She did not notice the bitterness which, whether I liked it or not, was
in my voice as I spoke, but went on: "It is not so much for myself; I
should be quite content to dream about past times, and if I could not
idealise them, yet at least idealise some of the people who lived in
them. But I think sometimes people are too careless of the history of
the past---too apt to leave it in the hands of old learned men like
Hammond. Who knows? Happy as we are, times may alter; we may be bitten
with some impulse towards change, and many things may seem too wonderful
for us to resist, too exciting not to catch at, if we do not know that
they are but phases of what has been before; and withal ruinous,
deceitful, and sordid."

As we went slowly down toward the boats she said again: "Not for myself
alone, dear friend; I shall have children; perhaps before the end a good
many;---I hope so. And though of course I cannot force any special kind
of knowledge upon them, yet, my Friend, I cannot help thinking that just
as they might be like me in body, so I might impress upon them some part
of my ways of thinking; that is, indeed, some of the essential part of
myself; that part which was not mere moods, created by the matters and
events round about me. What do you think?"

Of one thing I was sure, that her beauty and kindness and eagerness
combined, forced me to think as she did, when she was not earnestly
laying herself open to receive my thoughts. I said, what at the time was
true, that I thought it most important; and presently stood entranced by
the wonder of her grace as she stepped into the light boat, and held out
her hand to me. And so on we went up the Thames still---or whither?
