As we went down to the boat next morning, Walter could not quite keep
off the subject of last night, though he was more hopeful than he had
been then, and seemed to think that if the unlucky homicide could not be
got to go over-sea, he might at any rate go and live somewhere in the
neighbourhood pretty much by himself; at any rate, that was what he
himself had proposed. To Dick, and I must say to me also, this seemed a
strange remedy; and Dick said as much. Quoth he:

"Friend Walter, don't set the man brooding on the tragedy by letting him
live alone. That will only strengthen his idea that he has committed a
crime, and you will have him killing himself in good earnest."

Said Clara: "I don't know. If I may say what I think of it, it is that
he had better have his fill of gloom now, and, so to say, wake up
presently to see how little need there has been for it; and then he will
live happily afterwards. As for his killing himself, you need not be
afraid of that; for, from all you tell me, he is really very much in
love with the woman; and to speak plainly, until his love is satisfied,
he will not only stick to life as tightly as he can, but will also make
the most of every event of his life---will, so to say, hug himself up in
it; and I think that this is the real explanation of his taking the
whole matter with such an excess of tragedy."

Walter looked thoughtful, and said: "Well, you may be right; and perhaps
we should have treated it all more lightly: but you see, guest" (turning
to me), "such things happen so seldom, that when they do happen, we
cannot help being much taken up with it. For the rest, we are all
inclined, to excuse our poor friend for making us so unhappy, on the
ground that he does it out of an exaggerated respect for human life and
its happiness. Well, I will say no more about it; only this: will you
give me a cast up stream, as I want to look after a lonely habitation
for the poor fellow, since he will have it so, and I hear that there is
one which would suit us very well on the downs beyond Streatley; so if
you will put me ashore there I will walk up the hill and look to it."

"Is the house in question empty?" said I.

"No," said Walter, "but the man who lives there will go out of it, of
course, when he hears that we want it. You see, we think that the fresh
air of the downs and the very emptiness of the landscape will do our
friend good."

"Yes," said Clara, smiling, "and he will not be so far from his beloved
that they cannot easily meet if they have a mind to---as they certainly
will."

This talk had brought us down to the boat, and we were presently afloat
on the beautiful broad stream, Dick driving the prow swiftly through the
windless water of the early summer morning, for it was not yet six
o'clock. We were at the lock in a very little time; and as we lay rising
and rising on the in-coming water, I could not help wondering that my
old friend the pound-lock, and that of the very simplest and most rural
kind, should hold its place there; so I said:

"I have been wondering, as we passed lock after lock, that you people,
so prosperous as you are, and especially since you are so anxious for
pleasant work to do, have not invented something which would get rid of
this clumsy business of going up-stairs by means of these rude
contrivances."

Dick laughed. "My dear friend," said he, "as long as water has the
clumsy habit of running down hill, I fear we must humour it by going
up-stairs when we have our faces turned from the sea. And really I don't
see why you should fall foul of Maple-Durham lock, which I think a very
pretty place."

There was no doubt about the latter assertion, I thought, as I looked up
at the overhanging boughs of the great trees, with the sun coming
glittering through the leaves, and listened to the song of the summer
blackbirds as it mingled with the sound of the backwater near us. So not
being able to say why I wanted the locks away---which, indeed, I didn't
do at all---I held my peace. But Walter said---

"You see, guest, this is not an age of inventions. The last epoch did
all that for us, and we are now content to use such of its inventions as
we find handy, and leaving those alone which we don't want. I believe,
as a matter of fact, that some time ago (I can't give you a date) some
elaborate machinery was used for the locks, though people did not go so
far as try to make the water run up hill. However, it was troublesome, I
suppose, and the simple hatches, and the gates, with a big
counterpoising beam, were found to answer every purpose, and were easily
mended when wanted with material always to hand: so here they are, as
you see."

"Besides," said Dick, "this kind of lock is pretty, as you can see; and
I can't help thinking that your machine-lock, winding up like a watch,
would have been ugly and would have spoiled the look of the river: and
that is surely reason enough for keeping such locks as these. Good-bye,
old fellow!" said he to the lock, as he pushed us out through the now
open gates by a vigorous stroke of the boat-hook. "May you live long,
and have your green old age renewed for ever!"

On we went; and the water had the familiar aspect to me of the days
before Pangbourne had been thoroughly cocknified, as I have seen it. It
(Pangbourne) was distinctly a village still---\emph{i.e.}, a definite
group of houses, and as pretty as might be. The beech-woods still
covered the hill that rose above Basildon; but the flat fields beneath
them were much more populous than I remembered them, as there were five
large houses in sight, very carefully designed so as not to hurt the
character of the country. Down on the green lip of the river, just where
the water turns toward the Goring and Streatley reaches, were half a
dozen girls playing about on the grass. They hailed us as we were about
passing them, as they noted that we were travellers, and we stopped a
minute to talk with them. They had been bathing, and were light clad and
bare-footed, and were bound for the meadows on the Berkshire side, where
the haymaking had begun, and were passing the time merrily enough till
the Berkshire folk came in their punt to fetch them. At first nothing
would content them but we must go with them into the hay-field, and
breakfast with them; but Dick put forward his theory of beginning the
hay-harvest higher up the water, and not spoiling my pleasure therein by
giving me a taste of it elsewhere, and they gave way, though
unwillingly. In revenge they asked me a great many questions about the
country I came from and the manners of life there, which I found rather
puzzling to answer; and doubtless what answers I did give were puzzling
enough to them. I noticed both with these pretty girls and with
everybody else we met, that in default of serious news, such as we had
heard at Maple-Durham, they were eager to discuss all the little details
of life: the weather, the hay-crop, the last new house, the plenty or
lack of such and such birds, and so on; and they talked of these things
not in a fatuous and conventional way, but as taking, I say, real
interest in them. Moreover, I found that the women knew as much about
all these things as the men: could name a flower, and knew its
qualities; could tell you the habitat of such and such birds and fish,
and the like.

It is almost strange what a difference this intelligence made in my
estimate of the country life of that day; for it used to be said in past
times, and on the whole truly, that outside their daily work country
people knew little of the country, and at least could tell you nothing
about it; while here were these people as eager about all the goings on
in the fields and woods and downs as if they had been Cockneys newly
escaped from the tyranny of bricks and mortar.

I may mention as a detail worth noticing that not only did there seem to
be a great many more birds about of the non-predatory kinds, but their
enemies the birds of prey were also commoner. A kite hung over our heads
as we passed Medmenham yesterday; magpies were quite common in the
hedgerows; I saw several sparrow-hawks, and I think a merlin; and now
just as we were passing the pretty bridge which had taken the place of
Basildon railway-bridge, a couple of ravens croaked above our boat, as
they sailed off to the higher ground of the downs. I concluded from all
this that the days of the gamekeeper were over, and did not even need to
ask Dick a question about it.
