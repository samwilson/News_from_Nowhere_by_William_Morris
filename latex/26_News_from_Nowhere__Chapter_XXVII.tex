We set Walter ashore on the Berkshire side, amidst all the beauties of
Streatley, and so went our ways into what once would have been the
deeper country under the foot-hills of the White Horse; and though the
contrast between half-cocknified and wholly unsophisticated country
existed no longer, a feeling of exultation rose within me (as it used to
do) at sight of the familiar and still unchanged hills of the Berkshire
range.

We stopped at Wallingford for our mid-day meal; of course, all signs of
squalor and poverty had disappeared from the streets of the ancient
town, and many ugly houses had been taken down and many pretty new ones
built, but I thought it curious, that the town still looked like the old
place I remembered so well; for indeed it looked like that ought to have
looked.

At dinner we fell in with an old, but very bright and intelligent man,
who seemed in a country way to be another edition of old Hammond. He had
an extraordinary detailed knowledge of the ancient history of the
country-side from the time of Alfred to the days of the Parliamentary
Wars, many events of which, as you may know, were enacted round about
Wallingford. But, what was more interesting to us, he had detailed
record of the period of the change to the present state of things, and
told us a great deal about it, and especially of that exodus of the
people from the town to the country, and the gradual recovery by the
town-bred people on one side, and the country-bred people on the other,
of those arts of life which they had each lost; which loss, as he told
us, had at one time gone so far that not only was it impossible to find
a carpenter or a smith in a village or small country town, but that
people in such places had even forgotten how to bake bread, and that at
Wallingford, for instance, the bread came down with the newspapers by an
early train from London, worked in some way, the explanation of which I
could not understand. He told us also that the townspeople who came into
the country used to pick up the agricultural arts by carefully watching
the way in which the machines worked, gathering an idea of handicraft
from machinery; because at that time almost everything in and about the
fields was done by elaborate machines used quite unintelligently by the
labourers. On the other hand, the old men amongst the labourers managed
to teach the younger ones gradually a little artizanship, such as the
use of the saw and the plane, the work of the smithy, and so forth; for
once more, by that time it was as much as---or rather, more than---a man
could do to fix an ash pole to a rake by handiwork; so that it would
take a machine worth a thousand pounds, a group of workmen, and half a
day's travelling, to do five shillings' worth of work. He showed us,
among other things, an account of a certain village council who were
working hard at all this business; and the record of their intense
earnestness in getting to the bottom of some matter which in time past
would have been thought quite trivial, as, for example, the due
proportions of alkali and oil for soap-making for the village wash, or
the exact heat of the water into which a leg of mutton should be plunged
for boiling---all this joined to the utter absence of anything like
party feeling, which even in a village assembly would certainly have
made its appearance in an earlier epoch, was very amusing, and at the
same time instructive.

This old man, whose name was Henry Morsom, took us, after our meal and a
rest, into a biggish hall which contained a large collection of articles
of manufacture and art from the last days of the machine period to that
day; and he went over them with us, and explained them with great care.
They also were very interesting, showing the transition from the
makeshift work of the machines (which was at about its worst a little
after the Civil War before told of) into the first years of the new
handicraft period. Of course, there was much overlapping of the periods:
and at first the new handwork came in very slowly.

"You must remember," said the old antiquary, "that the handicraft was
not the result of what used to be called material necessity: on the
contrary, by that time the machines had been so much improved that
almost all necessary work might have been done by them: and indeed many
people at that time, and before it, used to think that machinery would
entirely supersede handicraft; which certainly, on the face of it,
seemed more than likely. But there was another opinion, far less
logical, prevalent amongst the rich people before the days of freedom,
which did not die out at once after that epoch had begun. This opinion,
which from all I can learn seemed as natural then, as it seems absurd
now, was, that while the ordinary daily work of the world would be done
entirely by automatic machinery, the energies of the more intelligent
part of mankind would be set free to follow the higher forms of the
arts, as well as science and the study of history. It was strange, was
it not, that they should thus ignore that aspiration after complete
equality which we now recognise as the bond of all happy human society?"

I did not answer, but thought the more. Dick looked thoughtful, and
said:

"Strange, neighbour? Well, I don't know. I have often heard my old
kinsman say the one aim of all people before our time was to avoid work,
or at least they thought it was; so of course the work which their daily
life forced them to do, seemed more like work than that which they
seemed to choose for themselves."

"True enough," said Morsom. "Anyhow, they soon began to find out their
mistake, and that only slaves and slave-holders could live solely by
setting machines going."

Clara broke in here, flushing a little as she spoke: "Was not their
mistake once more bred of the life of slavery that they had been
living?---a life which was always looking upon everything, except
mankind, animate and inanimate---'nature,' as people used to call
it---as one thing, and mankind as another, it was natural to people
thinking in this way, that they should try to make 'nature' their slave,
since they thought 'nature' was something outside them."

"Surely," said Morsom; "and they were puzzled as to what to do, till
they found the feeling against a mechanical life, which had begun before
the Great Change amongst people who had leisure to think of such things,
was spreading insensibly; till at last under the guise of pleasure that
was not supposed to be work, work that was pleasure began to push out
the mechanical toil, which they had once hoped at the best to reduce to
narrow limits indeed, but never to get rid of; and which, moreover, they
found they could not limit as they had hoped to do."

"When did this new revolution gather head?" said I.

"In the half-century that followed the Great Change," said Morsom, "it
began to be noteworthy; machine after machine was quietly dropped under
the excuse that the machines could not produce works of art, and that
works of art were more and more called for. Look here," he said, "here
are some of the works of that time---rough and unskilful in handiwork,
but solid and showing some sense of pleasure in the making."

"They are very curious," said I, taking up a piece of pottery from
amongst the specimens which the antiquary was showing us; "not a bit
like the work of either savages or barbarians, and yet with what would
once have been called a hatred of civilisation impressed upon them."

"Yes," said Morsom, "you must not look for delicacy there: in that
period you could only have got that from a man who was practically a
slave. But now, you see," said he, leading me on a little, "we have
learned the trick of handicraft, and have added the utmost refinement of
workmanship to the freedom of fancy and imagination."

I looked, and wondered indeed at the deftness and abundance of beauty of
the work of men who had at last learned to accept life itself as a
pleasure, and the satisfaction of the common needs of mankind and the
preparation for them, as work fit for the best of the race. I mused
silently; but at last I said---

"What is to come after this?"

The old man laughed. "I don't know," said he; "we will meet it when it
comes."

"Meanwhile," quoth Dick, "we have got to meet the rest of our day's
journey; so out into the street and down to the strand! Will you come a
turn with us, neighbour? Our friend is greedy of your stories."

"I will go as far as Oxford with you," said he; "I want a book or two
out of the Bodleian Library. I suppose you will sleep in the old city?"

"No," said Dick, "we are going higher up; the hay is waiting us there,
you know."

Morsom nodded, and we all went into the street together, and got into
the boat a little above the town bridge. But just as Dick was getting
the sculls into the rowlocks, the bows of another boat came thrusting
through the low arch. Even at first sight it was a gay little craft
indeed---bright green, and painted over with elegantly drawn flowers. As
it cleared the arch, a figure as bright and gay-clad as the boat rose up
in it; a slim girl dressed in light blue silk that fluttered in the
draughty wind of the bridge. I thought I knew the figure, and sure
enough, as she turned her head to us, and showed her beautiful face, I
saw with joy that it was none other than the fairy godmother from the
abundant garden on Runnymede---Ellen, to wit.

We all stopped to receive her. Dick rose in the boat and cried out a
genial good morrow; I tried to be as genial as Dick, but failed; Clara
waved a delicate hand to her; and Morsom nodded and looked on with
interest. As to Ellen, the beautiful brown of her face was deepened by a
flush, as she brought the gunwale of her boat alongside ours, and said:

"You see, neighbours, I had some doubt if you would all three come back
past Runnymede, or if you did, whether you would stop there; and
besides, I am not sure whether we---my father and I---shall not be away
in a week or two, for he wants to see a brother of his in the north
country, and I should not like him to go without me. So I thought I
might never see you again, and that seemed uncomfortable to me,
and---and so I came after you."

"Well," said Dick, "I am sure we are all very glad of that; although you
may be sure that as for Clara and me, we should have made a point of
coming to see you, and of coming the second time, if we had found you
away the first. But, dear neighbour, there you are alone in the boat,
and you have been sculling pretty hard I should think, and might find a
little quiet sitting pleasant; so we had better part our company into
two."

"Yes," said Ellen, "I thought you would do that, so I have brought a
rudder for my boat: will you help me to ship it, please?"

And she went aft in her boat and pushed along our side till she had
brought the stern close to Dick's hand. He knelt down in our boat and
she in hers, and the usual fumbling took place over hanging the rudder
on its hooks; for, as you may imagine, no change had taken place in the
arrangement of such an unimportant matter as the rudder of a
pleasure-boat. As the two beautiful young faces bent over the rudder,
they seemed to me to be very close together, and though it only lasted a
moment, a sort of pang shot through me as I looked on. Clara sat in her
place and did not look round, but presently she said, with just the
least stiffness in her tone:

"How shall we divide? Won't you go into Ellen's boat, Dick, since,
without offence to our guest, you are the better sculler?"

Dick stood up and laid his hand on her shoulder, and said: "No, no; let
Guest try what he can do---he ought to be getting into training now.
Besides, we are in no hurry: we are not going far above Oxford; and even
if we are benighted, we shall have the moon, which will give us nothing
worse of a night than a greyer day."

"Besides," said I, "I may manage to do a little more with my sculling
than merely keeping the boat from drifting down stream."

They all laughed at this, as if it had a been very good joke; and I
thought that Ellen's laugh, even amongst the others, was one of the
pleasantest sounds I had ever heard.

To be short, I got into the new-come boat, not a little elated, and
taking the sculls, set to work to show off a little. For---must I say
it?---I felt as if even that happy world were made the happier for my
being so near this strange girl; although I must say that of all the
persons I had seen in that world renewed, she was the most unfamiliar to
me, the most unlike what I could have thought of. Clara, for instance,
beautiful and bright as she was, was not unlike a \emph{very} pleasant
and unaffected young lady; and the other girls also seemed nothing more
than specimens of very much improved types which I had known in other
times. But this girl was not only beautiful with a beauty quite
different from that of "a young lady," but was in all ways so strangely
interesting; so that I kept wondering what she would say or do next to
surprise and please me. Not, indeed, that there was anything startling
in what she actually said or did; but it was all done in a new way, and
always with that indefinable interest and pleasure of life, which I had
noticed more or less in everybody, but which in her was more marked and
more charming than in anyone else that I had seen.

We were soon under way and going at a fair pace through the beautiful
reaches of the river, between Bensington and Dorchester. It was now
about the middle of the afternoon, warm rather than hot, and quite
windless; the clouds high up and light, pearly white, and gleaming,
softened the sun's burning, but did not hide the pale blue in most
places, though they seemed to give it height and consistency; the sky,
in short, looked really like a vault, as poets have sometimes called it,
and not like mere limitless air, but a vault so vast and full of light
that it did not in any way oppress the spirits. It was the sort of
afternoon that Tennyson must have been thinking about, when he said of
the Lotos-Eaters' land that it was a land where it was always afternoon.

Ellen leaned back in the stern and seemed to enjoy herself thoroughly. I
could see that she was really looking at things and let nothing escape
her, and as I watched her, an uncomfortable feeling that she had been a
little touched by love of the deft, ready, and handsome Dick, and that
she had been constrained to follow us because of it, faded out of my
mind; since if it had been so, she surely could not have been so
excitedly pleased, even with the beautiful scenes we were passing
through. For some time she did not say much, but at last, as we had
passed under Shillingford Bridge (new built, but somewhat on its old
lines), she bade me hold the boat while she had a good look at the
landscape through the graceful arch. Then she turned about to me and
said:

"I do not know whether to be sorry or glad that this is the first time
that I have been in these reaches. It is true that it is a great
pleasure to see all this for the first time; but if I had had a year or
two of memory of it, how sweetly it would all have mingled with my life,
waking or dreaming! I am so glad Dick has been pulling slowly, so as to
linger out the time here. How do you feel about your first visit to
these waters?"

I do not suppose she meant a trap for me, but anyhow I fell into it, and
said: "My first visit! It is not my first visit by many a time. I know
these reaches well; indeed, I may say that I know every yard of the
Thames from Hammersmith to Cricklade."

I saw the complications that might follow, as her eyes fixed mine with a
curious look in them, that I had seen before at Runnymede, when I had
said something which made it difficult for others to understand my
present position amongst these people. I reddened, and said, in order to
cover my mistake: "I wonder you have never been up so high as this,
since you live on the Thames, and moreover row so well that it would be
no great labour to you. Let alone," quoth I, insinuatingly, "that
anybody would be glad to row you."

She laughed, clearly not at my compliment (as I am sure she need not
have done, since it was a very commonplace fact), but at something which
was stirring in her mind; and she still looked at me kindly, but with
the above-said keen look in her eyes, and then she said:

"Well, perhaps it is strange, though I have a good deal to do at home,
what with looking after my father, and dealing with two or three young
men who have taken a special liking to me, and all of whom I cannot
please at once. But you, dear neighbour; it seems to me stranger that
you should know the upper river, than that I should not know it; for, as
I understand, you have only been in England a few days. But perhaps you
mean that you have read about it in books, and seen pictures of
it?---though that does not come to much, either."

"Truly," said I. "Besides, I have not read any books about the Thames:
it was one of the minor stupidities of our time that no one thought fit
to write a decent book about what may fairly be called our only English
river."

The words were no sooner out of my mouth than I saw that I had made
another mistake; and I felt really annoyed with myself, as I did not
want to go into a long explanation just then, or begin another series of
Odyssean lies. Somehow, Ellen seemed to see this, and she took no
advantage of my slip; her piercing look changed into one of mere frank
kindness, and she said:

"Well, anyhow I am glad that I am travelling these waters with you,
since you know our river so well, and I know little of it past
Pangbourne, for you can tell me all I want to know about it." She paused
a minute, and then said: "Yet you must understand that the part I do
know, I know as thoroughly as you do. I should be sorry for you to think
that I am careless of a thing so beautiful and interesting as the
Thames."

She said this quite earnestly, and with an air of affectionate appeal to
me which pleased me very much; but I could see that she was only keeping
her doubts about me for another time.

Presently we came to Day's Lock, where Dick and his two sitters had
waited for us. He would have me go ashore, as if to show me something
which I had never seen before; and nothing loth I followed him, Ellen by
my side, to the well-remembered Dykes, and the long church beyond them,
which was still used for various purposes by the good folk of
Dorchester: where, by the way, the village guest-house still had the
sign of the Fleur-de-luce which it used to bear in the days when
hospitality had to be bought and sold. This time, however, I made no
sign of all this being familiar to me: though as we sat for a while on
the mound of the Dykes looking up at Sinodun and its clear-cut trench,
and its sister \emph{mamelon} of Whittenham, I felt somewhat
uncomfortable under Ellen's serious attentive look, which almost drew
from me the cry, "How little anything is changed here!"

We stopped again at Abingdon, which, like Wallingford, was in a way both
old and new to me, since it had been lifted out of its
nineteenth-century degradation, and otherwise was as little altered as
might be.

Sunset was in the sky as we skirted Oxford by Oseney; we stopped a
minute or two hard by the ancient castle to put Henry Morsom ashore. It
was a matter of course that so far as they could be seen from the river,
I missed none of the towers and spires of that once don-beridden city;
but the meadows all round, which, when I had last passed through them,
were getting daily more and more squalid, more and more impressed with
the seal of the "stir and intellectual life of the nineteenth century,"
were no longer intellectual, but had once again become as beautiful as
they should be, and the little hill of Hinksey, with two or three very
pretty stone houses new-grown on it (I use the word advisedly; for they
seemed to belong to it) looked down happily on the full streams and
waving grass, grey now, but for the sunset, with its fast-ripening
seeds.

The railway having disappeared, and therewith the various level bridges
over the streams of Thames, we were soon through Medley Lock and in the
wide water that washes Port Meadow, with its numerous population of
geese nowise diminished; and I thought with interest how its name and
use had survived from the older imperfect communal period, through the
time of the confused struggle and tyranny of the rights of property,
into the present rest and happiness of complete Communism.

I was taken ashore again at Godstow, to see the remains of the old
nunnery, pretty nearly in the same condition as I had remembered them;
and from the high bridge over the cut close by, I could see, even in the
twilight, how beautiful the little village with its grey stone houses
had become; for we had now come into the stone-country, in which every
house must be either built, walls and roof, of grey stone or be a blot
on the landscape.

We still rowed on after this, Ellen taking the sculls in my boat; we
passed a weir a little higher up, and about three miles beyond it came
by moonlight again to a little town, where we slept at a house thinly
inhabited, as its folk were mostly tented in the hay-fields.
