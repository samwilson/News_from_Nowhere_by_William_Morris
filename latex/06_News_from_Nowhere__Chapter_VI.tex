As he spoke, we came suddenly out of the woodland into a short street of
handsomely built houses, which my companion named to me at once as
Piccadilly: the lower part of these I should have called shops, if it
had not been that, as far as I could see, the people were ignorant of
the arts of buying and selling. Wares were displayed in their finely
designed fronts, as if to tempt people in, and people stood and looked
at them, or went in and came out with parcels under their arms, just
like the real thing. On each side of the street ran an elegant arcade to
protect foot-passengers, as in some of the old Italian cities. About
halfway down, a huge building of the kind I was now prepared to expect
told me that this also was a centre of some kind, and had its special
public buildings.

Said Dick: "Here, you see, is another market on a different plan from
most others: the upper stories of these houses are used for
guest-houses; for people from all about the country are apt to drift up
hither from time to time, as folk are very thick upon the ground, which
you will see evidence of presently, and there are people who are fond of
crowds, though I can't say that I am."

I couldn't help smiling to see how long a tradition would last. Here was
the ghost of London still asserting itself as a centre,---an
intellectual centre, for aught I knew. However, I said nothing, except
that I asked him to drive very slowly, as the things in the booths
looked exceedingly pretty.

"Yes," said he, "this is a very good market for pretty things, and is
mostly kept for the handsomer goods, as the Houses-of-Parliament market,
where they set out cabbages and turnips and such like things, along with
beer and the rougher kind of wine, is so near."

Then he looked at me curiously, and said, "Perhaps you would like to do
a little shopping, as 'tis called."

I looked at what I could see of my rough blue duds, which I had plenty
of opportunity of contrasting with the gay attire of the citizens we had
come across; and I thought that if, as seemed likely, I should presently
be shown about as a curiosity for the amusement of this most
unbusinesslike people, I should like to look a little less like a
discharged ship's purser. But in spite of all that had happened, my hand
went down into my pocket again, where to my dismay it met nothing
metallic except two rusty old keys, and I remembered that amidst our
talk in the guest-hall at Hammersmith I had taken the cash out of my
pocket to show to the pretty Annie, and had left it lying there. My face
fell fifty per cent., and Dick, beholding me, said rather sharply---

"Hilloa, Guest! what's the matter now? Is it a wasp?"

"No," said I, "but I've left it behind."

"Well," said he, "whatever you have left behind, you can get in this
market again, so don't trouble yourself about it."

I had come to my senses by this time, and remembering the astounding
customs of this country, had no mind for another lecture on social
economy and the Edwardian coinage; so I said only---

"My clothes---Couldn't I? You see---What do think could be done about
them?"

He didn't seem in the least inclined to laugh, but said quite gravely:

"O don't get new clothes yet. You see, my great-grandfather is an
antiquarian, and he will want to see you just as you are. And, you know,
I mustn't preach to you, but surely it wouldn't be right for you to take
away people's pleasure of studying your attire, by just going and making
yourself like everybody else. You feel that, don't you?" said he,
earnestly.

I did \emph{not} feel it my duty to set myself up for a scarecrow amidst
this beauty-loving people, but I saw I had got across some ineradicable
prejudice, and that it wouldn't do to quarrel with my new friend. So I
merely said, "O certainly, certainly."

"Well," said he, pleasantly, "you may as well see what the inside of
these booths is like: think of something you want."

Said I: "Could I get some tobacco and a pipe?"

"Of course," said he; "what was I thinking of, not asking you before?
Well, Bob is always telling me that we non-smokers are a selfish lot,
and I'm afraid he is right. But come along; here is a place just handy."

Therewith he drew rein and jumped down, and I followed. A very handsome
woman, splendidly clad in figured silk, was slowly passing by, looking
into the windows as she went. To her quoth Dick: "Maiden, would you
kindly hold our horse while we go in for a little?" She nodded to us
with a kind smile, and fell to patting the horse with her pretty hand.

"What a beautiful creature!" said I to Dick as we entered.

"What, old Greylocks?" said he, with a sly grin.

"No, no," said I; "Goldylocks,---the lady."

"Well, so she is," said he. "'Tis a good job there are so many of them
that every Jack may have his Jill: else I fear that we should get
fighting for them. Indeed," said he, becoming very grave, "I don't say
that it does not happen even now, sometimes. For you know love is not a
very reasonable thing, and perversity and self-will are commoner than
some of our moralist's think." He added, in a still more sombre tone:
"Yes, only a month ago there was a mishap down by us, that in the end
cost the lives of two men and a woman, and, as it were, put out the
sunlight for us for a while. Don't ask me about it just now; I may tell
you about it later on."

By this time we were within the shop or booth, which had a counter, and
shelves on the walls, all very neat, though without any pretence of
showiness, but otherwise not very different to what I had been used to.
Within were a couple of children---a brown-skinned boy of about twelve,
who sat reading a book, and a pretty little girl of about a year older,
who was sitting also reading behind the counter; they were obviously
brother and sister.

"Good morning, little neighbours," said Dick. "My friend here wants
tobacco and a pipe; can you help him?"

"O yes, certainly," said the girl with a sort of demure alertness which
was somewhat amusing. The boy looked up, and fell to staring at my
outlandish attire, but presently reddened and turned his head, as if he
knew that he was not behaving prettily.

"Dear neighbour," said the girl, with the most solemn countenance of a
child playing at keeping shop, "what tobacco is it you would like?"

"Latakia," quoth I, feeling as if I were assisting at a child's game,
and wondering whether I should get anything but make-believe.

But the girl took a dainty little basket from a shelf beside her, went
to a jar, and took out a lot of tobacco and put the filled basket down
on the counter before me, where I could both smell and see that it was
excellent Latakia.

"But you haven't weighed it," said I, "and---and how much am I to take?"

"Why," she said, "I advise you to cram your bag, because you may be
going where you can't get Latakia. Where is your bag?"

I fumbled about, and at last pulled out my piece of cotton print which
does duty with me for a tobacco pouch. But the girl looked at it with
some disdain, and said---

"Dear neighbour, I can give you something much better than that cotton
rag." And she tripped up the shop and came back presently, and as she
passed the boy whispered something in his ear, and he nodded and got up
and went out. The girl held up in her finger and thumb a red morocco
bag, gaily embroidered, and said, "There, I have chosen one for you, and
you are to have it: it is pretty, and will hold a lot."

Therewith she fell to cramming it with the tobacco, and laid it down by
me and said, "Now for the pipe: that also you must let me choose for
you; there are three pretty ones just come in."

She disappeared again, and came back with a big-bowled pipe in her hand,
carved out of some hard wood very elaborately, and mounted in gold
sprinkled with little gems. It was, in short, as pretty and gay a toy as
I had ever seen; something like the best kind of Japanese work, but
better.

"Dear me!" said I, when I set eyes on it, "this is altogether too grand
for me, or for anybody but the Emperor of the World. Besides, I shall
lose it: I always lose my pipes."

The child seemed rather dashed, and said, "Don't you like it,
neighbour?"

"O yes," I said, "of course I like it."

"Well, then, take it," said she, "and don't trouble about losing it.
What will it matter if you do? Somebody is sure to find it, and he will
use it, and you can get another."

I took it out of her hand to look at it, and while I did so, forgot my
caution, and said, "But however am I to pay for such a thing as this?"

Dick laid his hand on my shoulder as I spoke, and turning I met his eyes
with a comical expression in them, which warned me against another
exhibition of extinct commercial morality; so I reddened and held my
tongue, while the girl simply looked at me with the deepest gravity, as
if I were a foreigner blundering in my speech, for she clearly didn't
understand me a bit.

"Thank you so very much," I said at last, effusively, as I put the pipe
in my pocket, not without a qualm of doubt as to whether I shouldn't
find myself before a magistrate presently.

"O, you are so very welcome," said the little lass, with an affectation
of grown-up manners at their best which was very quaint. "It is such a
pleasure to serve dear old gentlemen like you; especially when one can
see at once that you have come from far over sea."

"Yes, my dear," quoth I, "I have been a great traveller."

As I told this lie from pure politeness, in came the lad again, with a
tray in his hands, on which I saw a long flask and two beautiful
glasses. "Neighbours," said the girl (who did all the talking, her
brother being very shy, clearly) "please to drink a glass to us before
you go, since we do not have guests like this every day."

Therewith the boy put the tray on the counter and solemnly poured out a
straw-coloured wine into the long bowls. Nothing loth, I drank, for I
was thirsty with the hot day; and thinks I, I am yet in the world, and
the grapes of the Rhine have not yet lost their flavour; for if ever I
drank good Steinberg, I drank it that morning; and I made a mental note
to ask Dick how they managed to make fine wine when there were no longer
labourers compelled to drink rot-gut instead of the fine wine which they
themselves made.

"Don't you drink a glass to us, dear little neighbours?" said I.

"I don't drink wine," said the lass; "I like lemonade better: but I wish
your health!"

"And I like ginger-beer better," said the little lad.

Well, well, thought I, neither have children's tastes changed much. And
therewith we gave them good day and went out of the booth.

To my disappointment, like a change in a dream, a tall old man was
holding our horse instead of the beautiful woman. He explained to us
that the maiden could not wait, and that he had taken her place; and he
winked at us and laughed when he saw how our faces fell, so that we had
nothing for it but to laugh also---

"Where are you going?" said he to Dick.

"To Bloomsbury," said Dick.

"If you two don't want to be alone, I'll come with you," said the old
man.

"All right," said Dick, "tell me when you want to get down and I'll stop
for you. Let's get on."

So we got under way again; and I asked if children generally waited on
people in the markets. "Often enough," said he, "when it isn't a matter
of dealing with heavy weights, but by no means always. The children like
to amuse themselves with it, and it is good for them, because they
handle a lot of diverse wares and get to learn about them, how they are
made, and where they come from, and so on. Besides, it is such very easy
work that anybody can do it. It is said that in the early days of our
epoch there were a good many people who were hereditarily afflicted with
a disease called Idleness, because they were the direct descendants of
those who in the bad times used to force other people to work for
them---the people, you know, who are called slave-holders or employers
of labour in the history books. Well, these Idleness-stricken people
used to serve booths \emph{all} their time, because they were fit for so
little. Indeed, I believe that at one time they were actually
\emph{compelled} to do some such work, because they, especially the
women, got so ugly and produced such ugly children if their disease was
not treated sharply, that the neighbours couldn't stand it. However, I'm
happy to say that all that is gone by now; the disease is either
extinct, or exists in such a mild form that a short course of aperient
medicine carries it off. It is sometimes called the Blue-devils now, or
the Mulleygrubs. Queer names, ain't they?"

"Yes," said I, pondering much. But the old man broke in:

"Yes, all that is true, neighbour; and I have seen some of those poor
women grown old. But my father used to know some of them when they were
young; and he said that they were as little like young women as might
be: they had hands like bunches of skewers, and wretched little arms
like sticks; and waists like hour-glasses, and thin lips and peaked
noses and pale cheeks; and they were always pretending to be offended at
anything you said or did to them. No wonder they bore ugly children, for
no one except men like them could be in love with them---poor things!"

He stopped, and seemed to be musing on his past life, and then said:

"And do you know, neighbours, that once on a time people were still
anxious about that disease of Idleness: at one time we gave ourselves a
great deal of trouble in trying to cure people of it. Have you not read
any of the medical books on the subject?"

"No," said I; for the old man was speaking to me.

"Well," said he, "it was thought at the time that it was the survival of
the old mediaeval disease of leprosy: it seems it was very catching, for
many of the people afflicted by it were much secluded, and were waited
upon by a special class of diseased persons queerly dressed up, so that
they might be known. They wore amongst other garments, breeches made of
worsted velvet, that stuff which used to be called plush some years
ago."

All this seemed very interesting to me, and I should like to have made
the old man talk more. But Dick got rather restive under so much ancient
history: besides, I suspect he wanted to keep me as fresh as he could
for his great-grandfather. So he burst out laughing at last, and said:
"Excuse me, neighbours, but I can't help it. Fancy people not liking to
work!---it's too ridiculous. Why, even you like to work, old
fellow---sometimes," said he, affectionately patting the old horse with
the whip. "What a queer disease! it may well be called Mulleygrubs!"

And he laughed out again most boisterously; rather too much so, I
thought, for his usual good manners; and I laughed with him for
company's sake, but from the teeth outward only; for \emph{I} saw
nothing funny in people not liking to work, as you may well imagine.
