As I spoke, I heard footsteps near the door; the latch yielded, and in
came our two lovers, looking so handsome that one had no feeling of
shame in looking on at their little-concealed love-making; for indeed it
seemed as if all the world must be in love with them. As for old
Hammond, he looked on them like an artist who has just painted a picture
nearly as well as he thought he could when he began it, and was
perfectly happy. He said:

"Sit down, sit down, young folk, and don't make a noise. Our guest here
has still some questions to ask me."

"Well, I should suppose so," said Dick; "you have only been three hours
and a half together; and it isn't to be hoped that the history of two
centuries could be told in three hours and a half: let alone that, for
all I know, you may have been wandering into the realms of geography and
craftsmanship."

"As to noise, my dear kinsman," said Clara, "you will very soon be
disturbed by the noise of the dinner-bell, which I should think will be
very pleasant music to our guest, who breakfasted early, it seems, and
probably had a tiring day yesterday."

I said: "Well, since you have spoken the word, I begin to feel that it
is so; but I have been feeding myself with wonder this long time past:
really, it's quite true," quoth I, as I saw her smile, O so prettily!
But just then from some tower high up in the air came the sound of
silvery chimes playing a sweet clear tune, that sounded to my
unaccustomed ears like the song of the first blackbird in the spring,
and called a rush of memories to my mind, some of bad times, some of
good, but all sweetened now into mere pleasure.

"No more questions now before dinner," said Clara; and she took my hand
as an affectionate child would, and led me out of the room and down
stairs into the forecourt of the Museum, leaving the two Hammonds to
follow as they pleased.

We went into the market-place which I had been in before, a thinnish
stream of
elegantly\textsuperscript{\protect\hyperlink{cite_note-1}{{[}1{]}}}
dressed people going in along with us. We turned into the cloister and
came to a richly moulded and carved doorway, where a very pretty
dark-haired young girl gave us each a beautiful bunch of summer flowers,
and we entered a hall much bigger than that of the Hammersmith Guest
House, more elaborate in its architecture and perhaps more beautiful. I
found it difficult to keep my eyes off the wall-pictures (for I thought
it bad manners to stare at Clara all the time, though she was quite
worth it). I saw at a glance that their subjects were taken from queer
old-world myths and imaginations which in yesterday's world only about
half a dozen people in the country knew anything about; and when the two
Hammonds sat down opposite to us, I said to the old man, pointing to the
frieze:

"How strange to see such subjects here!"

"Why?" said he. "I don't see why you should be surprised; everybody
knows the tales; and they are graceful and pleasant subjects, not too
tragic for a place where people mostly eat and drink and amuse
themselves, and yet full of incident."

I smiled, and said: "Well, I scarcely expected to find record of the
Seven Swans and the King of the Golden Mountain and Faithful Henry, and
such curious pleasant imaginations as Jacob Grimm got together from the
childhood of the world, barely lingering even in his time: I should have
thought you would have forgotten such childishness by this time."

The old man smiled, and said nothing; but Dick turned rather red, and
broke out:

"What \emph{do} you mean, guest? I think them very beautiful, I mean not
only the pictures, but the stories; and when we were children we used to
imagine them going on in every wood-end, by the bight of every stream:
every house in the fields was the Fairyland King's House to us. Don't
you remember, Clara?"

"Yes," she said; and it seemed to me as if a slight cloud came over her
fair face. I was going to speak to her on the subject, when the pretty
waitresses came to us smiling, and chattering sweetly like reed warblers
by the river side, and fell to giving us our dinner. As to this, as at
our breakfast, everything was cooked and served with a daintiness which
showed that those who had prepared it were interested in it; but there
was no excess either of quantity or of gourmandise; everything was
simple, though so excellent of its kind; and it was made clear to us
that this was no feast, only an ordinary meal. The glass, crockery, and
plate were very beautiful to my eyes, used to the study of mediaeval
art; but a nineteenth-century club-haunter would, I daresay, have found
them rough and lacking in finish; the crockery being lead-glazed
pot-ware, though beautifully ornamented; the only porcelain being here
and there a piece of old oriental ware. The glass, again, though elegant
and quaint, and very varied in form, was somewhat bubbled and hornier in
texture than the commercial articles of the nineteenth century. The
furniture and general fittings of the ball were much of a piece with the
table-gear, beautiful in form and highly ornamented, but without the
commercial "finish" of the joiners and cabinet-makers of our time.
Withal, there was a total absence of what the nineteenth century calls
"comfort"---that is, stuffy inconvenience; so that, even apart from the
delightful excitement of the day, I had never eaten my dinner so
pleasantly before.

When we had done eating, and were sitting a little while, with a bottle
of very good Bordeaux wine before us, Clara came back to the question of
the subject-matter of the pictures, as though it had troubled her.

She looked up at them, and said: "How is it that though we are so
interested with our life for the most part, yet when people take to
writing poems or painting pictures they seldom deal with our modern
life, or if they do, take good care to make their poems or pictures
unlike that life? Are we not good enough to paint ourselves? How is it
that we find the dreadful times of the past so interesting to us---in
pictures and poetry?"

Old Hammond smiled. "It always was so, and I suppose always will be,"
said he, "however it may be explained. It is true that in the nineteenth
century, when there was so little art and so much talk about it, there
was a theory that art and imaginative literature ought to deal with
contemporary life; but they never did so; for, if there was any pretence
of it, the author always took care (as Clara hinted just now) to
disguise, or exaggerate, or idealise, and in some way or another make it
strange; so that, for all the verisimilitude there was, he might just as
well have dealt with the times of the Pharaohs."

"Well," said Dick, "surely it is but natural to like these things
strange; just as when we were children, as I said just now, we used to
pretend to be so-and-so in such-and-such a place. That's what these
pictures and poems do; and why shouldn't they?"

"Thou hast hit it, Dick," quoth old Hammond; "it is the child-like part
of us that produces works of imagination. When we are children time
passes so slow with us that we seem to have time for everything."

He sighed, and then smiled and said: "At least let us rejoice that we
have got back our childhood again. I drink to the days that are!"

"Second childhood," said I in a low voice, and then blushed at my double
rudeness, and hoped that he hadn't heard. But he had, and turned to me
smiling, and said: "Yes, why not? And for my part, I hope it may last
long; and that the world's next period of wise and unhappy manhood, if
that should happen, will speedily lead us to a third childhood: if
indeed this age be not our third. Meantime, my friend, you must know
that we are too happy, both individually and collectively, to trouble
ourselves about what is to come hereafter."

"Well, for my part," said Clara, "I wish we were interesting enough to
be written or painted about."

Dick answered her with some lover's speech, impossible to be written
down, and then we sat quiet a little.

\hypertarget{footnotesedit}{%
\subsection[{{{[}}\href{/w/index.php?title=News_from_Nowhere/Chapter_XVI\&action=edit\&section=1}{edit}{{]}}}]{\texorpdfstring{\protect\hypertarget{Footnotes}{}{Footnotes}{{{[}}\href{/w/index.php?title=News_from_Nowhere/Chapter_XVI\&action=edit\&section=1}{edit}{{]}}}}{Footnotes{[}edit{]}}}\label{footnotesedit}}

\begin{enumerate}
\tightlist
\item
  \protect\hypertarget{cite_note-1}{}{{\protect\hyperlink{cite_ref-1}{↑}}
  {"Elegant," I mean, as a Persian pattern is elegant; not like a rich
  "elegant" lady out for a morning call. I should rather call that
  genteel.}}
\end{enumerate}
